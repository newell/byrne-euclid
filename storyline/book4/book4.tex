\pagestyle{fancy}
\fancyhf{}
\renewcommand{\headrulewidth}{0pt}
\fancyfoot[LO,RE]{\textsc{IV.} \thepage}%

\begin{minipage}{0.165\textwidth}
    \phantom{}
\end{minipage}%
\begin{minipage}{0.67\textwidth}
    \section[Book IV]{\centering BOOK IV.}
    \label{sec:book4}

    \hfill

    \subsection[Definitions]{\centering \scshape{\LARGE{DEFINITIONS.}}}
    \label{subsec:definitions}
\end{minipage}%
\begin{minipage}{0.165\textwidth}
    \phantom{}
\end{minipage}%

\hfill

\begin{center}
    I.\phantomsection\label{book4def1}\\
\end{center}
\begin{minipage}{0.67\textwidth}
    \subsubsection{def. 1}
    \begin{center}
        \raggedright \lettrine[lines=3, loversize=1, nindent=0pt]{\euclidinitials{A}}{} rectilinear figure is ſaid to be \textit{inſcribed} in another, when all the angular points of the inſcribed figure are on the ſides of the figure in which it is ſaid to be inſcribed.
    \end{center}
\end{minipage}%
\begin{minipage}{0.33\textwidth}
    \begin{center}
        $\img[0][0][60]{book4_definition_1_figure}$
    \end{center}
\end{minipage}%

\hfill

\begin{minipage}{0.1\textwidth}
    \phantom{}
\end{minipage}%
\begin{minipage}{0.80\textwidth}
    \subsubsection{def. 2}
    \begin{center}
        II.\phantomsection\label{book4def2}\\
        \hfill\\
        \raggedright A \textsc{FIGURE} is ſaid to be \textit{deſcribed about} another figure, when all the ſides of the circumſcribed figure paſs through the angular points of the other figure.
    \end{center}
\end{minipage}
\begin{minipage}{0.1\textwidth}
    \phantom{}
\end{minipage}%

\hfill

\begin{center}
    III.\phantomsection\label{book4def3}\\
\end{center}
\begin{minipage}{0.67\textwidth}
    \subsubsection{def. 3}
    \begin{center}
        \raggedright A \textsc{RECTILINEAR} figure is ſaid to be \textit{inſcribed in} a circle, when the vertex of each angle of the figure is in the circumference of the circle.
    \end{center}
\end{minipage}%
\begin{minipage}{0.33\textwidth}
    \begin{center}
        $\img[0][0][60]{book4_definition_3_figure}$
    \end{center}
\end{minipage}%

\vspace{\baselineskip}

\begin{center}
    IV.\phantomsection\label{book4def4}\\
\end{center}
\begin{minipage}{0.67\textwidth}
    \subsubsection{def. 4}
    \begin{center}
        \raggedright A \textsc{RECTILINEAR} figure is ſaid to be \textit{circumſcribed about} a circle, when each of its ſides is a tangent to the circle.
    \end{center}
\end{minipage}%
\begin{minipage}{0.33\textwidth}
    \begin{center}
        $\img[0][0][60]{book4_definition_4_figure}$
    \end{center}
\end{minipage}%

\newpage

\begin{center}
    V.\phantomsection\label{book4def5}\\
\end{center}
\begin{minipage}{0.33\textwidth}
    \begin{center}
        $\img[0][0][60]{book4_definition_5_figure}$
    \end{center}
\end{minipage}%
\begin{minipage}{0.67\textwidth}
    \subsubsection{def. 5}
    \begin{center}
        \raggedright A \textsc{CIRCLE} is ſaid to be inſcribed in a rectilinear figure, when each ſide of the figure is a tangent to the circle.
    \end{center}
\end{minipage}%

\hfill

\begin{center}
    VI.\phantomsection\label{book4def6}\\
\end{center}
\begin{minipage}{0.33\textwidth}
    \begin{center}
        $\img[0][0][70]{book4_definition_6_figure}$
    \end{center}
\end{minipage}%
\begin{minipage}{0.67\textwidth}
    \subsubsection{def. 6}
    \begin{center}
        \raggedright A \textsc{CIRCLE} is ſaid to be \textit{circumſcribed about} a rectilinear figure, when the circumference paſſes through the vertex of each angle of the figure.
    \end{center}
\end{minipage}%
\begin{center}
    $\img{triangle_14}$ is circumſcribed.
\end{center}

\hfill

\begin{center}
    VII.\phantomsection\label{book4def7}\\
\end{center}
\begin{minipage}{0.33\textwidth}
    \begin{center}
        $\img[0][0][60]{book4_definition_7_figure}$
    \end{center}
\end{minipage}%
\begin{minipage}{0.67\textwidth}
    \subsubsection{def. 7}
    \begin{center}
        \raggedright A \textsc{STRAIGHT} line is ſaid to be \textit{inſcribe in} a circle, when its extremities are in the circumference.
    \end{center}
\end{minipage}%

\hfill

\begin{minipage}{0.165\textwidth}
    \phantom{}
\end{minipage}%
\begin{minipage}{0.67\textwidth}
    \raggedright \textit{The Fourth Book of the Elements is devoted to the ſolution of problems, chiefly relating to the inſcription and circumſcription of regular polygons and circles}.\\
    \hfill\\
    A regular polygon is one whoſe angles and ſides are equal.
\end{minipage}
\begin{minipage}{0.165\textwidth}
    \phantom{}
\end{minipage}%

\newpage

\subsection[Propositions]{\centering \scshape{\LARGE{PROPOSITIONS.}}}
\label{subsec:propositions}

\iconsectioninToC
% Propositions
\foreach \c in {1,...,16}{
        \input{book4/prop\c.tex}
        \newpage
    }
\stdsectioninToC
