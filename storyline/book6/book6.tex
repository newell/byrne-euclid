\pagestyle{fancy}
\fancyhf{}
\renewcommand{\headrulewidth}{0pt}
\fancyfoot[LE,RO]{\textsc{VI.} \thepage}%

\begin{minipage}{0.67\textwidth}
    \section[Book VI]{\centering \textcolor{black}{BOOK VI.}}
    \label{sec:book6}

    \hfill

    \subsection[Definitions]{\centering \scshape{\LARGE{DEFINITIONS.}}}
    \label{subsec:definitions}
\end{minipage}

\hfill

\begin{minipage}{0.67\textwidth}
    \subsubsection{def. 1}
    \begin{center}
        I.\phantomsection\label{book6def1}\\
        \hfill\\
        \raggedright \lettrine[lines=3, loversize=1, nindent=0pt]{\euclidinitials{R}}{}ECTILINEAR figures are ſaid to be ſimilar, when they have their ſeveral angles equal, each to each, and the ſides about the equal angles proportional.
    \end{center}
\end{minipage}%
\begin{minipage}{0.33\textwidth}
    \begin{center}
        $\img[-13][0][75]{book6_definition_1_figure}$
    \end{center}
\end{minipage}

\hfill

\begin{minipage}{0.67\textwidth}
    \subsubsection{def. 2}
    \begin{center}
        II.\phantomsection\label{book6def2}\\
        \hfill\\
        \raggedright{T\textsc{WO} ſides of one figure are ſaid to be reciprocally proportional to two ſides of another figure when one of the ſides of the firſt is to the ſecond, as the remaining ſide of the ſecond is to the remaining ſide of the firſt.}
        \hfill\\
        \hfill\\
        \centering
        \subsubsection{def. 3}
        III.\phantomsection\label{book6def3}\\
        \hfill\\
        \raggedright{A \textsc{STRAIGHT} line is ſaid to be cut in extreme and mean ratio, when the whole is to the greater ſegment, as the greater ſegment is to the leſs.}
    \end{center}
\end{minipage}

\hfill

\begin{minipage}{0.33\textwidth}
    \phantom{}
\end{minipage}%
\begin{minipage}{0.67\textwidth}
    \subsubsection{def. 4}
    \begin{center}
        IV.\phantomsection\label{book6def4}\\
        \hfill\\
        \raggedright T\textsc{HE} altitude of any figure is the ſtraight line drawn from its vertex perpendicular to its baſe, or the baſe produced.
    \end{center}
\end{minipage}
\begin{center}
    $\img[0][0][300]{book6_definition_4_figure}$
\end{center}

\newpage

\begin{minipage}{0.46\textwidth}
    \phantom{}
\end{minipage}%
\begin{minipage}{0.54\textwidth}
    \subsection[Propositions]{\centering \scshape{\LARGE{PROPOSITIONS.}}}
    \label{subsec:propositions}
\end{minipage}

\hfill

\iconsectioninToC
% Propositions
\foreach \c in {1,...,33,A,B,C,D}{
        \input{book6/prop\c.tex}
        \newpage
    }
\stdsectioninToC


