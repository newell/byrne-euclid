% !TEX TS-program = xelatex
% !TEX options = -shell-escape -synctex=1 -interaction=nonstopmode -file-line-error "%DOC%"
\documentclass[11pt,preview]{standalone}

\usepackage{standalone}
\usepackage{fancyhdr}
\usepackage{xpatch}
\usepackage{titletoc}
\usepackage{setspace}
\usepackage{minibox}
\usepackage{enumitem}
\usepackage{fullpage}
\usepackage{mathtools,mathrsfs}
\usepackage{amssymb,amsthm}
\usepackage{graphicx,xcolor}
\usepackage[scale=2]{ccicons}
\usepackage[breakable]{tcolorbox}
\tcbuselibrary{breakable}
\usepackage{subfig}
\usepackage{float}
\usepackage{parskip}
\usepackage{lettrine}
\usepackage{fontspec}
\usepackage[compact]{titlesec}
\usepackage{calc}
\usepackage{xparse}
\usepackage{tikz}
\usepackage{svg}
\usepackage{wrapfig}
\usepackage{xr-hyper}
\usepackage[colorlinks=true, citecolor=violet, linkcolor=black, urlcolor=black]{hyperref}
\usepackage[
    papersize={6.25in,9.25in},
    margin=0.75in,
    layoutsize={6in,9in},
    layoutoffset={0.125in,0.125in},
    includehead,
    includefoot,
    showcrop,
]{geometry}

% Set this PATH to the root directory of your repository
\newcommand*{\MyPath}{/home/newell/code/byrne-euclid}%
% Don't show subsubsection titles -- these still show up in TOC (as desired)
\makeatletter
\titleformat{\subsubsection}[runin]{}{}{0pt}{\@gobble}
\makeatother

\pagestyle{fancy}
\fancyhf{}
\renewcommand{\headrulewidth}{0pt}
\fancyfoot[LO,RE]{\thepage}%

% TOC - image setup
\newcounter{propimage}
\makeatletter
\newcommand\stdsectioninToC{
    \titlecontents{subsubsection}[3.8em]
    {}%
    {\contentslabel{2.3em}}%
    {\hspace*{2.3em}}%
    {\titlerule*[1em]{.}\contentspage}
}
\newcommand\iconsectioninToC{
    \titlecontents{subsubsection}[3.8em]
    {\vskip 2ex}%
    {\hspace*{-2.3em}}%
    {
        \contentslabel{2.3em}%
        \stepcounter{propimage}%
        \smash{\includegraphics[width=50pt,height=25pt,keepaspectratio]{\MyPath/toc-images/image-\the\value{propimage}}}\hspace{0.5em}%
    }%
    {\titlerule*[1em]{.}\contentspage}%
}
\AtBeginDocument{\stdsectioninToC}
\makeatother

\setcounter{secnumdepth}{-3}
\setcounter{tocdepth}{3}

% Set the Font
\setmainfont{EBGaramond-Regular.otf}[
    BoldFont = EBGaramond-Bold.otf,
    ItalicFont = EBGaramond-Italic.otf,
    BoldItalicFont = EBGaramond-BoldItalic.otf,
    SmallCapsFont = EBGaramond12-AllSC.otf,
    Numbers = {OldStyle,Proportional},
    Ligatures = {Discretionary},
    RawFeature = {+ss06}
]
\newfontfamily\euclidinitials{EuclidInitialsnormal.ttf}

% Set the SVG paths (when I tried multiple lines to make this more readable it errored on me)
\svgpath{{\MyPath/symbols/}{\MyPath/byrne-euclid-svg/lines/}{\MyPath/byrne-euclid-svg/lines/points/}{\MyPath/byrne-euclid-svg/lines/arcs/}{\MyPath/byrne-euclid-svg/circles/}{\MyPath/byrne-euclid-svg/triangles/}{\MyPath/byrne-euclid-svg/figures/}{\MyPath/byrne-euclid-svg/figures/definitions/}{\MyPath/byrne-euclid-svg/figures/propositions/}{\MyPath/byrne-euclid-svg/angles/}{\MyPath/byrne-euclid-svg/symbols/}{\MyPath/byrne-euclid-svg/lines/}{\MyPath/byrne-euclid-svg/lines/points/}{\MyPath/byrne-euclid-svg/lines/arcs/}{\MyPath/byrne-euclid-svg/circles/}{\MyPath/byrne-euclid-svg/triangles/}{\MyPath/byrne-euclid-svg/figures/}{\MyPath/byrne-euclid-svg/figures/definitions/}{\MyPath/byrne-euclid-svg/figures/propositions/}{\MyPath/byrne-euclid-svg/angles/}{\MyPath/byrne-euclid-svg/icons/}}

% Create counter for lists.
\newcounter{listcounter}

% Colors
\definecolor{cred}{RGB}{212,42,32} % red
\definecolor{cyellow}{RGB}{250,194,43} % yellow
\definecolor{cblue}{RGB}{14,99,142} % blue
\definecolor{ctrans}{RGB}{252,243,217} % transparent
\definecolor{background}{HTML}{fcf3d9}

\newlength\mytemplena
\newlength\mytemplenb
\DeclareDocumentCommand\myalignalign{sm}
{
    \settowidth{\mytemplena}{$\displaystyle #2$}%
    \setlength\mytemplenb{\widthof{$\displaystyle=$}/2}%
    \hskip-\mytemplena%
    \hskip\IfBooleanTF#1{-\mytemplenb}{+\mytemplenb}%
}

% Images
\NewDocumentCommand{\img}{%
    O{0}
    O{0}
    O{25}
    O{0}
    m
}{%
    \begin{array}{c}
        \vspace{#1ex}\hspace{#2ex}\includesvg[width=#3pt]{#5}\hspace{#4ex}
    \end{array}
}

% Common Symbols
\newcommand*{\bplus}{\img[-0.8][0][10]{plus}}
\newcommand*{\bminus}{\img[0.8][0][10]{minus}}
\newcommand*{\bcross}{\img[-0.85][0][10]{cross}}
\newcommand*{\bequals}{\img[-0.1][0][10]{equals}}
\newcommand*{\bnequals}{\img[-0.85][0][10]{nequals}}
\newcommand*{\bparallel}{\img[-0.85][0][7]{parallel}}
\newcommand*{\bnparallel}{\img[-0.85][0][11]{not_parallel}}
\newcommand*{\bperiod}{\img[-0.5][-1][3]{period}}
\newcommand*{\bmark}{\img[-0.5][-1][3]{mark}}
\newcommand*{\bmultiply}{\img[1][0][3]{period}}
\newcommand*{\bsemicolon}{\img[-1.4][-1][3]{semicolon}}
\newcommand*{\bcolon}{\img[-0.5][0][3]{colon}}
\newcommand*{\bbcolon}{\bcolon\hspace{-1.5ex}\bcolon}
\newcommand*{\bcomma}{\img[-1.5][-1][3]{comma}}
\newcommand*{\btherefore}{\img[-0.5][0][10]{therefore}}
\newcommand*{\bgt}{\img[-0.85][0][10]{greater_than}}
\newcommand*{\blt}{\img[-0.85][0][10]{less_than}}
\newcommand*{\bperp}{\img[0][0][15]{perpendicular}}

% Common Lines
\newcommand*{\bimg}[1]{\img[0.8][0][45]{#1}}
\newcommand*{\redlines}{\bimg{red_lines}}
\newcommand*{\bluelines}{\bimg{blue_lines}}
\newcommand*{\blacklines}{\bimg{black_lines}}

\newcommand*{\redline}{\img[0.8][0][30]{red_line}}
\newcommand*{\redthinline}{\img[0.8][0][30]{red_thin_line}}
\newcommand*{\dottedredline}{\img[0.8][0][30]{red_dotted_line}}
\newcommand*{\blueline}{\img[0.8][0][30]{blue_line}}
\newcommand*{\bluethinline}{\img[0.8][0][30]{blue_thin_line}}
\newcommand*{\dottedblueline}{\img[0.8][0][30]{blue_dotted_line}}
\newcommand*{\blackline}{\img[0.8][0][30]{black_line}}
\newcommand*{\blackthinline}{\img[0.8][0][30]{black_thin_line}}
\newcommand*{\blackthindottedline}{\img[0.8][0][30]{black_thin_dotted_line}}
\newcommand*{\dottedblackline}{\img[0.8][0][30]{black_dotted_line}}
\newcommand*{\yellowline}{\img[0.8][0][30]{yellow_line}}
\newcommand*{\yellowthinline}{\img[0.8][0][30]{yellow_thin_line}}
\newcommand*{\dottedyellowline}{\img[0.8][0][30]{yellow_dotted_line}}

% Common Angles
\newcommand*{\tworightangles}{\img[0][0][30]{two_right_angles}}

% Common Shapes -- Book 5
\newcommand*{\reddome}{\img[-0.8][-0.5][10][-0.5]{red_dome_icon}}
\newcommand*{\bluedome}{\img[-0.8][-0.5][10][-0.5]{blue_dome_icon}}
\newcommand*{\blackdome}{\img[-0.8][-0.5][10][-0.5]{black_dome_icon}}
\newcommand*{\yellowdome}{\img[-0.8][-0.5][10][-0.5]{yellow_dome_icon}}

\newcommand*{\redcircle}{\img[-0.8][-0.5][10][-0.5]{red_circle_icon}}
\newcommand*{\bluecircle}{\img[-0.8][-0.5][10][-0.5]{blue_circle_icon}}
\newcommand*{\blackcircle}{\img[-0.8][-0.5][10][-0.5]{black_circle_icon}}
\newcommand*{\yellowcircle}{\img[-0.8][-0.5][10][-0.5]{yellow_circle_icon}}

\newcommand*{\reddrop}{\img[-0.8][-0.5][10][-0.5]{red_drop_icon}}
\newcommand*{\bluedrop}{\img[-0.8][-0.5][10][-0.5]{blue_drop_icon}}
\newcommand*{\blackdrop}{\img[-0.8][-0.5][10][-0.5]{black_drop_icon}}
\newcommand*{\yellowdrop}{\img[-0.8][-0.5][10][-0.5]{yellow_drop_icon}}

\newcommand*{\redhome}{\img[-0.8][-0.5][10][-0.5]{red_home_icon}}
\newcommand*{\bluehome}{\img[-0.8][-0.5][10][-0.5]{blue_home_icon}}
\newcommand*{\blackhome}{\img[-0.8][-0.5][10][-0.5]{black_home_icon}}
\newcommand*{\yellowhome}{\img[-0.8][-0.5][10][-0.5]{yellow_home_icon}}

\newcommand*{\redsquare}{\img[-0.8][-0.5][10][-0.5]{red_square_icon}}
\newcommand*{\smallredsquare}{\img[-0.8][-0.5][5][-0.5]{red_square_icon}}
\newcommand*{\bluesquare}{\img[-0.8][-0.5][10][-0.5]{blue_square_icon}}
\newcommand*{\bblacksquare}{\img[-0.8][-0.5][10][-0.5]{black_square_icon}}
\newcommand*{\yellowsquare}{\img[-0.8][-0.5][10][-0.5]{yellow_square_icon}}
\newcommand*{\transsquare}{\img[-0.8][-0.5][10][-0.5]{trans_square_icon}}

\newcommand*{\reddiamond}{\img[-0.8][-0.5][8][-0.5]{red_diamond_icon}}
\newcommand*{\bluediamond}{\img[-0.8][-0.5][8][-0.5]{blue_diamond_icon}}
\newcommand*{\blackdiamond}{\img[-0.8][-0.5][8][-0.5]{black_diamond_icon}}
\newcommand*{\yellowdiamond}{\img[-0.8][-0.5][8][-0.5]{yellow_diamond_icon}}
\newcommand*{\transdiamond}{\img[-0.8][-0.5][8][-0.5]{trans_diamond_icon}}

\newcommand*{\redtriangle}{\img[-0.8][-0.5][10][-0.5]{red_triangle_icon}}
\newcommand*{\bluetriangle}{\img[-0.8][-0.5][10][-0.5]{blue_triangle_icon}}
\newcommand*{\bblacktriangle}{\img[-0.8][-0.5][10][-0.5]{black_triangle_icon}}

\newcommand*{\blackrectangle}{\img[-0.8][-0.5][10][-0.5]{black_rectangle_icon}}
\newcommand*{\yellowrectangle}{\img[-0.8][-0.5][10][-0.5]{yellow_rectangle_icon}}






\begin{document}

\subsubsection{XXXIII. Theorem}

\begin{minipage}[t]{0.54\textwidth}
    \begin{center}
        \textit{PROPOSITION XXXIII. THEOREM.}\phantomsection\label{book6pr33} \\
    \end{center}

    \hfill

    \begin{center}
        \raggedright \lettrine[lines=3, loversize=1, nindent=0pt]{\euclidinitials{I}}{}N \textit{equal circles}\\ (\hspace{-1ex}$\img{red_circle_2} \bcomma \img{blue_circle_4}$\hspace{-1ex}),
    \end{center}
    \textit{angles, whether at the centre or circumference, are in the ſame ratio to one another as the arcs on which they ſtand}\\
    (\hspace{-1ex}$\img[0][0][11]{black_angle_44} \bcolon \img[0][0][13]{black_outlined_angle_2} \bbcolon \img{thin_black_arc} \bcolon \img{yellow_dotted_arc}$\hspace{-1ex});\\ \textit{ſo alſo are ſectors}.

    \hfill

    \hfill

    \raggedright Take in the circumference of $\img{red_circle_2}$ any number of arcs $\img{thin_red_arc} \bcomma \img{thin_blue_arc} \bcomma$ \&c. each $\bequals \img{thin_black_arc} \bcomma$ and alſo in the circumference of $\img{blue_circle_4}$ take any number of arcs $\img{red_dotted_arc_2} \bcomma \img{blue_dotted_arc} \bcomma$ \&c. each $\bequals \img{yellow_dotted_arc} \bcomma$ draw the radii to the extremities of the equal arcs.
\end{minipage}%
\hfill
\begin{minipage}[t]{0.43\textwidth}
    \vspace{20pt}
    \includesvg[width=\textwidth]{book6_proposition_33_figure_1}
    \hfill\\
    \includesvg[width=\textwidth]{book6_proposition_33_figure_2}
\end{minipage}

\hfill

\begin{center}
    Then ſince the arcs $\img{thin_black_arc} \bcomma \img{thin_red_arc} \bcomma \img{thin_blue_arc} \bcomma$ \&c. are all equal, the angles $\img[0][0][11]{black_angle_44} \bcomma \img[0][0][8]{red_angle_56} \bcomma \img[0][0][11]{blue_angle_58} \bcomma$ \&c. are alſo equal [\hyperref[book3pr27]{\textsc{III.} 27}]; $\btherefore \img{black_red_and_blue_angle}$ is the ſame multiple of $\img[0][0][10]{black_angle_44}$ which the arc $\img[0][0][40]{thin_black_red_and_blue_arc}$ is of $\img{thin_black_arc} \bsemicolon$ and in the ſame manner $\img{black_red_and_blue_outlined_angle}$ is the ſame multiple of $\img[0][0][12]{black_outlined_angle_2} \bcomma$ which the arc $\img[0][0][40]{yellow_red_and_blue_dotted_arc}$ is of the arc $\img{yellow_dotted_arc} \bperiod$
\end{center}

\begin{center}
    Then it is evident [\hyperref[book3p27]{\textsc{III.} 27}], if $\img{black_red_and_blue_angle}$ (or if \textit{m} times $\img[0][0][11]{black_angle_44}$\hspace{-1ex}) $\bgt \bcomma \hspace{-1ex} \bequals \bcomma \hspace{-1ex} \blt \img{black_red_and_blue_outlined_angle}$ (or \textit{n} times $\img[0][0][13]{black_outlined_angle_2}$\hspace{-1ex}) then $\img[0][0][40]{thin_black_red_and_blue_arc}$ (or \textit{m} times $\img{thin_black_arc}$\hspace{-1ex}) $\bgt \bcomma \hspace{-1ex} \bequals \bcomma \hspace{-1ex}\blt$ $\img[0][0][40]{yellow_red_and_blue_dotted_arc}$ (or \textit{n} times $\img{yellow_dotted_arc}$\hspace{-1ex})$\bsemicolon$
\end{center}

\raggedright $\btherefore \img[0][0][11]{black_angle_44} \bcolon \img[0][0][13]{black_outlined_angle_2} \bbcolon \img{thin_black_arc} \bcolon \img{yellow_dotted_arc} \bcomma$ [\hyperref[book5def5]{\textsc{V.} def. 5}], or the angles at the centre are as the arcs on which they ſtand; but the angles at the circumference being halves of the angles at the centre [\hyperref[book3pr20]{\textsc{III.} 20}] are in the ſame ratio [\hyperref[book5pr15]{\textsc{V.} 15}], and therefore are as the arcs on which they ſtand.\\
\hfill\\
It is evident that ſectors in equal circles, and on equal arcs are equal\\
\hspace{0ex}[\hyperref[book1pr4]{\textsc{I.} 4}; \textsc{III.} \hyperref[book3pr24]{24}, \hyperref[book3pr27]{27}, and \hyperref[book3def9]{def 9}]. Hence, if the ſectors be ſubſtituted for the angles in the above demonſtration, the ſecond part of the propoſition will be eſtabliſhed, that is, in equal circles the ſectors have the ſame ratio to one another as the arcs on which they ſtand.

\hfill

\hfill Q.E.D.

\end{document}