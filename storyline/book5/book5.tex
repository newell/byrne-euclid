\pagestyle{fancy}
\fancyhf{}
\renewcommand{\headrulewidth}{0pt}
\fancyfoot[C]{\textsc{V.} \thepage}%

\section[Book V]{\centering BOOK V.}
\label{sec:book5}

\hfill

\subsection[Definitions]{\centering \scshape{\LARGE{DEFINITIONS.}}}
\label{subsec:definitions}

\hfill

\begin{center}
    \begin{minipage}{0.8\textwidth}
        \subsubsection{def. 1}
        \begin{center}
            I.\phantomsection\label{book5def1}\\
            \raggedright \lettrine[lines=4, loversize=1, nindent=0pt]{\euclidinitials{A}}{} leſs magnitude is ſaid to be an aliquot part or ſubmultiple of a\\ greater magnitude, when the leſs meaſures the greater; that is, when\\ the leſs is contained a certain number of times exactly in the greater.
        \end{center}
        \hfill\\
        \subsubsection{def. 2}
        \begin{center}
            II.\phantomsection\label{book5def2}\\
            \hfill\\
            \raggedright A \textsc{GREATER} magnitude is ſaid to be a multiple of a leſs, when the greater is meaſured by the leſs; that is, when the greater contains the leſs a certain number of times exactly.
        \end{center}
        \hfill\\
        \subsubsection{def. 3}
        \begin{center}
            III.\phantomsection\label{book5def3}\\
            \hfill\\
            \raggedright R\textsc{ATIO} is the relation which one quantity bears to another of the ſame kind, with reſpect to magnitude.
        \end{center}
        \hfill\\
        \subsubsection{def. 4}
        \begin{center}
            IV.\phantomsection\label{book5def4}\\
            \hfill\\
            \raggedright M\textsc{AGNITUDES} are ſaid to have a ratio to one another, when they are of the ſame kind and the one which is not the greater can be multiplied ſo as to exceed the other.
        \end{center}
        \hfill\\
        \textit{The other definitions will be given throughout the book where their aid is firſt required}.
    \end{minipage}
\end{center}

\newpage

\subsection[Axioms]{\centering \scshape{\LARGE{AXIOMS.}}}
\label{subsec:axioms}

\hfill

\begin{center}
    \begin{minipage}{0.80\textwidth}
        \subsubsection{ax. 1}
        \begin{center}
            I.\phantomsection\label{book5ax1}\\
            \raggedright \lettrine[lines=4, loversize=1, nindent=0pt]{\euclidinitials{E}}{}QUIMULTIPLES or equiſubmultiples of the ſame, or of equal\\ magnitudes, are equal.
        \end{center}

        \hfill

        \hfill

        \centering

        $\begin{aligned}
                \text{if A}                                                       & \bequals \text{B, then}                                         \\
                \text{twice A}                                                    & \bequals \text{twice B, that is,}                               \\
                \text{2 A}                                                        & \bequals \text{2 B;}                                            \\
                \text{3 A}                                                        & \bequals \text{3 B;}                                            \\
                \text{4 A}                                                        & \bequals \text{4 B;}                                            \\
                                                                                  & \text{\&c. \&c.}                                                \\
                \hfill                                                                                                                              \\
                \text{and } \dfrac{\text{\large 1}}{\text{\large 2}} \text{ of A} & \bequals \dfrac{\text{\large 1}}{\text{\large 2}} \text{ of B;} \\
                \dfrac{\text{\large 1}}{\text{\large 3}} \text{ of A}             & \bequals \dfrac{\text{\large 1}}{\text{\large 3}} \text{ of B;} \\
                                                                                  & \text{\&c. \&c.}
            \end{aligned}$\\

        \hfill

        \hfill

        \subsubsection{ax. 2}
        II.\phantomsection\label{book5ax2}\\
        \hfill\\
        A \textsc{MULTIPLE} of a greater magnitude is greater than the ſame multiple of a leſs.\\
        \hfill\\
        $\begin{aligned}
                \text{Let A} & \bgt \text{B, then} \\
                \text{2 A}   & \bgt \text{2 B;}    \\
                \text{3 A}   & \bgt \text{3 B;}    \\
                \text{4 A}   & \bgt \text{4 B;}    \\
                             & \text{\&c. \&c.}
            \end{aligned}$
    \end{minipage}%
\end{center}

\hfill

\begin{center}
    \begin{minipage}{0.80\textwidth}
        \subsubsection{ax. 3}
        \begin{center}

            III.\phantomsection\label{book5ax3}\\
            \hfill\\
            {\begingroup
            \raggedright T\textsc{HAT} magnitude, of which a multiple is greater than the ſame multiple of another, is greater than the other.\\
            \endgroup}

            \hfill\\
            $\begin{aligned}
                    \text{Let 2 A}              & \bgt \text{2 B, then}          \\
                    \text{A}                    & \bgt \text{B;}                 \\
                    \text{or, let 3 A}          & \bgt \text{3 B, then}          \\
                    \text{A}                    & \bgt \text{B}                  \\
                    \text{or, let \textit{m} A} & \bgt \text{\textit{m} B, then} \\
                    \text{A}                    & \bgt \text{B.}                 \\
                                                & \text{\&c. \&c.}
                \end{aligned}$
        \end{center}
    \end{minipage}
\end{center}

\newpage

\subsection[Propositions]{\centering \scshape{\LARGE{PROPOSITIONS.}}}
\label{subsec:propositions}

\hfill

% Propositions
\foreach \c in {prop1,prop2,prop3,def5,prop4,prop5,prop6,propA,def14,propB,propC,propD,prop7,def7,prop8,prop9,prop10,prop11,prop12,prop13,prop14,prop15,def13,prop16,def16,prop17,def15,prop18,prop19,def17,propE,def18,def19,def20,prop20,prop21,prop22,prop23,prop24,prop25,def10,def11,defA,propF,propG,propH,propK}{
        \vspace*{\fill}
        \input{book5/\c.tex}
        \vspace*{\fill}
        \newpage
    }
