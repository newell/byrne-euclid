% !TEX TS-program = xelatex
% !TEX options = -shell-escape -synctex=1 -interaction=nonstopmode -file-line-error "%DOC%"
\documentclass[11pt,preview]{standalone}

\usepackage{standalone}
\usepackage{fancyhdr}
\usepackage{xpatch}
\usepackage{titletoc}
\usepackage{setspace}
\usepackage{minibox}
\usepackage{enumitem}
\usepackage{fullpage}
\usepackage{mathtools,mathrsfs}
\usepackage{amssymb,amsthm}
\usepackage{graphicx,xcolor}
\usepackage[scale=2]{ccicons}
\usepackage[breakable]{tcolorbox}
\tcbuselibrary{breakable}
\usepackage{subfig}
\usepackage{float}
\usepackage{parskip}
\usepackage{lettrine}
\usepackage{fontspec}
\usepackage[compact]{titlesec}
\usepackage{calc}
\usepackage{xparse}
\usepackage{tikz}
\usepackage{svg}
\usepackage{wrapfig}
\usepackage{xr-hyper}
\usepackage[colorlinks=true, citecolor=violet, linkcolor=black, urlcolor=black]{hyperref}
\usepackage[
    papersize={6.25in,9.25in},
    margin=0.75in,
    layoutsize={6in,9in},
    layoutoffset={0.125in,0.125in},
    includehead,
    includefoot,
    showcrop,
]{geometry}

% Set this PATH to the root directory of your repository
\newcommand*{\MyPath}{/home/newell/code/byrne-euclid}%
% Don't show subsubsection titles -- these still show up in TOC (as desired)
\makeatletter
\titleformat{\subsubsection}[runin]{}{}{0pt}{\@gobble}
\makeatother

\pagestyle{fancy}
\fancyhf{}
\renewcommand{\headrulewidth}{0pt}
\fancyfoot[LO,RE]{\thepage}%

% TOC - image setup
\newcounter{propimage}
\makeatletter
\newcommand\stdsectioninToC{
    \titlecontents{subsubsection}[3.8em]
    {}%
    {\contentslabel{2.3em}}%
    {\hspace*{2.3em}}%
    {\titlerule*[1em]{.}\contentspage}
}
\newcommand\iconsectioninToC{
    \titlecontents{subsubsection}[3.8em]
    {\vskip 2ex}%
    {\hspace*{-2.3em}}%
    {
        \contentslabel{2.3em}%
        \stepcounter{propimage}%
        \smash{\includegraphics[width=50pt,height=25pt,keepaspectratio]{\MyPath/toc-images/image-\the\value{propimage}}}\hspace{0.5em}%
    }%
    {\titlerule*[1em]{.}\contentspage}%
}
\AtBeginDocument{\stdsectioninToC}
\makeatother

\setcounter{secnumdepth}{-3}
\setcounter{tocdepth}{3}

% Set the Font
\setmainfont{EBGaramond-Regular.otf}[
    BoldFont = EBGaramond-Bold.otf,
    ItalicFont = EBGaramond-Italic.otf,
    BoldItalicFont = EBGaramond-BoldItalic.otf,
    SmallCapsFont = EBGaramond12-AllSC.otf,
    Numbers = {OldStyle,Proportional},
    Ligatures = {Discretionary},
    RawFeature = {+ss06}
]
\newfontfamily\euclidinitials{EuclidInitialsnormal.ttf}

% Set the SVG paths (when I tried multiple lines to make this more readable it errored on me)
\svgpath{{\MyPath/symbols/}{\MyPath/byrne-euclid-svg/lines/}{\MyPath/byrne-euclid-svg/lines/points/}{\MyPath/byrne-euclid-svg/lines/arcs/}{\MyPath/byrne-euclid-svg/circles/}{\MyPath/byrne-euclid-svg/triangles/}{\MyPath/byrne-euclid-svg/figures/}{\MyPath/byrne-euclid-svg/figures/definitions/}{\MyPath/byrne-euclid-svg/figures/propositions/}{\MyPath/byrne-euclid-svg/angles/}{\MyPath/byrne-euclid-svg/symbols/}{\MyPath/byrne-euclid-svg/lines/}{\MyPath/byrne-euclid-svg/lines/points/}{\MyPath/byrne-euclid-svg/lines/arcs/}{\MyPath/byrne-euclid-svg/circles/}{\MyPath/byrne-euclid-svg/triangles/}{\MyPath/byrne-euclid-svg/figures/}{\MyPath/byrne-euclid-svg/figures/definitions/}{\MyPath/byrne-euclid-svg/figures/propositions/}{\MyPath/byrne-euclid-svg/angles/}{\MyPath/byrne-euclid-svg/icons/}}

% Create counter for lists.
\newcounter{listcounter}

% Colors
\definecolor{cred}{RGB}{212,42,32} % red
\definecolor{cyellow}{RGB}{250,194,43} % yellow
\definecolor{cblue}{RGB}{14,99,142} % blue
\definecolor{ctrans}{RGB}{252,243,217} % transparent
\definecolor{background}{HTML}{fcf3d9}

\newlength\mytemplena
\newlength\mytemplenb
\DeclareDocumentCommand\myalignalign{sm}
{
    \settowidth{\mytemplena}{$\displaystyle #2$}%
    \setlength\mytemplenb{\widthof{$\displaystyle=$}/2}%
    \hskip-\mytemplena%
    \hskip\IfBooleanTF#1{-\mytemplenb}{+\mytemplenb}%
}

% Images
\NewDocumentCommand{\img}{%
    O{0}
    O{0}
    O{25}
    O{0}
    m
}{%
    \begin{array}{c}
        \vspace{#1ex}\hspace{#2ex}\includesvg[width=#3pt]{#5}\hspace{#4ex}
    \end{array}
}

% Common Symbols
\newcommand*{\bplus}{\img[-0.8][0][10]{plus}}
\newcommand*{\bminus}{\img[0.8][0][10]{minus}}
\newcommand*{\bcross}{\img[-0.85][0][10]{cross}}
\newcommand*{\bequals}{\img[-0.1][0][10]{equals}}
\newcommand*{\bnequals}{\img[-0.85][0][10]{nequals}}
\newcommand*{\bparallel}{\img[-0.85][0][7]{parallel}}
\newcommand*{\bnparallel}{\img[-0.85][0][11]{not_parallel}}
\newcommand*{\bperiod}{\img[-0.5][-1][3]{period}}
\newcommand*{\bmark}{\img[-0.5][-1][3]{mark}}
\newcommand*{\bmultiply}{\img[1][0][3]{period}}
\newcommand*{\bsemicolon}{\img[-1.4][-1][3]{semicolon}}
\newcommand*{\bcolon}{\img[-0.5][0][3]{colon}}
\newcommand*{\bbcolon}{\bcolon\hspace{-1.5ex}\bcolon}
\newcommand*{\bcomma}{\img[-1.5][-1][3]{comma}}
\newcommand*{\btherefore}{\img[-0.5][0][10]{therefore}}
\newcommand*{\bgt}{\img[-0.85][0][10]{greater_than}}
\newcommand*{\blt}{\img[-0.85][0][10]{less_than}}
\newcommand*{\bperp}{\img[0][0][15]{perpendicular}}

% Common Lines
\newcommand*{\bimg}[1]{\img[0.8][0][45]{#1}}
\newcommand*{\redlines}{\bimg{red_lines}}
\newcommand*{\bluelines}{\bimg{blue_lines}}
\newcommand*{\blacklines}{\bimg{black_lines}}

\newcommand*{\redline}{\img[0.8][0][30]{red_line}}
\newcommand*{\redthinline}{\img[0.8][0][30]{red_thin_line}}
\newcommand*{\dottedredline}{\img[0.8][0][30]{red_dotted_line}}
\newcommand*{\blueline}{\img[0.8][0][30]{blue_line}}
\newcommand*{\bluethinline}{\img[0.8][0][30]{blue_thin_line}}
\newcommand*{\dottedblueline}{\img[0.8][0][30]{blue_dotted_line}}
\newcommand*{\blackline}{\img[0.8][0][30]{black_line}}
\newcommand*{\blackthinline}{\img[0.8][0][30]{black_thin_line}}
\newcommand*{\blackthindottedline}{\img[0.8][0][30]{black_thin_dotted_line}}
\newcommand*{\dottedblackline}{\img[0.8][0][30]{black_dotted_line}}
\newcommand*{\yellowline}{\img[0.8][0][30]{yellow_line}}
\newcommand*{\yellowthinline}{\img[0.8][0][30]{yellow_thin_line}}
\newcommand*{\dottedyellowline}{\img[0.8][0][30]{yellow_dotted_line}}

% Common Angles
\newcommand*{\tworightangles}{\img[0][0][30]{two_right_angles}}

% Common Shapes -- Book 5
\newcommand*{\reddome}{\img[-0.8][-0.5][10][-0.5]{red_dome_icon}}
\newcommand*{\bluedome}{\img[-0.8][-0.5][10][-0.5]{blue_dome_icon}}
\newcommand*{\blackdome}{\img[-0.8][-0.5][10][-0.5]{black_dome_icon}}
\newcommand*{\yellowdome}{\img[-0.8][-0.5][10][-0.5]{yellow_dome_icon}}

\newcommand*{\redcircle}{\img[-0.8][-0.5][10][-0.5]{red_circle_icon}}
\newcommand*{\bluecircle}{\img[-0.8][-0.5][10][-0.5]{blue_circle_icon}}
\newcommand*{\blackcircle}{\img[-0.8][-0.5][10][-0.5]{black_circle_icon}}
\newcommand*{\yellowcircle}{\img[-0.8][-0.5][10][-0.5]{yellow_circle_icon}}

\newcommand*{\reddrop}{\img[-0.8][-0.5][10][-0.5]{red_drop_icon}}
\newcommand*{\bluedrop}{\img[-0.8][-0.5][10][-0.5]{blue_drop_icon}}
\newcommand*{\blackdrop}{\img[-0.8][-0.5][10][-0.5]{black_drop_icon}}
\newcommand*{\yellowdrop}{\img[-0.8][-0.5][10][-0.5]{yellow_drop_icon}}

\newcommand*{\redhome}{\img[-0.8][-0.5][10][-0.5]{red_home_icon}}
\newcommand*{\bluehome}{\img[-0.8][-0.5][10][-0.5]{blue_home_icon}}
\newcommand*{\blackhome}{\img[-0.8][-0.5][10][-0.5]{black_home_icon}}
\newcommand*{\yellowhome}{\img[-0.8][-0.5][10][-0.5]{yellow_home_icon}}

\newcommand*{\redsquare}{\img[-0.8][-0.5][10][-0.5]{red_square_icon}}
\newcommand*{\smallredsquare}{\img[-0.8][-0.5][5][-0.5]{red_square_icon}}
\newcommand*{\bluesquare}{\img[-0.8][-0.5][10][-0.5]{blue_square_icon}}
\newcommand*{\bblacksquare}{\img[-0.8][-0.5][10][-0.5]{black_square_icon}}
\newcommand*{\yellowsquare}{\img[-0.8][-0.5][10][-0.5]{yellow_square_icon}}
\newcommand*{\transsquare}{\img[-0.8][-0.5][10][-0.5]{trans_square_icon}}

\newcommand*{\reddiamond}{\img[-0.8][-0.5][8][-0.5]{red_diamond_icon}}
\newcommand*{\bluediamond}{\img[-0.8][-0.5][8][-0.5]{blue_diamond_icon}}
\newcommand*{\blackdiamond}{\img[-0.8][-0.5][8][-0.5]{black_diamond_icon}}
\newcommand*{\yellowdiamond}{\img[-0.8][-0.5][8][-0.5]{yellow_diamond_icon}}
\newcommand*{\transdiamond}{\img[-0.8][-0.5][8][-0.5]{trans_diamond_icon}}

\newcommand*{\redtriangle}{\img[-0.8][-0.5][10][-0.5]{red_triangle_icon}}
\newcommand*{\bluetriangle}{\img[-0.8][-0.5][10][-0.5]{blue_triangle_icon}}
\newcommand*{\bblacktriangle}{\img[-0.8][-0.5][10][-0.5]{black_triangle_icon}}

\newcommand*{\blackrectangle}{\img[-0.8][-0.5][10][-0.5]{black_rectangle_icon}}
\newcommand*{\yellowrectangle}{\img[-0.8][-0.5][10][-0.5]{yellow_rectangle_icon}}





\usepackage{minibox}

\begin{document}

\subsubsection{K. Theorem}

\newcommand{\bluea}{{\color{cblue}{\textit{a}}}}
\newcommand{\blueb}{{\color{cblue}{\textit{b}}}}
\newcommand{\bluec}{{\color{cblue}{\textit{c}}}}
\newcommand{\blued}{{\color{cblue}{\textit{d}}}}
\newcommand{\bluee}{{\color{cblue}{\textit{e}}}}
\newcommand{\bluef}{{\color{cblue}{\textit{f}}}}
\newcommand{\blueg}{{\color{cblue}{\textit{g}}}}

\newcommand{\redA}{{\color{cred}{A}}}
\newcommand{\redB}{{\color{cred}{B}}}
\newcommand{\redC}{{\color{cred}{C}}}
\newcommand{\redD}{{\color{cred}{D}}}
\newcommand{\redE}{{\color{cred}{E}}}
\newcommand{\redF}{{\color{cred}{F}}}
\newcommand{\redG}{{\color{cred}{G}}}
\newcommand{\redH}{{\color{cred}{H}}}
\newcommand{\redK}{{\color{cred}{K}}}
\newcommand{\redL}{{\color{cred}{L}}}
\newcommand{\redM}{{\color{cred}{M}}}
\newcommand{\redN}{{\color{cred}{N}}}

\newcommand{\yellowO}{{\color{cyellow}{O}}}
\newcommand{\yellowP}{{\color{cyellow}{P}}}
\newcommand{\yellowQ}{{\color{cyellow}{Q}}}
\newcommand{\yellowR}{{\color{cyellow}{R}}}
\newcommand{\yellowS}{{\color{cyellow}{S}}}
\newcommand{\yellowT}{{\color{cyellow}{T}}}
\newcommand{\yellowV}{{\color{cyellow}{V}}}
\newcommand{\yellowW}{{\color{cyellow}{W}}}
\newcommand{\yellowX}{{\color{cyellow}{X}}}
\newcommand{\yellowY}{{\color{cyellow}{Y}}}

\begin{minipage}{\textwidth}
    \begin{center}
        \textit{PROPOSITION K. THEOREM.}\phantomsection\label{book5prK} \\
    \end{center}

    \begin{center}
        \raggedright \lettrine[lines=3, loversize=1, nindent=0pt]{\euclidinitials{I}}{}F \textit{there be any number of ratios, and any number of other ratios, ſuch that the ratio which is compounded of ratios, which are the ſame to the firſt ratios, each to each, is the ſame to the ratio which is compounded of ratios, which are the ſame, each to each, to the laſt ratios—and if one of the firſt ratios, or the ratio which is compounded of ratios, which are the ſame to ſeveral of the firſt ratios, each to each, be the ſame to one of the laſt ratios, or to the ratio which is compounded of ratios, which are the ſame, each to each, to ſeveral of the laſt ratios—then the remaining ratio of the firſt; or, if there be more than one, the ratio which is compounded of ratios, which are the ſame, each to each, to the remaining ratios of the firſt, ſhall be the ſame to the remaining ratio of the laſt; or, if there be more than one, to the ratio which is compounded of ratios, which are the ſame, each to each, to theſe remaining ratios}.
    \end{center}
\end{minipage}

\hfill

\begin{center}
    \minibox[frame,c]{h\ \ k\ \ m\ \ n\ \ s \\ \redA\ \ \redB,\ \ \redC\ \ \redD,\ \ \redE\ \ \redF,\ \ \redG\ \ \redH,\ \ \redK\ \ \redL,\ \ \redM\ \ \redN, \qquad \qquad \bluea\ \blueb\ \bluec\ \blued\ \bluee\ \bluef\ \blueg \\ \yellowO\ \ \yellowP,\ \ \yellowQ\ \ \yellowR,\ \ \yellowS\ \ \yellowT,\ \ \yellowV\ \ \yellowW,\ \ \yellowX\ \ \yellowY, \qquad \qquad \qquad\  \textit{h}\ \textit{k}\ \textit{l}\ \textit{m}\ \textit{n}\ \textit{p} \\ \hspace{-20ex}a\ \ b\ \ \ \ c\ \ d \qquad e\ \ \ \ \ f\ \ g}
\end{center}

\raggedright Let \redA\ : \redB, \redC\ : \redD, \redE\ : \redF, \redG\ : \redH, \redK\ : \redL, \redM\ : \redN, be the firſt ratios, and \yellowO\ : \yellowP, \yellowQ\ : \yellowR, \yellowS\ : \yellowT, \yellowV\ : \yellowW, \yellowX\ : \yellowY, the other ratios;

\begin{center}
    $\begin{aligned}
            \text{and let \redA\ : \redB} & \bequals \text{\bluea\ : \blueb,} \\
            \text{\redC\ : \redD}         & \bequals \text{\blueb\ : \bluec,} \\
            \text{\redE\ : \redF}         & \bequals \text{\bluec\ : \blued,} \\
            \text{\redG\ : \redH}         & \bequals \text{\blued\ : \bluee,} \\
            \text{\redK\ : \redL}         & \bequals \text{\bluee\ : \bluef,} \\
            \text{\redM\ : \redN}         & \bequals \text{\bluef\ : \blueg.}
        \end{aligned}$
\end{center}

Then, by the definition of a compound ratio, the ratio of \bluea\ : \blueg\ is compounded of the ratios of \bluea\ : \blueb, \blueb\ : \bluec, \bluec\ : \blued, \blued\ : \bluee, \bluee\ : \bluef, \bluef\ : \blueg, which are the ſame as the ratio of \redA\ : \redB, \redC\ : \redD, \redE\ : \redF, \redG\ : \redH, \redK\ : \redL, \redM\ : \redN, each to each.

\begin{center}
    $\begin{aligned}
            \text{Alſo, \yellowO\ : \yellowP} & \bequals \text{\textit{h}\ : \textit{k},} \\
            \text{\yellowQ\ : \yellowR}       & \bequals \text{\textit{k}\ : \textit{l},} \\
            \text{\yellowS\ : \yellowT}       & \bequals \text{\textit{l}\ : \textit{m},} \\
            \text{\yellowV\ : \yellowW}       & \bequals \text{\textit{m}\ : \textit{n},} \\
            \text{\yellowX\ : \yellowY}       & \bequals \text{\textit{n}\ : \textit{p}.}
        \end{aligned}$
\end{center}

Then will the ratio of \textit{h}\ : \textit{p} be the ratio compounded of the ratios \textit{h}\ : \textit{k}, \textit{k}\ : \textit{l}, \textit{l}\ : \textit{m}, \textit{m}\ : \textit{n}, \textit{n} : \textit{p}, which are the ſame ratios of \yellowO\ : \yellowP, \yellowQ\ : \yellowR, \yellowS\ : \yellowT, \yellowV\ : \yellowW, \yellowX\ : \yellowY, each to each.

\begin{center}
    $\btherefore$ by the hypotheſis, \bluea\ : \blueg $\bequals$ \textit{h}\ : \textit{p}.
\end{center}

Alſo, let the ratio which is compounded of the ratios of \redA\ : \redB, \redC\ : \redD, two of the firſt ratios (or the ratios of \bluea\ : \bluec, for \redA\ : \redB $\bequals$ \bluea\ : \blueb, and \redC\ : \redD $\bequals$ \blueb\ : \bluec), be the ſame as the ratio of a : d, which is compounded of the ratios a : b, b : c, c : d, which are the ſame as the ratios of \yellowO\ : \yellowP, \yellowQ\ : \yellowR, \yellowS\ : \yellowT, three of the other ratios.

And let the ratios of h : s, which is compounded of the ratios h : k, k : m, m : n, n : s, which are the ſame as the remaining firſt ratios, namely, \redE : \redF, \redG : \redH, \redK : \redL, \redM : \redN; alſo, let the ratio of e : g, be that which is compounded of the ratios e : f, f : g, which are the ſame, each to each, to the remaining other ratios, namely, \yellowV : \yellowW, \yellowX : \yellowY. Then the ratio of h : s ſhall be the ſame as the ratio of e : g; or h : s $\bequals$ e : g.

\begin{center}
    \begin{onehalfspace}
        For $\dfrac{\text{\redA\ × \redC\ × \redE\ × \redG\ × \redK\ × \redM}}{\text{\redB\ × \redD\ × \redF\ × \redH\ × \redL\ × \redN}} \bequals \dfrac{\text{\bluea\ × \blueb\ × \bluec\ × \blued\ × \bluee\ × \bluef}}{\text{\blueb\ × \bluec\ × \blued\ × \bluee\ × \bluef\ × \blueg}}$,\\
        \hfill\\
        and $\dfrac{\text{\yellowO\ × \yellowQ\ × \yellowS\ × \yellowV\ × \yellowX}}{\text{\yellowP\ × \yellowR\ × \yellowT\ × \yellowW\ × \yellowY}} \bequals \dfrac{\text{\textit{h}\ × \textit{k}\ × \textit{l}\ × \textit{m}\ × \textit{n}}}{\text{\textit{k}\ × \textit{l}\ × \textit{m}\ × \textit{n}\ × \textit{p}}}$,\\
        \hfill\\
        by the compoſition of the ratios;\\
        \hfill\\
        $\btherefore \dfrac{\text{\bluea\ × \blueb\ × \bluec\ × \blued\ × \bluee\ × \bluef}}{\text{\blueb\ × \bluec\ × \blued\ × \bluee\ × \bluef\ × \blueg}} \bequals \dfrac{\text{\textit{h}\ × \textit{k}\ × \textit{l}\ × \textit{m}\ × \textit{n}}}{\text{\textit{k}\ × \textit{l}\ × \textit{m} × \textit{n} × \textit{p}}}$, [hyp.],\\
        \hfill\\
        or $\dfrac{\text{\bluea\ × \blueb}}{\text{\blueb\ × \bluec}} \bcross \dfrac{\text{\bluec\ × \blued\ × \bluee\ × \bluef}}{\text{\blued\ × \bluee\ × \bluef\ × \blueg}} \bequals \dfrac{\text{\textit{h}\ × \textit{k}\ × \textit{l}}}{\text{\textit{k}\ × \textit{l}\ × \textit{m}}} \bcross \dfrac{\text{\textit{m}\ × \textit{n}}}{\text{\textit{n}\ × \textit{p}}}$,\\
        \hfill\\
        but $\dfrac{\text{\bluea\ × \blueb}}{\text{\blueb\ × \bluec}} \bequals \dfrac{\text{\redA × \redC}}{\text{\redB\ × \redD}} \bequals \dfrac{\text{\yellowO\ × \yellowQ\ × \yellowS}}{\text{\yellowP\ × \yellowR\ × \yellowT}} \bequals \dfrac{\text{a × b × c}}{\text{b × c × d}} \bequals \dfrac{\text{\textit{h}\ × \textit{k}\ × \textit{l}}}{\text{\textit{k}\ × \textit{l}\ × \textit{m}}}$;\\
    \end{onehalfspace}
\end{center}

\begin{center}
    \begin{onehalfspace}
        $\btherefore \dfrac{\text{\bluec\ × \blued\ × \bluee\ × \bluef}}{\text{\blued\ × \bluee\ × \bluef\ × \blueg}} \bequals \dfrac{\text{\textit{m}\ × \textit{n}}}{\text{\textit{n}\ × \textit{p}}}$ [hyp.],\\
        \hfill\\
        And $\dfrac{\text{\bluec\ × \blued\ × \bluee\ × \bluef}}{\text{\blued\ × \bluee\ × \bluef\ × \blueg}} \bequals \dfrac{\text{h × k × m × n}}{\text{k × m × n × s}}$ [hyp.],\\
        \hfill\\
        and $\dfrac{\text{\textit{m}\ × \textit{n}}}{\text{\textit{n}\ × \textit{p}}} \bequals \dfrac{\text{e × f}}{\text{f × g}}$ [hyp.],\\
        \hfill\\
        $\btherefore \dfrac{\text{h × k × m × n}}{\text{k × m × n × s}} \bequals \dfrac{\text{ef}}{\text{fg}}$,\\
        \hfill\\
        $\btherefore \dfrac{\text{h}}{\text{s}} \bequals \dfrac{\text{e}}{\text{g}}$,\\
        \hfill\\
        $\btherefore$ h : s $\bequals$ e : g.
    \end{onehalfspace}
\end{center}

\hfill

$\btherefore$ If there be any number, \&c.

\end{document}