% !TEX TS-program = xelatex
% !TEX options = -shell-escape -synctex=1 -interaction=nonstopmode -file-line-error "%DOC%"
\documentclass[12pt,preview]{standalone}

\usepackage{standalone}
\usepackage{lipsum}
\usepackage{fancyhdr}
\usepackage{xpatch}
% \usepackage{tocloft}
\usepackage{titletoc}
\usepackage{setspace}
\usepackage{minibox}
\usepackage{enumitem}
\usepackage{fullpage}
\usepackage{mathtools,mathrsfs}
\usepackage{amssymb,amsthm}
\usepackage{graphicx,xcolor}
\usepackage[scale=2]{ccicons}
\usepackage[breakable]{tcolorbox}
\tcbuselibrary{breakable}
\usepackage{subfig}
\usepackage{float}
\usepackage{parskip}
\usepackage{lettrine}
\usepackage{fontspec}
\usepackage[compact]{titlesec}
\usepackage{calc}
\usepackage{xparse}
\usepackage{tikz}
\usepackage{svg}
\usepackage{xr-hyper}
\usepackage[colorlinks=true, citecolor=violet, linkcolor=black, urlcolor=black]{hyperref}

% Set this PATH to the root directory of your repository
\newcommand*{\MyPath}{/home/newell/code/byrne-euclid}%
% Don't show subsubsection titles -- these still show up in TOC (as desired)
\makeatletter
\titleformat{\subsubsection}[runin]{}{}{0pt}{\@gobble}
\makeatother

% \pagestyle{fancy}
% \fancyhf{}
% \renewcommand{\headrulewidth}{0pt}
% \fancyfoot[LO,RE]{\thepage}%

% TOC - image setup
\newcounter{propimage}
\makeatletter
\newcommand\stdsectioninToC{
    \titlecontents{subsubsection}[3.8em]
    {}%
    {\contentslabel{2.3em}}%
    {\hspace*{2.3em}}%
    {\titlerule*[1em]{.}\contentspage}
}
\newcommand\iconsectioninToC{
    \titlecontents{subsubsection}[3.8em]
    {\vskip 2ex}%
    {\hspace*{-2.3em}}%
    {
        \contentslabel{2.3em}%
        \stepcounter{propimage}%
        \smash{\includegraphics[width=50pt,height=25pt,keepaspectratio]{\MyPath/toc-images/image-\the\value{propimage}}}\hspace{0.5em}%
    }%
    {\titlerule*[1em]{.}\contentspage}%
}
\AtBeginDocument{\stdsectioninToC}
\makeatother

\setcounter{secnumdepth}{-3}
\setcounter{tocdepth}{3}

% Set the Font
\setmainfont{EBGaramond-Regular.otf}[
    BoldFont = EBGaramond-Bold.otf,
    ItalicFont = EBGaramond-Italic.otf,
    BoldItalicFont = EBGaramond-BoldItalic.otf,
    SmallCapsFont = EBGaramond12-AllSC.otf,
    Numbers = {OldStyle,Proportional},
    Ligatures = {Discretionary},
    RawFeature = {+ss06}
]
\newfontfamily\euclidinitials{EuclidInitialsnormal.ttf}

% Set the SVG paths (when I tried multiple lines to make this more readable it errored on me)
\svgpath{{\MyPath/symbols/}{\MyPath/byrne-euclid-svg/lines/}{\MyPath/byrne-euclid-svg/lines/points/}{\MyPath/byrne-euclid-svg/lines/arcs/}{\MyPath/byrne-euclid-svg/circles/}{\MyPath/byrne-euclid-svg/triangles/}{\MyPath/byrne-euclid-svg/figures/}{\MyPath/byrne-euclid-svg/figures/definitions/}{\MyPath/byrne-euclid-svg/figures/propositions/}{\MyPath/byrne-euclid-svg/angles/}{\MyPath/byrne-euclid-svg/symbols/}{\MyPath/byrne-euclid-svg/lines/}{\MyPath/byrne-euclid-svg/lines/points/}{\MyPath/byrne-euclid-svg/lines/arcs/}{\MyPath/byrne-euclid-svg/circles/}{\MyPath/byrne-euclid-svg/triangles/}{\MyPath/byrne-euclid-svg/figures/}{\MyPath/byrne-euclid-svg/figures/definitions/}{\MyPath/byrne-euclid-svg/figures/propositions/}{\MyPath/byrne-euclid-svg/angles/}{\MyPath/byrne-euclid-svg/icons/}}

% Create counter for lists.
\newcounter{listcounter}

% Colors
\definecolor{cred}{RGB}{212,42,32} % red
\definecolor{cyellow}{RGB}{250,194,43} % yellow
\definecolor{cblue}{RGB}{14,99,142} % blue
\definecolor{ctrans}{RGB}{252,243,217} % transparent
\definecolor{background}{HTML}{fcf3d9}

\newlength\mytemplena
\newlength\mytemplenb
\DeclareDocumentCommand\myalignalign{sm}
{
    \settowidth{\mytemplena}{$\displaystyle #2$}%
    \setlength\mytemplenb{\widthof{$\displaystyle=$}/2}%
    \hskip-\mytemplena%
    \hskip\IfBooleanTF#1{-\mytemplenb}{+\mytemplenb}%
}

% Images
\NewDocumentCommand{\img}{%
    O{0}
    O{0}
    O{25}
    O{0}
    m
}{%
    \begin{array}{c}
        \vspace{#1ex}\hspace{#2ex}\includesvg[width=#3pt]{#5}\hspace{#4ex}
    \end{array}
}

% Common Symbols
\newcommand*{\bplus}{\img[-0.8][0][10]{plus}}
\newcommand*{\bminus}{\img[0.8][0][10]{minus}}
\newcommand*{\bcross}{\img[-0.85][0][10]{cross}}
\newcommand*{\bequals}{\img[-0.1][0][10]{equals}}
\newcommand*{\bnequals}{\img[-0.85][0][10]{nequals}}
\newcommand*{\bparallel}{\img[-0.85][0][7]{parallel}}
\newcommand*{\bnparallel}{\img[-0.85][0][11]{not_parallel}}
\newcommand*{\bperiod}{\img[-0.5][-1][3]{period}}
\newcommand*{\bmark}{\img[-0.5][-1][3]{mark}}
\newcommand*{\bmultiply}{\img[1][0][3]{period}}
\newcommand*{\bsemicolon}{\img[-1.4][-1][3]{semicolon}}
\newcommand*{\bcolon}{\img[-0.5][0][3]{colon}}
\newcommand*{\bbcolon}{\bcolon\hspace{-1.5ex}\bcolon}
\newcommand*{\bcomma}{\img[-1.5][-1][3]{comma}}
\newcommand*{\btherefore}{\img[-0.5][0][10]{therefore}}
\newcommand*{\bgt}{\img[-0.85][0][10]{greater_than}}
\newcommand*{\blt}{\img[-0.85][0][10]{less_than}}
\newcommand*{\bperp}{\img[0][0][15]{perpendicular}}

% Common Lines
\newcommand*{\bimg}[1]{\img[0.8][0][45]{#1}}
\newcommand*{\redlines}{\bimg{red_lines}}
\newcommand*{\bluelines}{\bimg{blue_lines}}
\newcommand*{\blacklines}{\bimg{black_lines}}

\newcommand*{\redline}{\img[0.8][0][30]{red_line}}
\newcommand*{\redthinline}{\img[0.8][0][30]{red_thin_line}}
\newcommand*{\dottedredline}{\img[0.8][0][30]{red_dotted_line}}
\newcommand*{\blueline}{\img[0.8][0][30]{blue_line}}
\newcommand*{\bluethinline}{\img[0.8][0][30]{blue_thin_line}}
\newcommand*{\dottedblueline}{\img[0.8][0][30]{blue_dotted_line}}
\newcommand*{\blackline}{\img[0.8][0][30]{black_line}}
\newcommand*{\blackthinline}{\img[0.8][0][30]{black_thin_line}}
\newcommand*{\blackthindottedline}{\img[0.8][0][30]{black_thin_dotted_line}}
\newcommand*{\dottedblackline}{\img[0.8][0][30]{black_dotted_line}}
\newcommand*{\yellowline}{\img[0.8][0][30]{yellow_line}}
\newcommand*{\yellowthinline}{\img[0.8][0][30]{yellow_thin_line}}
\newcommand*{\dottedyellowline}{\img[0.8][0][30]{yellow_dotted_line}}

% Common Angles
\newcommand*{\tworightangles}{\img[0][0][30]{two_right_angles}}

% Common Shapes -- Book 5
\newcommand*{\reddome}{\img[-0.8][-0.5][10][-0.5]{red_dome_icon}}
\newcommand*{\bluedome}{\img[-0.8][-0.5][10][-0.5]{blue_dome_icon}}
\newcommand*{\blackdome}{\img[-0.8][-0.5][10][-0.5]{black_dome_icon}}
\newcommand*{\yellowdome}{\img[-0.8][-0.5][10][-0.5]{yellow_dome_icon}}

\newcommand*{\redcircle}{\img[-0.8][-0.5][10][-0.5]{red_circle_icon}}
\newcommand*{\bluecircle}{\img[-0.8][-0.5][10][-0.5]{blue_circle_icon}}
\newcommand*{\blackcircle}{\img[-0.8][-0.5][10][-0.5]{black_circle_icon}}
\newcommand*{\yellowcircle}{\img[-0.8][-0.5][10][-0.5]{yellow_circle_icon}}

\newcommand*{\reddrop}{\img[-0.8][-0.5][10][-0.5]{red_drop_icon}}
\newcommand*{\bluedrop}{\img[-0.8][-0.5][10][-0.5]{blue_drop_icon}}
\newcommand*{\blackdrop}{\img[-0.8][-0.5][10][-0.5]{black_drop_icon}}
\newcommand*{\yellowdrop}{\img[-0.8][-0.5][10][-0.5]{yellow_drop_icon}}

\newcommand*{\redhome}{\img[-0.8][-0.5][10][-0.5]{red_home_icon}}
\newcommand*{\bluehome}{\img[-0.8][-0.5][10][-0.5]{blue_home_icon}}
\newcommand*{\blackhome}{\img[-0.8][-0.5][10][-0.5]{black_home_icon}}
\newcommand*{\yellowhome}{\img[-0.8][-0.5][10][-0.5]{yellow_home_icon}}

\newcommand*{\redsquare}{\img[-0.8][-0.5][10][-0.5]{red_square_icon}}
\newcommand*{\smallredsquare}{\img[-0.8][-0.5][5][-0.5]{red_square_icon}}
\newcommand*{\bluesquare}{\img[-0.8][-0.5][10][-0.5]{blue_square_icon}}
\newcommand*{\bblacksquare}{\img[-0.8][-0.5][10][-0.5]{black_square_icon}}
\newcommand*{\yellowsquare}{\img[-0.8][-0.5][10][-0.5]{yellow_square_icon}}
\newcommand*{\transsquare}{\img[-0.8][-0.5][10][-0.5]{trans_square_icon}}

\newcommand*{\reddiamond}{\img[-0.8][-0.5][8][-0.5]{red_diamond_icon}}
\newcommand*{\bluediamond}{\img[-0.8][-0.5][8][-0.5]{blue_diamond_icon}}
\newcommand*{\blackdiamond}{\img[-0.8][-0.5][8][-0.5]{black_diamond_icon}}
\newcommand*{\yellowdiamond}{\img[-0.8][-0.5][8][-0.5]{yellow_diamond_icon}}
\newcommand*{\transdiamond}{\img[-0.8][-0.5][8][-0.5]{trans_diamond_icon}}

\newcommand*{\redtriangle}{\img[-0.8][-0.5][10][-0.5]{red_triangle_icon}}
\newcommand*{\bluetriangle}{\img[-0.8][-0.5][10][-0.5]{blue_triangle_icon}}
\newcommand*{\bblacktriangle}{\img[-0.8][-0.5][10][-0.5]{black_triangle_icon}}

\newcommand*{\blackrectangle}{\img[-0.8][-0.5][10][-0.5]{black_rectangle_icon}}
\newcommand*{\yellowrectangle}{\img[-0.8][-0.5][10][-0.5]{yellow_rectangle_icon}}






\begin{document}

\subsubsection{VII. Definition}

\begin{minipage}{\textwidth}

    \begin{center}
        \textit{DEFINITION VII.}\phantomsection\label{book5def7} \\
    \end{center}

    \hfill

    \raggedright W\textsc{HEN} of the equimultiples of four magnitudes (taken as in the fifth definition), the multiple of the firſt is greater than that of the ſecond, but the multiple of the third is not greater than the multiple of the fourth; then the firſt is ſaid to have to the ſecond a greater ratio than the third magnitude has to the fourth: and, on the contrary, the third is ſaid to have the fourth a leſs ratio than the firſt has to the ſecond.

    \hfill

    If, among the equimultiples of four magnitudes, compared as the in the \hyperref[book5def5]{fifth definition}, we ſhould find $\redcircle\hspace{-0.75ex}\redcircle\hspace{-0.75ex}\redcircle\hspace{-0.75ex}\redcircle\hspace{-0.75ex}\redcircle \bgt \yellowsquare\hspace{-0.75ex}\yellowsquare\hspace{-0.75ex}\yellowsquare\hspace{-0.75ex}\yellowsquare \bcomma$ but $\bluediamond\hspace{-0.75ex}\bluediamond\hspace{-0.75ex}\bluediamond\hspace{-0.75ex}\bluediamond\hspace{-0.75ex}\bluediamond \bequals$ or $\blt \bblacksquare\hspace{-0.75ex}\bblacksquare\hspace{-0.75ex}\bblacksquare\hspace{-0.75ex}\bblacksquare \bcomma$ or if we ſhould find any particular multiple $\text{M}'$ of the firſt and third, and a particular multiple $\textit{m}'$ of the ſecond and fourth, ſuch, that $\text{M}'$ times the firſt is $\bgt \textit{m}'$ times the ſecond, the $\text{M}'$ times the third is not $\bgt \textit{m}'$ times the fourth, i.e. $\bequals$ or $\blt \textit{m}'$ times the fourth; then the firſt is ſaid to have to the ſecond a greater ratio than the third has to the fourth; or the third has to the fourth, under ſuch circumſtances, a leſs ratio than the firſt has to the second: although ſeveral other equimultiples may tend to ſhow that the four magnitudes are proportionals.\\

    \hfill

    This definition will in future be expreſſed thus:---\\

    \hfill

    \begin{center}
        If $\text{M}' \redhome \bgt \textit{m}' \blackdome \bcomma$ but $\text{M}' \bluesquare \bequals$ or $\blt \textit{m}' \yellowdiamond \bcomma$\\
        then $\redhome \bcolon \blackdome \bgt \bluesquare \bcolon \yellowdiamond \bperiod$
    \end{center}

    \hfill

    In the above general expreſſion, $\text{M}'$ and $\textit{m}'$ are to be conſidered particular multiples, not like the multiples M and \textit{m} introduced in the fifth definition, which are in that definition conſidered to be every pair of multiples that can be taken. It muſt alſo be here obſerved, that $\redhome \bcomma \blackdome \bcomma \bluesquare \bcomma$ and the like ſymbols are to be conſidered merely the representatives of geometrical magnitudes.\\

    \hfill

    In a partial arithmetical way, this may be ſet forth as follows:\\

\end{minipage}

\newpage

\begin{minipage}{\textwidth}
    Let us take four numbers, {\color{cred}{8}}, 7, {\color{cblue}{10}}, and {\color{cyellow}{9}}.

    \hfill

    \begin{center}
        \begin{tabular}{ |c|c|c|c| }
            \hline
            \hfill                                    & \hfill                     & \hfill                                     & \hfill                                       \\
            \textit{Firſt.}                           & \textit{Second.}           & \textit{Third.}                            & \textit{Fourth.}                             \\
            {\color{cred}{8}}                         & 7                          & {\color{cblue}{10}}                        & {\color{cyellow}{9}}                         \\
            \hfill                                    & \hfill                     & \hfill                                     & \hfill                                       \\
            \hline
            \multicolumn{1}{|r|}{{\color{cred}{16}}}  & \multicolumn{1}{|r|}{14}   & \multicolumn{1}{|r|}{{\color{cblue}{20}}}  & \multicolumn{1}{|r|}{{\color{cyellow}{18}}}  \\
            \multicolumn{1}{|r|}{{\color{cred}{24}}}  & \multicolumn{1}{|r|}{21}   & \multicolumn{1}{|r|}{{\color{cblue}{30}}}  & \multicolumn{1}{|r|}{{\color{cyellow}{27}}}  \\
            \multicolumn{1}{|r|}{{\color{cred}{32}}}  & \multicolumn{1}{|r|}{28}   & \multicolumn{1}{|r|}{{\color{cblue}{40}}}  & \multicolumn{1}{|r|}{{\color{cyellow}{36}}}  \\
            \multicolumn{1}{|r|}{{\color{cred}{40}}}  & \multicolumn{1}{|r|}{35}   & \multicolumn{1}{|r|}{{\color{cblue}{50}}}  & \multicolumn{1}{|r|}{{\color{cyellow}{45}}}  \\
            \multicolumn{1}{|r|}{{\color{cred}{48}}}  & \multicolumn{1}{|r|}{42}   & \multicolumn{1}{|r|}{{\color{cblue}{60}}}  & \multicolumn{1}{|r|}{{\color{cyellow}{54}}}  \\
            \multicolumn{1}{|r|}{{\color{cred}{56}}}  & \multicolumn{1}{|r|}{49}   & \multicolumn{1}{|r|}{{\color{cblue}{70}}}  & \multicolumn{1}{|r|}{{\color{cyellow}{63}}}  \\
            \multicolumn{1}{|r|}{{\color{cred}{64}}}  & \multicolumn{1}{|r|}{56}   & \multicolumn{1}{|r|}{{\color{cblue}{80}}}  & \multicolumn{1}{|r|}{{\color{cyellow}{72}}}  \\
            \multicolumn{1}{|r|}{{\color{cred}{72}}}  & \multicolumn{1}{|r|}{63}   & \multicolumn{1}{|r|}{{\color{cblue}{90}}}  & \multicolumn{1}{|r|}{{\color{cyellow}{81}}}  \\
            \multicolumn{1}{|r|}{{\color{cred}{80}}}  & \multicolumn{1}{|r|}{70}   & \multicolumn{1}{|r|}{{\color{cblue}{100}}} & \multicolumn{1}{|r|}{{\color{cyellow}{90}}}  \\
            \multicolumn{1}{|r|}{{\color{cred}{88}}}  & \multicolumn{1}{|r|}{77}   & \multicolumn{1}{|r|}{{\color{cblue}{110}}} & \multicolumn{1}{|r|}{{\color{cyellow}{99}}}  \\
            \multicolumn{1}{|r|}{{\color{cred}{96}}}  & \multicolumn{1}{|r|}{84}   & \multicolumn{1}{|r|}{{\color{cblue}{120}}} & \multicolumn{1}{|r|}{{\color{cyellow}{108}}} \\
            \multicolumn{1}{|r|}{{\color{cred}{104}}} & \multicolumn{1}{|r|}{91}   & \multicolumn{1}{|r|}{{\color{cblue}{130}}} & \multicolumn{1}{|r|}{{\color{cyellow}{117}}} \\
            \multicolumn{1}{|r|}{{\color{cred}{112}}} & \multicolumn{1}{|r|}{98}   & \multicolumn{1}{|r|}{{\color{cblue}{140}}} & \multicolumn{1}{|r|}{{\color{cyellow}{126}}} \\
            \multicolumn{1}{|r|}{\&c;}                & \multicolumn{1}{|r|}{\&c;} & \multicolumn{1}{|r|}{\&c;}                 & \multicolumn{1}{|r|}{\&c;}                   \\
            \hline
        \end{tabular}
    \end{center}

    \hfill

    \hfill

    Among the above multiples we find {\color{cred}{16}} $\bgt$ 14 and {\color{cblue}{20}} $\bgt$ {\color{cyellow}{18}}; that is, twice the firſt is greater than twice the ſecond, and twice the third is greater than twice the fourth; and {\color{cred}{16}} $\bgt$ 21 and {\color{cblue}{20}} $\blt$ {\color{cyellow}{27}}; that is, twice the firſt is leſs than three times the ſecond, and twice the third is leſs than three times the fourth; and among the ſame multiples we can find {\color{cred}{72}} $\bgt$ 56 and {\color{cblue}{90}} $\bgt$ {\color{cyellow}{72}}; that is 9 times the firſt is greater than 8 times the ſecond, and 9 times the third is greater than 8 times the fourth. Many other equimultiples might be selected, which would tend to ſhow that the numbers {\color{cred}{8}}, 7, {\color{cblue}{10}}, {\color{cyellow}{9}}, were proportionals, but they are not, for we can find a multiple of the firſt $\bgt$ a multiple of the ſecond, but the ſame multiple of the third that has been taken of the firſt not $\bgt$ than the ſame multiple of the fourth which has been taken of the ſecond; for inſtance, 9 times the firſt is $\bgt$ 10 times the ſecond, but 9 times the third is not $\bgt$ 10 times the fourth, that is, {\color{cred}{72}} $\bgt$ 70, but {\color{cblue}{90}} not $\bgt$ {\color{cyellow}{90}}, or 8 times the firſt we find $\bgt$ 9 times the ſecond, but 8 times the third is not greater than 9 times the fourth, that is {\color{cred}{64}} $\bgt$ 63, but {\color{cblue}{80}} is not $\bgt$ {\color{cyellow}{81}}. When any ſuch multiples as theſe can be found, the first {\color{cred}{(8)}} is ſaid to have the ſecond (7) a greater ratio than the third {\color{cblue}{(10)}} has to the fourth {\color{cyellow}{(9)}}, and on the contrary the third {\color{cblue}{(10)}} is ſaid to have the fourth {\color{cyellow}{(9)}} a leſs ratio than the firſt {\color{cred}{(8)}} has to the ſecond (7).

\end{minipage}

\end{document}