% !TEX TS-program = xelatex
% !TEX options = -shell-escape -synctex=1 -interaction=nonstopmode -file-line-error "%DOC%"
\documentclass[11pt,preview]{standalone}

\usepackage{standalone}
\usepackage{fancyhdr}
\usepackage{xpatch}
\usepackage{titletoc}
\usepackage{setspace}
\usepackage{minibox}
\usepackage{enumitem}
\usepackage{fullpage}
\usepackage{mathtools,mathrsfs}
\usepackage{amssymb,amsthm}
\usepackage{graphicx,xcolor}
\usepackage[scale=2]{ccicons}
\usepackage[breakable]{tcolorbox}
\tcbuselibrary{breakable}
\usepackage{subfig}
\usepackage{float}
\usepackage{parskip}
\usepackage{lettrine}
\usepackage{fontspec}
\usepackage[compact]{titlesec}
\usepackage{calc}
\usepackage{xparse}
\usepackage{tikz}
\usepackage{svg}
\usepackage{wrapfig}
\usepackage{xr-hyper}
\usepackage[colorlinks=true, citecolor=violet, linkcolor=black, urlcolor=black]{hyperref}
\usepackage[
    papersize={6.25in,9.25in},
    margin=0.75in,
    layoutsize={6in,9in},
    layoutoffset={0.125in,0.125in},
    includehead,
    includefoot,
    showcrop,
]{geometry}

% Set this PATH to the root directory of your repository
\newcommand*{\MyPath}{/home/newell/code/byrne-euclid}%
% Don't show subsubsection titles -- these still show up in TOC (as desired)
\makeatletter
\titleformat{\subsubsection}[runin]{}{}{0pt}{\@gobble}
\makeatother

\pagestyle{fancy}
\fancyhf{}
\renewcommand{\headrulewidth}{0pt}
\fancyfoot[LO,RE]{\thepage}%

% TOC - image setup
\newcounter{propimage}
\makeatletter
\newcommand\stdsectioninToC{
    \titlecontents{subsubsection}[3.8em]
    {}%
    {\contentslabel{2.3em}}%
    {\hspace*{2.3em}}%
    {\titlerule*[1em]{.}\contentspage}
}
\newcommand\iconsectioninToC{
    \titlecontents{subsubsection}[3.8em]
    {\vskip 2ex}%
    {\hspace*{-2.3em}}%
    {
        \contentslabel{2.3em}%
        \stepcounter{propimage}%
        \smash{\includegraphics[width=50pt,height=25pt,keepaspectratio]{\MyPath/toc-images/image-\the\value{propimage}}}\hspace{0.5em}%
    }%
    {\titlerule*[1em]{.}\contentspage}%
}
\AtBeginDocument{\stdsectioninToC}
\makeatother

\setcounter{secnumdepth}{-3}
\setcounter{tocdepth}{3}

% Set the Font
\setmainfont{EBGaramond-Regular.otf}[
    BoldFont = EBGaramond-Bold.otf,
    ItalicFont = EBGaramond-Italic.otf,
    BoldItalicFont = EBGaramond-BoldItalic.otf,
    SmallCapsFont = EBGaramond12-AllSC.otf,
    Numbers = {OldStyle,Proportional},
    Ligatures = {Discretionary},
    RawFeature = {+ss06}
]
\newfontfamily\euclidinitials{EuclidInitialsnormal.ttf}

% Set the SVG paths (when I tried multiple lines to make this more readable it errored on me)
\svgpath{{\MyPath/symbols/}{\MyPath/byrne-euclid-svg/lines/}{\MyPath/byrne-euclid-svg/lines/points/}{\MyPath/byrne-euclid-svg/lines/arcs/}{\MyPath/byrne-euclid-svg/circles/}{\MyPath/byrne-euclid-svg/triangles/}{\MyPath/byrne-euclid-svg/figures/}{\MyPath/byrne-euclid-svg/figures/definitions/}{\MyPath/byrne-euclid-svg/figures/propositions/}{\MyPath/byrne-euclid-svg/angles/}{\MyPath/byrne-euclid-svg/symbols/}{\MyPath/byrne-euclid-svg/lines/}{\MyPath/byrne-euclid-svg/lines/points/}{\MyPath/byrne-euclid-svg/lines/arcs/}{\MyPath/byrne-euclid-svg/circles/}{\MyPath/byrne-euclid-svg/triangles/}{\MyPath/byrne-euclid-svg/figures/}{\MyPath/byrne-euclid-svg/figures/definitions/}{\MyPath/byrne-euclid-svg/figures/propositions/}{\MyPath/byrne-euclid-svg/angles/}{\MyPath/byrne-euclid-svg/icons/}}

% Create counter for lists.
\newcounter{listcounter}

% Colors
\definecolor{cred}{RGB}{212,42,32} % red
\definecolor{cyellow}{RGB}{250,194,43} % yellow
\definecolor{cblue}{RGB}{14,99,142} % blue
\definecolor{ctrans}{RGB}{252,243,217} % transparent
\definecolor{background}{HTML}{fcf3d9}

\newlength\mytemplena
\newlength\mytemplenb
\DeclareDocumentCommand\myalignalign{sm}
{
    \settowidth{\mytemplena}{$\displaystyle #2$}%
    \setlength\mytemplenb{\widthof{$\displaystyle=$}/2}%
    \hskip-\mytemplena%
    \hskip\IfBooleanTF#1{-\mytemplenb}{+\mytemplenb}%
}

% Images
\NewDocumentCommand{\img}{%
    O{0}
    O{0}
    O{25}
    O{0}
    m
}{%
    \begin{array}{c}
        \vspace{#1ex}\hspace{#2ex}\includesvg[width=#3pt]{#5}\hspace{#4ex}
    \end{array}
}

% Common Symbols
\newcommand*{\bplus}{\img[-0.8][0][10]{plus}}
\newcommand*{\bminus}{\img[0.8][0][10]{minus}}
\newcommand*{\bcross}{\img[-0.85][0][10]{cross}}
\newcommand*{\bequals}{\img[-0.1][0][10]{equals}}
\newcommand*{\bnequals}{\img[-0.85][0][10]{nequals}}
\newcommand*{\bparallel}{\img[-0.85][0][7]{parallel}}
\newcommand*{\bnparallel}{\img[-0.85][0][11]{not_parallel}}
\newcommand*{\bperiod}{\img[-0.5][-1][3]{period}}
\newcommand*{\bmark}{\img[-0.5][-1][3]{mark}}
\newcommand*{\bmultiply}{\img[1][0][3]{period}}
\newcommand*{\bsemicolon}{\img[-1.4][-1][3]{semicolon}}
\newcommand*{\bcolon}{\img[-0.5][0][3]{colon}}
\newcommand*{\bbcolon}{\bcolon\hspace{-1.5ex}\bcolon}
\newcommand*{\bcomma}{\img[-1.5][-1][3]{comma}}
\newcommand*{\btherefore}{\img[-0.5][0][10]{therefore}}
\newcommand*{\bgt}{\img[-0.85][0][10]{greater_than}}
\newcommand*{\blt}{\img[-0.85][0][10]{less_than}}
\newcommand*{\bperp}{\img[0][0][15]{perpendicular}}

% Common Lines
\newcommand*{\bimg}[1]{\img[0.8][0][45]{#1}}
\newcommand*{\redlines}{\bimg{red_lines}}
\newcommand*{\bluelines}{\bimg{blue_lines}}
\newcommand*{\blacklines}{\bimg{black_lines}}

\newcommand*{\redline}{\img[0.8][0][30]{red_line}}
\newcommand*{\redthinline}{\img[0.8][0][30]{red_thin_line}}
\newcommand*{\dottedredline}{\img[0.8][0][30]{red_dotted_line}}
\newcommand*{\blueline}{\img[0.8][0][30]{blue_line}}
\newcommand*{\bluethinline}{\img[0.8][0][30]{blue_thin_line}}
\newcommand*{\dottedblueline}{\img[0.8][0][30]{blue_dotted_line}}
\newcommand*{\blackline}{\img[0.8][0][30]{black_line}}
\newcommand*{\blackthinline}{\img[0.8][0][30]{black_thin_line}}
\newcommand*{\blackthindottedline}{\img[0.8][0][30]{black_thin_dotted_line}}
\newcommand*{\dottedblackline}{\img[0.8][0][30]{black_dotted_line}}
\newcommand*{\yellowline}{\img[0.8][0][30]{yellow_line}}
\newcommand*{\yellowthinline}{\img[0.8][0][30]{yellow_thin_line}}
\newcommand*{\dottedyellowline}{\img[0.8][0][30]{yellow_dotted_line}}

% Common Angles
\newcommand*{\tworightangles}{\img[0][0][30]{two_right_angles}}

% Common Shapes -- Book 5
\newcommand*{\reddome}{\img[-0.8][-0.5][10][-0.5]{red_dome_icon}}
\newcommand*{\bluedome}{\img[-0.8][-0.5][10][-0.5]{blue_dome_icon}}
\newcommand*{\blackdome}{\img[-0.8][-0.5][10][-0.5]{black_dome_icon}}
\newcommand*{\yellowdome}{\img[-0.8][-0.5][10][-0.5]{yellow_dome_icon}}

\newcommand*{\redcircle}{\img[-0.8][-0.5][10][-0.5]{red_circle_icon}}
\newcommand*{\bluecircle}{\img[-0.8][-0.5][10][-0.5]{blue_circle_icon}}
\newcommand*{\blackcircle}{\img[-0.8][-0.5][10][-0.5]{black_circle_icon}}
\newcommand*{\yellowcircle}{\img[-0.8][-0.5][10][-0.5]{yellow_circle_icon}}

\newcommand*{\reddrop}{\img[-0.8][-0.5][10][-0.5]{red_drop_icon}}
\newcommand*{\bluedrop}{\img[-0.8][-0.5][10][-0.5]{blue_drop_icon}}
\newcommand*{\blackdrop}{\img[-0.8][-0.5][10][-0.5]{black_drop_icon}}
\newcommand*{\yellowdrop}{\img[-0.8][-0.5][10][-0.5]{yellow_drop_icon}}

\newcommand*{\redhome}{\img[-0.8][-0.5][10][-0.5]{red_home_icon}}
\newcommand*{\bluehome}{\img[-0.8][-0.5][10][-0.5]{blue_home_icon}}
\newcommand*{\blackhome}{\img[-0.8][-0.5][10][-0.5]{black_home_icon}}
\newcommand*{\yellowhome}{\img[-0.8][-0.5][10][-0.5]{yellow_home_icon}}

\newcommand*{\redsquare}{\img[-0.8][-0.5][10][-0.5]{red_square_icon}}
\newcommand*{\smallredsquare}{\img[-0.8][-0.5][5][-0.5]{red_square_icon}}
\newcommand*{\bluesquare}{\img[-0.8][-0.5][10][-0.5]{blue_square_icon}}
\newcommand*{\bblacksquare}{\img[-0.8][-0.5][10][-0.5]{black_square_icon}}
\newcommand*{\yellowsquare}{\img[-0.8][-0.5][10][-0.5]{yellow_square_icon}}
\newcommand*{\transsquare}{\img[-0.8][-0.5][10][-0.5]{trans_square_icon}}

\newcommand*{\reddiamond}{\img[-0.8][-0.5][8][-0.5]{red_diamond_icon}}
\newcommand*{\bluediamond}{\img[-0.8][-0.5][8][-0.5]{blue_diamond_icon}}
\newcommand*{\blackdiamond}{\img[-0.8][-0.5][8][-0.5]{black_diamond_icon}}
\newcommand*{\yellowdiamond}{\img[-0.8][-0.5][8][-0.5]{yellow_diamond_icon}}
\newcommand*{\transdiamond}{\img[-0.8][-0.5][8][-0.5]{trans_diamond_icon}}

\newcommand*{\redtriangle}{\img[-0.8][-0.5][10][-0.5]{red_triangle_icon}}
\newcommand*{\bluetriangle}{\img[-0.8][-0.5][10][-0.5]{blue_triangle_icon}}
\newcommand*{\bblacktriangle}{\img[-0.8][-0.5][10][-0.5]{black_triangle_icon}}

\newcommand*{\blackrectangle}{\img[-0.8][-0.5][10][-0.5]{black_rectangle_icon}}
\newcommand*{\yellowrectangle}{\img[-0.8][-0.5][10][-0.5]{yellow_rectangle_icon}}






\begin{document}

\subsubsection{VIII. Theorem}

\newcommand{\triandsquare}{\hspace{-1.5ex}\begin{tabular}{c} $\blacktriangle$ \\ $\redsquare$ \end{tabular}\hspace{-1ex}}

\begin{minipage}{\textwidth}
    \begin{center}
        \textit{PROPOSITION VIII. THEOREM.}\phantomsection\label{book5pr8} \\
    \end{center}

    \hfill

    \begin{center}
        \raggedright \lettrine[lines=3, loversize=1, nindent=0pt]{\euclidinitials{I}}{}F \textit{unequal magnitudes the greater has a greater ratio to the ſame\\ than the leſs has: and the ſame magnitude has a greater ratio to\\ the leſs than it has to the greater}.
    \end{center}
\end{minipage}

\hfill

\hfill

\begin{center}
    Let \triandsquare and $\yellowsquare$ be two unequal magnitudes, and $\bluecircle$ and other.
\end{center}

We ſhall firſt prove that \triandsquare which is the greater of the two unequal magnitudes, has a greater ratio to $\bluecircle$ than $\yellowsquare \bcomma$ the leſs, has to $\bluecircle \bsemicolon$\\

\begin{center}
    that is, $\triandsquare \bcolon \bluecircle \bgt \yellowsquare \bcolon \bluecircle \bsemicolon$\\
    \hfill\\
    take $\text{M}' \triandsquare \bcomma\ \textit{m}'\ \bluecircle \bcomma\ \text{M}'\ \yellowsquare \bcomma$ and $\textit{m}'\ \bluecircle \bsemicolon$\\
    ſuch, that $\text{M}'\ \blacktriangle$ and $\text{M}'\ \redsquare$ ſhall be each $\bgt \bluecircle \bsemicolon$\\
    alſo take $\textit{m}'\ \bluecircle$ the leaſt multiple of $\bluecircle \bcomma$\\
    which will make $\textit{m}'\ \bluecircle \bgt \text{M}'\ \yellowsquare \bequals \text{M}'\ \redsquare \bsemicolon$\\
    $\btherefore \text{M}'\ \yellowsquare$ is not $\bgt \textit{m}'\ \bluecircle \bcomma$
\end{center}

\begin{center}
    but $\text{M}'\ \triandsquare$ is $\bgt \textit{m}'\ \bluecircle \bcomma$ for,\\
    as $\textit{m}'\ \bluecircle$ is the firſt multiple which firſt becomes $\bgt \text{M}'\ \redsquare \bcomma$ than ($\textit{m}'$ minus 1) $\bluecircle$ or $\textit{m}'\ \bluecircle$\\
    minus $\bluecircle$ is not $\bgt \text{M}'\ \redsquare \bcomma$ and $\bluecircle$ is not $\bgt \text{M}'\ \blacktriangle \bcomma$\\
    $\btherefore \textit{m}'$ minus $\bluecircle \bplus \bluecircle$ muſt be $\blt \text{M}'\ \redsquare \bplus \text{M}' \blacktriangle \bsemicolon$\\
    that is, $\textit{m}'\ \bluecircle$ muſt be $\blt \text{M}' \redsquare \bsemicolon$\\
\end{center}

\begin{center}
    $\btherefore \text{M}'\ \triandsquare$ is $\bgt \textit{m}'\ \bluecircle \bsemicolon$ but it has been ſhown above that\\
    $\text{M}'\ \yellowsquare$ is not $\bgt \textit{m}'\ \bluecircle \bcomma$ therefore, by the \hyperref[book5def7]{ſeventh definition},\\
    \triandsquare has to $\bluecircle$ a greater ratio than $\yellowsquare \bcolon \bluecircle \bperiod$
    \hfill\\
    Next we ſhall prove that $\bluecircle$ has a greater ratio to $\yellowsquare \bcomma$ the leſs than it has to $\triandsquare \bcomma$ the greater;\\
    or, $\bluecircle \bcolon \yellowsquare \bgt \bluecircle \bcolon \triandsquare \bperiod$\\
    \hfill\\
    Take $\textit{m}'\ \bluecircle \bcomma\ \text{M}'\ \yellowsquare \bcomma\ \textit{m}'\ \bluecircle \bcomma$ and $\text{M}'\ \triandsquare \bcomma$\\
    the ſame as in the firſt caſe, ſuch that\\
    $\text{M}' \blacktriangle$ and $\text{M}'\ \redsquare$ will be each $\bgt \bluecircle \bcomma$ and $\textit{m}'\ \bluecircle$ the leaſt multiple of $\bluecircle \bcomma$ which firſt becomes\\
    greater than $\text{M}'\ \redsquare \bequals \text{M}'\ \yellowsquare \bperiod$\\
    \hfill\\
    $\btherefore \textit{m}'\ \bluecircle$ minus $\bluecircle$ is not $\bgt \text{M}'\ \redsquare \bcomma$\\
    and $\bluecircle$ is not $\bgt \text{M}' \blacktriangle \bsemicolon$ conſequently\\
    $\textit{m}'\ \bluecircle$ minus $\bluecircle \bplus \bluecircle$ is $\blt \text{M}'\ \redsquare \bplus \text{M}'\ \blacktriangle \bsemicolon$\\
    \hfill\\
    $\btherefore \textit{m}'$ is $\blt \text{M}'\ \triandsquare \bcomma$ and $\btherefore$ by the \hyperref[book5def7]{ſeventh definition},\\
    $\bluecircle$ has to $\yellowsquare$ a greater ratio than $\bluecircle$ has to $\triandsquare \bperiod$
\end{center}

$\btherefore$ Of unequal magnitudes, \&c.

\raggedright The contrivance employed in this propoſition for finding among multiples taken, as in the \hyperref[book5def5]{fifth definition}, a multiple of the firſt greater than the multiple of the ſecond, but the ſame multiple of the third which has been taken of the firſt, not greater than the ſame multiple of the fourth which has been taken of the ſecond, may be illuſtrated numerically as follows:---\\

\hfill

The number 9 has a greater ratio to {\color{cblue}{7}} than {\color{cyellow}{8}} has to {\color{cblue}{7}}: that is, 9 : {\color{cblue}{7}} $\bgt$ {\color{cyellow}{8}} : {\color{cblue}{7}} $\bgt$ {\color{cyellow}{8}} : {\color{cblue}{7}}; or, {\color{cred}{8}} $\bplus$ 1 : {\color{cblue}{7}} $\bgt$ {\color{cyellow}{8}} : {\color{cblue}{7}}.\\

\hfill

The multiple of 1, which firſt becomes greater than {\color{cblue}{7}}, is 8 times, therefore we may multiply the firſt and third by 8, 9, 10, or any other greater number; in this caſe, let us multiply the firſt and third by 8, and we have {\color{cred}{64}} $\bplus$ 8 and {\color{cyellow}{64}}: again, the firſt multiple of {\color{cblue}{7}} which becomes greater than 64 is 10 times; then, by multiplying the ſecond and fourth by 10, we ſhall have {\color{cblue}{70}} and {\color{cblue}{70}}; then, arranging theſe multiples, we have---\\

\begin{center}
    $\overset{\text{8 times the first.}}{{\color{cred}{64}} \bplus 8} \qquad\qquad \overset{\text{10 times the second.}}{{\color{cblue}{70}}} \qquad\qquad \overset{\text{8 times the third.}}{{\color{cyellow}{64}}} \qquad\qquad \overset{\text{10 times the fourth.}}{{\color{cblue}{70}}}$\\
\end{center}

Conſequently, {\color{cred}{64}} $\bplus$ 8, or 72, is greater than {\color{cblue}{70}}, but {\color{cyellow}{64}} is not greater than {\color{cblue}{70}}, $\therefore$ by the \hyperref[book5def7]{ſeventh definition}, 9 has a greater ratio to {\color{cblue}{7}} than {\color{cyellow}{8}} has to {\color{cblue}{7}}.\\

\hfill

The above is merely illuſtrative of the foregoing demonſtration, for this property could be ſhown of theſe or other numbers very readily in the following manner; becauſe if an antecedent contains it conſequent a greater number of times than another antecedent contains its conſequent, or when a fraction is formed of an antecedent for the numerator, and its conſequent for the denominator be greater than another fraction which is formed of another antecedent for the numerator and its conſequent for the denominator, the ratio of the firſt antecedent to its conſequent is greater than the ratio of the laſt antecedent to its conſequent.\\

\hfill

Thus, the number 9 has a greater ratio to 7, than 8 has to 7, for $\dfrac{\text{9}}{\text{7}}$ is greater than $\dfrac{\text{8}}{\text{7}}$.\\

\hfill

Again, 17 : 19 is a greater ratio than 13 : 15, becauſe $\dfrac{\text{17}}{\text{19}} \bequals \dfrac{\text{17} \times \text{15}}{\text{19} \times \text{15}} \bequals \dfrac{\text{255}}{\text{285}}$, and $\dfrac{\text{13}}{\text{15}} \bequals \dfrac{\text{13} \times \text{19}}{\text{15} \times \text{19}} \bequals \dfrac{\text{247}}{\text{285}}$, hence it is evident that $\dfrac{\text{255}}{\text{285}}$ is greater than $\dfrac{\text{247}}{\text{285}}$, $\therefore \dfrac{\text{17}}{\text{19}}$ is greater than $\dfrac{\text{13}}{\text{15}}$, and, according to what has been above ſhown, 17 has to 19 a greater ratio than 13 has to 15.\\

\hfill

So that the general terms upon which a greater, equal, or leſs ratio exiſts are as follows:---\\

\hfill

If $\dfrac{\text{A}}{\text{B}}$ be greater than $\dfrac{\text{C}}{\text{D}}$, A is ſaid to have to B a greater ratio than C has to D; if $\dfrac{\text{A}}{\text{B}}$ equal to $\dfrac{\text{C}}{\text{D}}$, then A has to B the ſame ratio which C has to D; and if $\dfrac{\text{A}}{\text{B}}$ be leſs than $\dfrac{\text{C}}{\text{D}}$, A is ſaid to have to B a leſs ratio than C has to D.\\

\hfill

The ſtudent ſhould underſtand all up to this propoſition perfectly before proceeding further, in order to fully comprehend the following propositions in of this book. We therefore ſtrongly recommend the learner to commence again, and read up to this ſlowly, and carefully reaſon at each ſtep, as he proceeds, particularly guarding againſt the miſchievous ſyſtem of depending wholly on the memory. By following theſe inſtructions, he will find that the parts which uſually preſent conſiderable difficulties will preſent no difficulties whatever, in proſecuting the ſtudy of this important book.

\end{document}