\pagestyle{fancy}
\fancyhf{}
\renewcommand{\headrulewidth}{0pt}
\fancyfoot[LE,RO]{\textsc{I.} \thepage}%

\begin{minipage}{0.33\textwidth}
    \phantom{}
\end{minipage}
\begin{minipage}{0.67\textwidth}
    \null\removelastskip\nointerlineskip\vspace*{-\baselineskip}
    \section[Book I]{\centering THE ELEMENTS OF EUCLID\\ BOOK I.}
    \label{sec:book1}

    \hfill

    \subsection[Definitions]{\centering \scshape{\LARGE{DEFINITIONS.}}}
    \label{subsec:definitions}

    \hfill

    \subsubsection{def. 1}
    \begin{center}
        I.\phantomsection\label{book1def1}
        \hfill\\
        A \textit{point} is that which has no part.\\
    \end{center}
    \subsubsection{def. 2}
    \begin{center}
        II.\phantomsection\label{book1def2}\\
        \hfill\\
        A \textit{line} is length without breadth.\\
    \end{center}
    \subsubsection{def. 3}
    \begin{center}
        III.\phantomsection\label{book1def3}\\
        \hfill\\
        The extremities of a line are points.\\
    \end{center}
    \subsubsection{def. 4}
    \begin{center}
        IV.\phantomsection\label{book1def4}\\
        \hfill\\
        A ſtraight or right line is that which lies evenly between its \mbox{extremities}.
    \end{center}
    \subsubsection{def. 5}
    \begin{center}
        V.\phantomsection\label{book1def5}\\
        \hfill\\\
        A ſurface is that which has length and breadth only.\\
    \end{center}
    \subsubsection{def. 6}
    \begin{center}
        VI.\phantomsection\label{book1def6}\\
        \hfill\\
        The extremities of a ſurface are lines.\\
    \end{center}
\end{minipage}

\hfill

\begin{minipage}{0.67\textwidth}
    \subsubsection{def. 7}
    \begin{center}
        VII.\phantomsection\label{book1def7}\\
        \hfill\\
        A plane ſurface is that which lies evenly between its extremities.\\
    \end{center}
    \subsubsection{def. 8}
    \begin{center}
        VIII.\phantomsection\label{book1def8}\\
        \hfill\\
        A plane angle is the inclination of two lines to one another, in a plane, which meet together, but are not in the ſame direction.
    \end{center}
\end{minipage}%
\begin{minipage}{0.33\textwidth}
    \phantom{}
\end{minipage}

\hfill

\begin{minipage}{0.67\textwidth}
    \subsubsection{def. 9}
    \begin{center}
        IX.\phantomsection\label{book1def9}\\
        \hfill\\
        A plane rectilinear angle is the inclination of two ſtraight lines to one another, which meet together, but are not in the ſame ſtraight line.
    \end{center}
\end{minipage}%
\begin{minipage}{0.33\textwidth}
    \begin{center}
        $\img[0][0][70]{book1_definition_9_figure}$
    \end{center}
\end{minipage}

\hfill

\begin{minipage}{0.67\textwidth}
    \subsubsection{def. 10}
    \begin{center}
        X.\phantomsection\label{book1def10}\\
        \hfill\\
        When one ſtraight line ſtanding on another ſtraight line makes the adjacent angles equal, each of theſe angles is called a \textit{right angle}, and each of theſe lines is ſaid to be \textit{perpendicular} to the other.
    \end{center}
\end{minipage}%
\begin{minipage}{0.33\textwidth}
    \begin{center}
        $\img[10][0][70]{book1_definition_10_figure}$
    \end{center}
\end{minipage}

\hfill

\begin{minipage}{0.67\textwidth}
    \subsubsection{def. 11}
    \begin{center}
        XI.\phantomsection\label{book1def11}\\
        \hfill\\
        An obtuſe angle is an angle greater than a right angle.
    \end{center}
\end{minipage}%
\begin{minipage}{0.33\textwidth}
    \begin{center}
        $\img[0][0][70]{book1_definition_11_figure}$
    \end{center}
\end{minipage}

\hfill

\begin{minipage}{0.33\textwidth}
    \begin{center}
        $\img[0][0][70]{book1_definition_12_figure}$
    \end{center}
\end{minipage}%
\begin{minipage}{0.67\textwidth}
    \subsubsection{def. 12}
    \begin{center}
        XII.\phantomsection\label{book1def12}\\
        \hfill\\
        An acute angle is leſs than a right angle.
    \end{center}
\end{minipage}

\hfill

\begin{minipage}{0.33\textwidth}
    \phantom{}
\end{minipage}%
\begin{minipage}{0.67\textwidth}
    \subsubsection{def. 13}
    \begin{center}
        XIII.\phantomsection\label{book1def13}\\
        \hfill\\
        A term or boundary is the extremity of any thing.
    \end{center}
    \subsubsection{def. 14}
    \begin{center}
        XIV.\phantomsection\label{book1def14}\\
        \hfill\\
        A figure is a ſurface encloſed on all ſides by a line or lines.
    \end{center}
\end{minipage}

\hfill

\begin{minipage}{0.33\textwidth}
    \begin{center}
        $\img[0][0][70]{book1_definition_15_figure}$
    \end{center}
\end{minipage}%
\begin{minipage}{0.67\textwidth}
    \subsubsection{def. 15}
    \begin{center}
        XV.\phantomsection\label{book1def15}\\
        \hfill\\
        A circle is a plane figure, bounded by one continued line, called its circumference or periphery; and having a certain point within it, from which all ſtraight lines drawn to its circumference are equal.
    \end{center}
\end{minipage}

\hfill

\begin{minipage}{0.33\textwidth}
    \phantom{}
\end{minipage}%
\begin{minipage}{0.67\textwidth}
    \subsubsection{def. 16}
    \begin{center}
        XVI.\phantomsection\label{book1def16}\\
        \hfill\\
        This point (from which the equal lines are drawn) is called the \mbox{centre} of the circle.
    \end{center}
\end{minipage}

\hfill

\begin{minipage}{0.67\textwidth}
    \subsubsection{def. 17}
    \begin{center}
        XVII.\phantomsection\label{book1def17}\\
        \hfill\\
        A diameter of a circle is a ſtraight line drawn through the centre, terminated both ways in the circumference.
    \end{center}
\end{minipage}%
\begin{minipage}{0.33\textwidth}
    \begin{center}
        $\img[0][0][70]{book1_definition_17_figure}$
    \end{center}
\end{minipage}

\hfill

\begin{minipage}{0.67\textwidth}
    \subsubsection{def. 18}
    \begin{center}
        XVIII.\phantomsection\label{book1def18}\\
        \hfill\\
        A ſemicircle is the figure contained by the diameter, and the part of the circle cut off by the diameter.
    \end{center}
\end{minipage}%
\begin{minipage}{0.33\textwidth}
    \begin{center}
        $\img[0][0][70]{book1_definition_18_figure}$
    \end{center}
\end{minipage}

\hfill

\begin{minipage}{0.67\textwidth}
    \subsubsection{def. 19}
    \begin{center}
        XIX.\phantomsection\label{book1def19}\\
        \hfill\\
        A ſegment of a circle is a figure contained by a ſtraight line, and the part of the circumference which it cuts off.
    \end{center}
\end{minipage}%
\begin{minipage}{0.33\textwidth}
    \begin{center}
        $\img[0][0][70]{book1_definition_19_figure}$
    \end{center}
\end{minipage}

\hfill

\begin{minipage}{0.67\textwidth}
    \subsubsection{def. 20}
    \begin{center}
        XX.\phantomsection\label{book1def20}\\
        \hfill\\
        A figure contained by ſtraight lines only, is called a rectilinear figure.\\
    \end{center}
\end{minipage}

\hfill

\begin{minipage}{0.67\textwidth}
    \subsubsection{def. 21}
    \begin{center}
        XXI.\phantomsection\label{book1def21}\\
        \hfill\\
        A triangle is a rectilinear figure included by three ſides.\\
    \end{center}
\end{minipage}%
\begin{minipage}{0.33\textwidth}
    \phantom{}
\end{minipage}

\hfill

\begin{minipage}{0.33\textwidth}
    \begin{center}
        $\img[0][0][70]{book1_definition_22_figure}$
    \end{center}
\end{minipage}%
\begin{minipage}{0.67\textwidth}
    \subsubsection{def. 22}
    \begin{center}
        XXII.\phantomsection\label{book1def22}\\
        \hfill\\
        A quadrilateral figure is one which is bounded by four ſides. The ſtraight lines $\blueline$ and $\redline$ connecting the vertices of the oppoſite angles of a quadrilateral figure, are called its diagonal.
    \end{center}
\end{minipage}

\hfill

\begin{minipage}{0.33\textwidth}
    \phantom{}
\end{minipage}%
\begin{minipage}{0.67\textwidth}
    \subsubsection{def. 23}
    \begin{center}
        XXIII.\phantomsection\label{book1def23}\\
        \hfill\\
        A polygon is a rectilinear figure bounded by more than four ſides.\\
    \end{center}
\end{minipage}

\hfill

\begin{minipage}{0.33\textwidth}
    \begin{center}
        $\img[0][0][70]{book1_definition_24_figure}$
    \end{center}
\end{minipage}%
\begin{minipage}{0.67\textwidth}
    \subsubsection{def. 24}
    \begin{center}
        XXIV.\phantomsection\label{book1def24}\\
        \hfill\\
        A triangle whoſe three ſides are equal, is ſaid to be equilateral.
    \end{center}
\end{minipage}

\hfill

\begin{minipage}{0.33\textwidth}
    \begin{center}
        $\img[0][0][70]{book1_definition_25_figure}$
    \end{center}
\end{minipage}%
\begin{minipage}{0.67\textwidth}
    \subsubsection{def. 25}
    \begin{center}
        XXV.\phantomsection\label{book1def25}\\
        \hfill\\
        A triangle which has only two ſides equal is called an iſoſceles \mbox{triangle}.
    \end{center}
\end{minipage}

\hfill

\begin{minipage}{0.33\textwidth}
    \phantom{}
\end{minipage}%
\begin{minipage}{0.67\textwidth}
    \subsubsection{def. 26}
    \begin{center}
        XXVI.\phantomsection\label{book1def26}\\
        \hfill\\
        A ſcalene triangle is one which has no two ſides equal.\\
    \end{center}
\end{minipage}

\hfill

\begin{minipage}{0.67\textwidth}
    \subsubsection{def. 27}
    \begin{center}
        XXVII.\phantomsection\label{book1def27}\\
        \hfill\\
        A right angled triangle is that which has a right angle.
    \end{center}
\end{minipage}%
\begin{minipage}{0.33\textwidth}
    \begin{center}
        $\img[0][0][70]{book1_definition_27_figure}$
    \end{center}
\end{minipage}

\hfill

\begin{minipage}{0.67\textwidth}
    \subsubsection{def. 28}
    \begin{center}
        XXVIII.\phantomsection\label{book1def28}\\
        \hfill\\
        An obtuſe angled triangle is that which has an obtuſe angle.
    \end{center}
\end{minipage}%
\begin{minipage}{0.33\textwidth}
    \begin{center}
        $\img[0][0][70]{book1_definition_28_figure}$
    \end{center}
\end{minipage}

\hfill

\begin{minipage}{0.67\textwidth}
    \subsubsection{def. 29}
    \begin{center}
        XXIX.\phantomsection\label{book1def29}\\
        \hfill\\
        An acute angled triangle is that which has three acute angles.
    \end{center}
\end{minipage}%
\begin{minipage}{0.33\textwidth}
    \begin{center}
        $\img[0][0][70]{book1_definition_29_figure}$
    \end{center}
\end{minipage}

\hfill

\begin{minipage}{0.67\textwidth}
    \subsubsection{def. 30}
    \begin{center}
        XXX.\phantomsection\label{book1def30}\\
        \hfill\\
        Of four-ſided figures, a ſquare is that which has all its ſides equal, and all its angles right angles.
    \end{center}
\end{minipage}%
\begin{minipage}{0.33\textwidth}
    \begin{center}
        $\img[0][0][70]{book1_definition_30_figure}$
    \end{center}
\end{minipage}

\hfill

\begin{minipage}{0.67\textwidth}
    \subsubsection{def. 31}
    \begin{center}
        XXXI.\phantomsection\label{book1def31}\\
        \hfill\\
        A rhombus is that which has all its ſides equal, but its angles are not right angles.
    \end{center}
\end{minipage}%
\begin{minipage}{0.33\textwidth}
    \begin{center}
        $\img[0][0][70]{book1_definition_31_figure}$
    \end{center}
\end{minipage}

\hfill

\begin{minipage}{0.33\textwidth}
    \begin{center}
        $\img[0][0][70]{book1_definition_32_figure}$
    \end{center}
\end{minipage}%
\begin{minipage}{0.67\textwidth}
    \subsubsection{def. 32}
    \begin{center}
        XXXII.\phantomsection\label{book1def32}\\
        \hfill\\
        An oblong is that which has all its angles right angles, but has not all its ſides equal.
    \end{center}
\end{minipage}

\hfill

\begin{minipage}{0.33\textwidth}
    \begin{center}
        $\img[0][0][70]{book1_definition_33_figure}$
    \end{center}
\end{minipage}%
\begin{minipage}{0.67\textwidth}
    \subsubsection{def. 33}
    \begin{center}
        XXXIII.\phantomsection\label{book1def33}\\
        \hfill\\
        A rhomboid is that which has its oppoſite ſides equal to one another, but all its ſides are not equal, nor its angles right angles.
    \end{center}
\end{minipage}

\hfill

\begin{minipage}{0.33\textwidth}
    \phantom{}
\end{minipage}%
\begin{minipage}{0.67\textwidth}
    \subsubsection{def. 34}
    \begin{center}
        XXXIV.\phantomsection\label{book1def34}\\
        \hfill\\
        All other quadrilateral figures are called trapeziums.\\
    \end{center}
\end{minipage}

\hfill

\begin{minipage}{0.33\textwidth}
    \begin{center}
        $\img[0][0][70]{book1_definition_35_figure}$
    \end{center}
\end{minipage}%
\begin{minipage}{0.67\textwidth}
    \subsubsection{def. 35}
    \begin{center}
        XXXV.\phantomsection\label{book1def35}\\
        \hfill\\
        Parallel ſtraight lines are ſuch as are in the ſame plane, and which being produced continually in both directions, would never meet.
    \end{center}
\end{minipage}

\hfill

\hfill

\begin{minipage}{0.67\textwidth}
    \subsection[Postulates]{\centering \scshape{\LARGE{POSTULATES.}}}
    \label{subsec:postulates}

    \hfill

    \subsubsection{poſt. 1}
    \begin{center}
        I.\phantomsection\label{post1}\\
        \hfill\\
        Let it be granted that a ſtraight line may be drawn from any one point to any other point.
    \end{center}
    \subsubsection{poſt. 2}
    \begin{center}
        II.\phantomsection\label{post2}\\
        \hfill\\
        Let it be granted that a finite ſtraight line may be produced to any length in a ſtraight line.
    \end{center}
    \subsubsection{poſt. 3}
    \begin{center}
        III.\phantomsection\label{post3}\\
        \hfill\\
        Let it be granted that a circle may be deſcribed with any centre at any diſtance from that centre.
    \end{center}
\end{minipage}

\hfill

\begin{minipage}{0.67\textwidth}
    \subsection[Axioms]{\centering \scshape{\LARGE{AXIOMS.}}}
    \label{subsec:axioms}

    \hfill

    \subsubsection{ax. 1}
    \begin{center}
        I.\phantomsection\label{ax1}\\
        \hfill\\
        Magnitudes which are equal to the ſame are equal to each other.\\
    \end{center}
    \subsubsection{ax. 2}
    \begin{center}
        II.\phantomsection\label{ax2}\\
        \hfill\\
        If equals be added to equals the ſums will be equal.\\
    \end{center}
\end{minipage}

\hfill

\begin{minipage}{0.33\textwidth}
    \phantom{}
\end{minipage}%
\begin{minipage}{0.67\textwidth}
    \subsubsection{ax. 3}
    \begin{center}
        III.\phantomsection\label{ax3}\\
        \hfill\\
        If equals be taken away from equals the remainders will be equal.\\
    \end{center}
    \subsubsection{ax. 4}
    \begin{center}
        IV.\phantomsection\label{ax4}\\
        \hfill\\
        If equals be added to unequals the ſums will be unequal.\\
    \end{center}
    \subsubsection{ax. 5}
    \begin{center}
        V.\phantomsection\label{ax5}\\
        \hfill\\
        If equals be taken away from unequals the remainders will be \mbox{unequal}.
    \end{center}
    \subsubsection{ax. 6}
    \begin{center}
        VI.\phantomsection\label{ax6}\\
        \hfill\\
        The doubles of the ſame or equal magnitudes are equal.\\
    \end{center}
    \subsubsection{ax. 7}
    \begin{center}
        VII.\phantomsection\label{ax7}\\
        \hfill\\
        The halves of the ſame or equal magnitudes are equal.\\
    \end{center}
    \subsubsection{ax. 8}
    \begin{center}
        VIII.\phantomsection\label{ax8}\\
        \hfill\\
        Magnitudes which coincide with one another, or exactly fill the ſame ſpace, are equal.
    \end{center}
\end{minipage}

\hfill

\begin{minipage}{0.67\textwidth}
    \subsubsection{ax. 9}
    \begin{center}
        IX.\phantomsection\label{ax9}\\
        \hfill\\
        The whole is greater than its part.\\
    \end{center}
    \subsubsection{ax. 10}
    \begin{center}
        X.\phantomsection\label{ax10}\\
        \hfill\\
        Two ſtraight lines cannot include a ſpace.\\
    \end{center}
    \subsubsection{ax. 11}
    \begin{center}
        XI.\phantomsection\label{ax11}\\
        \hfill\\
        All right angles are equal.\\
    \end{center}
\end{minipage}%
\begin{minipage}{0.33\textwidth}
    \phantom{}
\end{minipage}

\begin{minipage}{0.67\textwidth}
    \subsubsection{ax. 12}
    \begin{center}
        XII.\phantomsection\label{ax12}\\
        \hfill\\
        If two ſtraight lines (\hspace{-1ex}$\img{red_and_blue_lines}$\hspace{-1ex}) meet a third ſtraight line (\hspace{-1ex}$\blackline$\hspace{-1ex}) ſo as to make the two interior angles (\hspace{-1ex}$\img[0][0][20]{yellow_angle_14}$ and $\img[0][0][20]{red_angle_4}$\hspace{-1ex}) on the ſame ſide leſs than two right angles, theſe two ſtraight lines will meet if they be produced on that ſide on which the angles are leſs than two right angles.\\
        The fifth poſtulate may be expreſſed in any of the following ways:\\
        \begin{enumerate}
            \item Two diverging ſtraight lines cannot be both parallel to the ſame ſtraight line.
            \item If a ſtraight line interſect one of the two parallel ſtraight lines it muſt also interſect the other.
            \item Only one ſtraight line can be drawn through a given point, parallel to a given ſtraight line.
        \end{enumerate}
    \end{center}
\end{minipage}%
\begin{minipage}{0.33\textwidth}
    \begin{center}
        $\img[0][0][70]{axiom_12_figure}$
    \end{center}
\end{minipage}

\pagebreak

\begin{minipage}{0.20\textwidth}
    \phantom{}
\end{minipage}%
\begin{minipage}{0.80\textwidth}
    \begin{center}
        \subsection[Elucidations]{\centering \scshape{\LARGE{ELUCIDATIONS.}}}
        \label{subsec:elucidations}
    \end{center}

    \hfill

    Geometry has for its principal objects the expoſition and explanation of the properties of \textit{figure}, and figure is defined to be the relation which ſubſiſts between the boundaries of ſpace. Space or magnitude is of three kinds, \textit{linear, ſuperficial}, and \textit{ſolid}.\\

    \hfill

    \begin{wrapfigure}{r}{0.5\textwidth}
        \centering
        \includesvg[width=70pt]{vertex}
    \end{wrapfigure}
    Angles might properly be conſidered as a fourth ſpecies of magnitude. Angular magnitude evidently conſiſts of parts, and muſt therefore be admitted to be a ſpecies of quantity. The ſtudent muſt not ſuppoſe that the magnitude of an angle is affected by the length of the ſtraight lines which include it, and of whoſe mutual divergence it is the meaſure. The \textit{vertex} of an angle is the point where the \textit{ſides} or the \textit{legs} of the angle meet, as A.

    \hfill

    \begin{wrapfigure}{r}{0.5\textwidth}
        \centering
        \includesvg[width=125pt]{angles_diagram}
    \end{wrapfigure}
    An angle is often deſignated by a ſingle letter when its legs are the only lines which meet together at its vertex. Thus the red and blue lines form the yellow angle, which in other ſyſtems would be called the angle A. But when more than two lines meet in the ſame point, it was neceſſary by former methods, in order to avoid confuſion, to employ three letters to
\end{minipage}

\begin{minipage}{0.80\textwidth}
    deſignate an angle about that point, the letter which marked the vertex of the angle being always placed in the middle.  Thus the black and red lines meeting together at C, form the blue angle, and has been uſually denominated the angle FCD or DCF. The lines FC and CD are the legs of the angle; the point C is its vertex. In like manner the black angle would be deſignated the angle DCB or BCD. The red and blue angles added together, or the angle HCF added to FCD, make the angle HCD; and ſo of the other angles.  When the legs of an angle are produced or prolonged beyond its vertex, the angles made by them on both ſides of the vertex are ſaid to be vertically oppoſite to each other: Thus the red and yellow angles are ſaid to be \textit{vertically oppoſite} angles.\\

    \textit{Superpoſition} is the proceſs by which one magnitude may be conceived to be placed upon another, ſo as exactly to cover it, or ſo that every part of each ſhall exactly coincide.\\

    A line is ſaid to be \textit{produced}, when it is extended, prolonged, or has its length increaſed, and the increaſe of length which it receives is called its \textit{produced part}, or its \textit{production}.\\

    The entire length of the line or lines which encloſe a figure, is called its \textit{perimeter}. The firſt ſix books of Euclid treat of plane figures only. A line drawn from the centre of a circle to its circumference, is called a \textit{radius}. The lines which include a figure are called its \textit{ſides}. That ſide of a right angled triangle, which is oppoſite to the right angle, is called the \textit{hypotenuſe}. An \textit{oblong} is defined in the ſecond book, called a \textit{rectangle}. All the lines which are conſidered in the firſt ſix books of the Elements are ſuppoſed to be in the ſame plane.
\end{minipage}

\hfill

\begin{minipage}{0.20\textwidth}
    \phantom{}
\end{minipage}%
\begin{minipage}{0.80\textwidth}
    The \textit{ſtraight-edge} and \textit{compaſſes} are the only inſtruments, the uſe of which is permitted in Euclid, or plane Geometry. To declare this reſtriction is the object of the \textit{poſtulates}.\\

    The \textit{Axioms} of geometry are certain general propoſitions, the truth of which is taken to be ſelf-evident and incapable of being eſtablished by demonſtration.\\

    \textit{Propoſitions} are thoſe reſults which are obtained in geometry by a proceſs of reaſoning. There are two ſpecies of propoſitions in geometry, \textit{problems} and \textit{theorems}.\\

    A \textit{Problem} is a propoſition in which ſomething is propoſed to be done; as a line to be drawn under ſome given conditions, a circle to be deſcribed, ſome figure to be conſtructed, \&c.\\

    The \textit{ſolution} of the problem conſiſts in ſhowing how the thing required may be done by the aid of the rule or ſtraight-edge and compaſſes.\\

    The \textit{demonſtration} conſiſts in proving that the proceſs indicated in the ſolution really attains the required end.\\

    A \textit{Theorem} is a propoſition in which the truth of ſome principle is aſſerted. This principle muſt be deduced from the axioms and definitions, or other truths previously and independently eſtablished. To ſhow this is the object of demonſtration.\\

    A \textit{Problem} is analogous to a poſtulate.
\end{minipage}

\hfill

\begin{minipage}{0.80\textwidth}
    A \textit{Theorem} reſembles an axiom.\\

    A \textit{Poſtulate} is a problem, the ſolution of which is aſſumed.\\

    An \textit{Axiom} is a theorem, the truth of which is granted without demonſtration.\\

    A \textit{Corollary} is an inference deduced immediately from a propoſition.\\

    A \textit{Scholium} is a note or obſervation on a propoſition not containing an inference of ſufficient importance to entitle it to the name of a \textit{corollary}.\\

    A \textit{Lemma} is a propoſition merely introduced for the purpoſe of eſtablishing ſome more important propoſition.
\end{minipage}

\pagebreak

\pagestyle{fancy}
\fancyhf{}
\renewcommand{\headrulewidth}{0pt}
\fancyfoot[C]{\textsc{I.} \thepage}%

\subsection[Symbols and Abbreviations]{\centering \scshape{\LARGE{SYMBOLS AND ABBREVIATIONS.}}}
\label{subsec:symbolsandabbreviations}

\hfill

\begin{minipage}[t]{0.20\textwidth}
    \begin{center}
        $\btherefore$
    \end{center}
\end{minipage}%
\begin{minipage}[t]{0.80\textwidth}
    expreſſes the word \textit{therefore}.
\end{minipage}

\begin{minipage}[t]{0.20\textwidth}
    \begin{center}
        $\bequals$
    \end{center}
\end{minipage}%
\begin{minipage}[t]{0.80\textwidth}
    $\ldots$ \textit{equal}. This ſign of equality may be read \textit{equal to}, or \textit{is equal to}, or \textit{are equal to}; but any diſcrepancy in regard to the introduction of the auxiliary verbs \textit{is, are}, \&c. cannot affect the geometrical rigour.
\end{minipage}

\begin{minipage}[t]{0.20\textwidth}
    \begin{center}
        $\bnequals$
    \end{center}
\end{minipage}%
\begin{minipage}[t]{0.80\textwidth}
    means the ſame as if the words ‘not equal’ were written.
\end{minipage}

\begin{minipage}[t]{0.20\textwidth}
    \begin{center}
        $\bgt$
    \end{center}
\end{minipage}%
\begin{minipage}[t]{0.80\textwidth}
    ſignifies \textit{greater than}.
\end{minipage}

\begin{minipage}[t]{0.20\textwidth}
    \begin{center}
        $\blt$
    \end{center}
\end{minipage}%
\begin{minipage}[t]{0.80\textwidth}
    $\ldots$ \textit{leſs than}.
\end{minipage}

\begin{minipage}[t]{0.20\textwidth}
    \begin{center}
        $\bplus$
    \end{center}
\end{minipage}%
\begin{minipage}[t]{0.80\textwidth}
    is read \textit{plus} (\textit{more}), the ſign of addition; when interpoſed between two or more magnitudes, ſignifies their ſum.
\end{minipage}

\begin{minipage}[t]{0.20\textwidth}
    \begin{center}
        $\bminus$
    \end{center}
\end{minipage}%
\begin{minipage}[t]{0.80\textwidth}
    is read \textit{minus} (\textit{leſs}), ſignifies ſubtraction; and when placed between two quantities denotes that the latter is to be taken from the former.
\end{minipage}

\begin{minipage}[t]{0.20\textwidth}
    \begin{center}
        $\bcross$
    \end{center}
\end{minipage}%
\begin{minipage}[t]{0.80\textwidth}
    this ſign expreſſes the product of two or more numbers when placed between them in arithmetic and algebra; but in geometry it is generally uſed to expreſs a \textit{rectangle}, when placed between “two ſtraight lines which contain one of its right angles.” A \textit{rectangle} may alſo be repreſented by placing a point between two of its conterminous ſides.
\end{minipage}

% Really weird why I need this to push up the minipages to fill up
% the Symbols and Abbreviations subsection
\hfill

\hfill

\hfill

\hfill

\begin{minipage}[t]{0.20\textwidth}
    \begin{center}
        $\bcolon\ \bbcolon\ \bcolon$
    \end{center}
\end{minipage}%
\begin{minipage}[t]{0.80\textwidth}
    expreſſes an analogy or proportion; thus, if A, B, C and D, repreſent four magnitudes, and A has to B the ſame ratio that C has to D, the propoſition is thus briefly written,\\
    \begin{center}
        A $\bcolon$ B $\bbcolon$ C $\bcolon$ D\\
        A $\bcolon$ B $\bequals$ C $\bcolon$ D\\
        or $\dfrac{\text{A}}{\text{B}} \bequals \dfrac{\text{C}}{\text{D}}$.
    \end{center}
\end{minipage}

\begin{minipage}[t]{0.20\textwidth}
    \phantom{}
\end{minipage}
\begin{minipage}[t]{0.80\textwidth}
    This equality or ſameneſs of ratio is read,\\
    \begin{center}
        as A is to B, ſo is C to D;\\
        or A is to B, as C is to D.
    \end{center}
\end{minipage}

\begin{minipage}[t]{0.20\textwidth}
    \begin{center}
        $\bparallel$
    \end{center}
\end{minipage}
\begin{minipage}[t]{0.80\textwidth}
    ſignifies \textit{parallel to}.
\end{minipage}

\begin{minipage}[t]{0.20\textwidth}
    \begin{center}
        $\bperp$
    \end{center}
\end{minipage}
\begin{minipage}[t]{0.80\textwidth}
    $\ldots$ \textit{perpendicular to}.
\end{minipage}

\begin{minipage}[t]{0.20\textwidth}
    \begin{center}
        $\img{small_pie_slice}$
    \end{center}
\end{minipage}
\begin{minipage}[t]{0.80\textwidth}
    $\ldots$ \textit{angle}.
\end{minipage}

\begin{minipage}[t]{0.20\textwidth}
    \begin{center}
        $\img[0][0][20]{right_angle_2}$
    \end{center}
\end{minipage}
\begin{minipage}[t]{0.80\textwidth}
    $\ldots$ \textit{right angle}.
\end{minipage}

\begin{minipage}[t]{0.20\textwidth}
    \begin{center}
        $\tworightangles$
    \end{center}
\end{minipage}
\begin{minipage}[t]{0.80\textwidth}
    $\ldots$ \textit{two right angles}.
\end{minipage}

\begin{minipage}[t]{0.20\textwidth}
    \begin{center}
        \hspace{-2ex}$\img{three_lines_joined_at_a_point}$ or $\img[0][0][15]{two_lines_joined_at_a_point}$
    \end{center}
\end{minipage}
\begin{minipage}[t]{0.80\textwidth}
    briefly deſignates a \textit{point}.
\end{minipage}

\begin{minipage}[t]{0.20\textwidth}
    \begin{center}
        $\blackline^{\text{\large 2}}$
    \end{center}
\end{minipage}
\begin{minipage}[t]{0.80\textwidth}
    The ſquare deſcribed on a line is conciſely written thus.
\end{minipage}

\begin{minipage}[t]{0.20\textwidth}
    \begin{center}
        $\text{\large 2} \blackline^{\text{\large 2}}$
    \end{center}
\end{minipage}
\begin{minipage}[t]{0.80\textwidth}
    In the ſame manner twice the ſquare of, is expreſſed.
\end{minipage}

\begin{minipage}[t]{0.20\textwidth}
    \begin{center}
        def.
    \end{center}
\end{minipage}
\begin{minipage}[t]{0.80\textwidth}
    ſignifies \textit{definition}.
\end{minipage}

\begin{minipage}[t]{0.20\textwidth}
    \begin{center}
        poſt.
    \end{center}
\end{minipage}
\begin{minipage}[t]{0.80\textwidth}
    $\ldots$ \textit{poſtulate}.
\end{minipage}

\begin{minipage}[t]{0.20\textwidth}
    \begin{center}
        ax.
    \end{center}
\end{minipage}
\begin{minipage}[t]{0.80\textwidth}
    $\ldots$ \textit{axiom}.
\end{minipage}

\begin{minipage}[t]{0.20\textwidth}
    \begin{center}
        hyp.
    \end{center}
\end{minipage}
\begin{minipage}[t]{0.80\textwidth}
    $\ldots$ \textit{hypotheſis}. It may be neceſſary here to remark that the \textit{hypotheſis} is the condition aſſumed or taken for granted. Thus, the hypotheſis of the propoſition given in the Introduction, is that the triangle is iſoſceles, or that its legs are equal.
\end{minipage}

\begin{minipage}[t]{0.20\textwidth}
    \begin{center}
        conſt.
    \end{center}
\end{minipage}
\begin{minipage}[t]{0.80\textwidth}
    $\ldots$ \textit{conſtruction}. The \textit{conſtruction} is the change made in the original figure, by drawing lines, making angles, deſcribing circles, \&c. in order to adapt it to the argument of the demonſtration or the ſolution of the problem. The conditions under which theſe changes are made, are indisputable as thoſe contained in the hypotheſis. For inſtance, if we make an angle equal to a given angle, theſe two angles are equal by conſtruction.
\end{minipage}

\begin{minipage}[t]{0.20\textwidth}
    \begin{center}
        Q.E.F.
    \end{center}
\end{minipage}
\begin{minipage}[t]{0.80\textwidth}
    $\ldots$ \textit{Quod erat faciendum}.\\
    $\ldots$ Which was to be done.
\end{minipage}

\begin{minipage}[t]{0.20\textwidth}
    \begin{center}
        Q.E.D.
    \end{center}
\end{minipage}
\begin{minipage}[t]{0.80\textwidth}
    $\ldots$ \textit{Quod erat demonſtrandum}.\\
    $\ldots$ Which was to be demonſtrated.
\end{minipage}

\pagebreak

\pagestyle{fancy}
\fancyhf{}
\renewcommand{\headrulewidth}{0pt}
\fancyfoot[LE,RO]{\textsc{I.} \thepage}%

\subsection[Propositions]{\centering \scshape{\LARGE{PROPOSITIONS.}}}
\label{subsec:propositions}

\iconsectioninToC
% Propositions
\foreach \c in {1,...,48}{
        \input{book1/prop\c.tex}
        \newpage
    }
\stdsectioninToC
