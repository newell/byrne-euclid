% !TEX TS-program = xelatex
% !TEX options = -shell-escape -synctex=1 -interaction=nonstopmode -file-line-error "%DOC%"
\documentclass[twoside,11pt]{report}

\usepackage{standalone}
\usepackage{lipsum}
\usepackage{fancyhdr}
\usepackage{xpatch}
% \usepackage{tocloft}
\usepackage{titletoc}
\usepackage{setspace}
\usepackage{minibox}
\usepackage{enumitem}
\usepackage{fullpage}
\usepackage{mathtools,mathrsfs}
\usepackage{amssymb,amsthm}
\usepackage{graphicx,xcolor}
\usepackage[scale=2]{ccicons}
\usepackage[breakable]{tcolorbox}
\tcbuselibrary{breakable}
\usepackage{subfig}
\usepackage{float}
\usepackage{parskip}
\usepackage{lettrine}
\usepackage{fontspec}
\usepackage[compact]{titlesec}
\usepackage{calc}
\usepackage{xparse}
\usepackage{tikz}
\usepackage{svg}
\usepackage{xr-hyper}
\usepackage[colorlinks=true, citecolor=violet, linkcolor=black, urlcolor=black]{hyperref}

% Set this PATH to the root directory of your repository
\newcommand*{\MyPath}{/home/newell/code/byrne-euclid}%
% Don't show subsubsection titles -- these still show up in TOC (as desired)
\makeatletter
\titleformat{\subsubsection}[runin]{}{}{0pt}{\@gobble}
\makeatother

% \pagestyle{fancy}
% \fancyhf{}
% \renewcommand{\headrulewidth}{0pt}
% \fancyfoot[LO,RE]{\thepage}%

% TOC - image setup
\newcounter{propimage}
\makeatletter
\newcommand\stdsectioninToC{
    \titlecontents{subsubsection}[3.8em]
    {}%
    {\contentslabel{2.3em}}%
    {\hspace*{2.3em}}%
    {\titlerule*[1em]{.}\contentspage}
}
\newcommand\iconsectioninToC{
    \titlecontents{subsubsection}[3.8em]
    {\vskip 2ex}%
    {\hspace*{-2.3em}}%
    {
        \contentslabel{2.3em}%
        \stepcounter{propimage}%
        \smash{\includegraphics[width=50pt,height=25pt,keepaspectratio]{\MyPath/toc-images/image-\the\value{propimage}}}\hspace{0.5em}%
    }%
    {\titlerule*[1em]{.}\contentspage}%
}
\AtBeginDocument{\stdsectioninToC}
\makeatother

\setcounter{secnumdepth}{-3}
\setcounter{tocdepth}{3}

% Set the Font
\setmainfont{EBGaramond-Regular.otf}[
    BoldFont = EBGaramond-Bold.otf,
    ItalicFont = EBGaramond-Italic.otf,
    BoldItalicFont = EBGaramond-BoldItalic.otf,
    SmallCapsFont = EBGaramond12-AllSC.otf,
    Numbers = {OldStyle,Proportional},
    Ligatures = {Discretionary},
    RawFeature = {+ss06}
]
\newfontfamily\euclidinitials{EuclidInitialsnormal.ttf}

% Set the SVG paths (when I tried multiple lines to make this more readable it errored on me)
\svgpath{{\MyPath/symbols/}{\MyPath/byrne-euclid-svg/lines/}{\MyPath/byrne-euclid-svg/lines/points/}{\MyPath/byrne-euclid-svg/lines/arcs/}{\MyPath/byrne-euclid-svg/circles/}{\MyPath/byrne-euclid-svg/triangles/}{\MyPath/byrne-euclid-svg/figures/}{\MyPath/byrne-euclid-svg/figures/definitions/}{\MyPath/byrne-euclid-svg/figures/propositions/}{\MyPath/byrne-euclid-svg/angles/}{\MyPath/byrne-euclid-svg/symbols/}{\MyPath/byrne-euclid-svg/lines/}{\MyPath/byrne-euclid-svg/lines/points/}{\MyPath/byrne-euclid-svg/lines/arcs/}{\MyPath/byrne-euclid-svg/circles/}{\MyPath/byrne-euclid-svg/triangles/}{\MyPath/byrne-euclid-svg/figures/}{\MyPath/byrne-euclid-svg/figures/definitions/}{\MyPath/byrne-euclid-svg/figures/propositions/}{\MyPath/byrne-euclid-svg/angles/}{\MyPath/byrne-euclid-svg/icons/}}

% Create counter for lists.
\newcounter{listcounter}

% Colors
\definecolor{cred}{RGB}{212,42,32} % red
\definecolor{cyellow}{RGB}{250,194,43} % yellow
\definecolor{cblue}{RGB}{14,99,142} % blue
\definecolor{ctrans}{RGB}{252,243,217} % transparent
\definecolor{background}{HTML}{fcf3d9}

\newlength\mytemplena
\newlength\mytemplenb
\DeclareDocumentCommand\myalignalign{sm}
{
    \settowidth{\mytemplena}{$\displaystyle #2$}%
    \setlength\mytemplenb{\widthof{$\displaystyle=$}/2}%
    \hskip-\mytemplena%
    \hskip\IfBooleanTF#1{-\mytemplenb}{+\mytemplenb}%
}

% Images
\NewDocumentCommand{\img}{%
    O{0}
    O{0}
    O{25}
    O{0}
    m
}{%
    \begin{array}{c}
        \vspace{#1ex}\hspace{#2ex}\includesvg[width=#3pt]{#5}\hspace{#4ex}
    \end{array}
}

% Common Symbols
\newcommand*{\bplus}{\img[-0.8][0][10]{plus}}
\newcommand*{\bminus}{\img[0.8][0][10]{minus}}
\newcommand*{\bcross}{\img[-0.85][0][10]{cross}}
\newcommand*{\bequals}{\img[-0.1][0][10]{equals}}
\newcommand*{\bnequals}{\img[-0.85][0][10]{nequals}}
\newcommand*{\bparallel}{\img[-0.85][0][7]{parallel}}
\newcommand*{\bnparallel}{\img[-0.85][0][11]{not_parallel}}
\newcommand*{\bperiod}{\img[-0.5][-1][3]{period}}
\newcommand*{\bmark}{\img[-0.5][-1][3]{mark}}
\newcommand*{\bmultiply}{\img[1][0][3]{period}}
\newcommand*{\bsemicolon}{\img[-1.4][-1][3]{semicolon}}
\newcommand*{\bcolon}{\img[-0.5][0][3]{colon}}
\newcommand*{\bbcolon}{\bcolon\hspace{-1.5ex}\bcolon}
\newcommand*{\bcomma}{\img[-1.5][-1][3]{comma}}
\newcommand*{\btherefore}{\img[-0.5][0][10]{therefore}}
\newcommand*{\bgt}{\img[-0.85][0][10]{greater_than}}
\newcommand*{\blt}{\img[-0.85][0][10]{less_than}}
\newcommand*{\bperp}{\img[0][0][15]{perpendicular}}

% Common Lines
\newcommand*{\bimg}[1]{\img[0.8][0][45]{#1}}
\newcommand*{\redlines}{\bimg{red_lines}}
\newcommand*{\bluelines}{\bimg{blue_lines}}
\newcommand*{\blacklines}{\bimg{black_lines}}

\newcommand*{\redline}{\img[0.8][0][30]{red_line}}
\newcommand*{\redthinline}{\img[0.8][0][30]{red_thin_line}}
\newcommand*{\dottedredline}{\img[0.8][0][30]{red_dotted_line}}
\newcommand*{\blueline}{\img[0.8][0][30]{blue_line}}
\newcommand*{\bluethinline}{\img[0.8][0][30]{blue_thin_line}}
\newcommand*{\dottedblueline}{\img[0.8][0][30]{blue_dotted_line}}
\newcommand*{\blackline}{\img[0.8][0][30]{black_line}}
\newcommand*{\blackthinline}{\img[0.8][0][30]{black_thin_line}}
\newcommand*{\blackthindottedline}{\img[0.8][0][30]{black_thin_dotted_line}}
\newcommand*{\dottedblackline}{\img[0.8][0][30]{black_dotted_line}}
\newcommand*{\yellowline}{\img[0.8][0][30]{yellow_line}}
\newcommand*{\yellowthinline}{\img[0.8][0][30]{yellow_thin_line}}
\newcommand*{\dottedyellowline}{\img[0.8][0][30]{yellow_dotted_line}}

% Common Angles
\newcommand*{\tworightangles}{\img[0][0][30]{two_right_angles}}

% Common Shapes -- Book 5
\newcommand*{\reddome}{\img[-0.8][-0.5][10][-0.5]{red_dome_icon}}
\newcommand*{\bluedome}{\img[-0.8][-0.5][10][-0.5]{blue_dome_icon}}
\newcommand*{\blackdome}{\img[-0.8][-0.5][10][-0.5]{black_dome_icon}}
\newcommand*{\yellowdome}{\img[-0.8][-0.5][10][-0.5]{yellow_dome_icon}}

\newcommand*{\redcircle}{\img[-0.8][-0.5][10][-0.5]{red_circle_icon}}
\newcommand*{\bluecircle}{\img[-0.8][-0.5][10][-0.5]{blue_circle_icon}}
\newcommand*{\blackcircle}{\img[-0.8][-0.5][10][-0.5]{black_circle_icon}}
\newcommand*{\yellowcircle}{\img[-0.8][-0.5][10][-0.5]{yellow_circle_icon}}

\newcommand*{\reddrop}{\img[-0.8][-0.5][10][-0.5]{red_drop_icon}}
\newcommand*{\bluedrop}{\img[-0.8][-0.5][10][-0.5]{blue_drop_icon}}
\newcommand*{\blackdrop}{\img[-0.8][-0.5][10][-0.5]{black_drop_icon}}
\newcommand*{\yellowdrop}{\img[-0.8][-0.5][10][-0.5]{yellow_drop_icon}}

\newcommand*{\redhome}{\img[-0.8][-0.5][10][-0.5]{red_home_icon}}
\newcommand*{\bluehome}{\img[-0.8][-0.5][10][-0.5]{blue_home_icon}}
\newcommand*{\blackhome}{\img[-0.8][-0.5][10][-0.5]{black_home_icon}}
\newcommand*{\yellowhome}{\img[-0.8][-0.5][10][-0.5]{yellow_home_icon}}

\newcommand*{\redsquare}{\img[-0.8][-0.5][10][-0.5]{red_square_icon}}
\newcommand*{\smallredsquare}{\img[-0.8][-0.5][5][-0.5]{red_square_icon}}
\newcommand*{\bluesquare}{\img[-0.8][-0.5][10][-0.5]{blue_square_icon}}
\newcommand*{\bblacksquare}{\img[-0.8][-0.5][10][-0.5]{black_square_icon}}
\newcommand*{\yellowsquare}{\img[-0.8][-0.5][10][-0.5]{yellow_square_icon}}
\newcommand*{\transsquare}{\img[-0.8][-0.5][10][-0.5]{trans_square_icon}}

\newcommand*{\reddiamond}{\img[-0.8][-0.5][8][-0.5]{red_diamond_icon}}
\newcommand*{\bluediamond}{\img[-0.8][-0.5][8][-0.5]{blue_diamond_icon}}
\newcommand*{\blackdiamond}{\img[-0.8][-0.5][8][-0.5]{black_diamond_icon}}
\newcommand*{\yellowdiamond}{\img[-0.8][-0.5][8][-0.5]{yellow_diamond_icon}}
\newcommand*{\transdiamond}{\img[-0.8][-0.5][8][-0.5]{trans_diamond_icon}}

\newcommand*{\redtriangle}{\img[-0.8][-0.5][10][-0.5]{red_triangle_icon}}
\newcommand*{\bluetriangle}{\img[-0.8][-0.5][10][-0.5]{blue_triangle_icon}}
\newcommand*{\bblacktriangle}{\img[-0.8][-0.5][10][-0.5]{black_triangle_icon}}

\newcommand*{\blackrectangle}{\img[-0.8][-0.5][10][-0.5]{black_rectangle_icon}}
\newcommand*{\yellowrectangle}{\img[-0.8][-0.5][10][-0.5]{yellow_rectangle_icon}}






\setlength{\parskip}{1em}

\allowdisplaybreaks

\begin{document}
% Cover
\onehalfspacing
% \definecolor{light-gray}{gray}{0.7}
% \pagecolor{light-gray}
% \thispagestyle{empty}
% \centerline{\Huge{BYRNE'S EUCLID}}
% \vspace{4ex}
% \centerline{\Large{The First Six Books Of The Elements Of Euclid}}

% \hfill

% \begin{center}
%   \begin{minipage}[b]{\textwidth}
%     \begin{center}
%       \includesvg[width=0.6\textwidth]{book1_proposition_47_figure_cover}
%     \end{center}
%   \end{minipage}
% \end{center}

% \vspace*{\fill}

% \begin{center}
%   \LARGE{Newell Jensen}\\
%   \LARGE{2023}
% \end{center}
% \newpage

% First Page
\pagecolor{background}
\thispagestyle{empty}

\hfill

\hfill

\hfill

\begin{flushright}
  \LARGE{Byrne's}\\
  \Huge{\color{cred}{E}\color{cblue}{u}\color{cyellow}{c}\color{cred}{l}\color{cblue}{i}\color{cyellow}{d}}
\end{flushright}

\vspace*{\fill}
\newpage

% Copyright and License
\thispagestyle{empty}

Newell Jensen\\
\url{https://newell.github.io}
% \the\textwidth
\vspace*{\fill}

\large{\textcopyright\ 2023 by Newell Jensen}

\hfill

\hfill

\centerline{\ccbyncsa}

\begin{center}
  \large{This work is licensed under the Creative Commons Attribution-NonCommercial-ShareAlike 4.0 International License (\href{https://creativecommons.org/licenses/by-nc-sa/4.0/}{CC BY-NC-SA 4.0}).}
\end{center}
\vspace*{\fill}
\newpage

% Dedication
\thispagestyle{empty}

\vspace*{\fill}
\begin{quote}
  \centering
  \Large{Dedicated to the \textsc{COL{\color{cred}{O}}{\color{cblue}{U}}{\color{cyellow}{R}}FUL} memory of my father, \LARGE{\textsc{William Henry Jensen}}.}\\
\end{quote}
\vspace*{\fill}
\newpage

% Preface
\pagenumbering{gobble}
\centerline{\LARGE{Preface}}

\hfill

\normalsize % Change back the size of the font


\hspace{2em}\textit{Let no one destitute of geometry enter my doors.}

\hspace{16em} --- Plato

\hfill

\begin{center}
  \textit{Sire, there is no royal road to geometry.}

  \hspace{10.5em}--- Euclid
\end{center}

\hfill

\begin{flushright}
  \textit{The mathematician's patterns, like the painter's or the poet's must be beautiful;}
  \textit{the ideas like the colours or the words, must fit together in a harmonious way.}
  \textit{Beauty is the first test: there is no permanent place in the world}\\
  \textit{for ugly mathematics.}
\end{flushright}

\hfill --- G. H. Hardy\\

Legend has it that the words, \textit{"Let no one destitute of geometry enter my doors"} were inscribed in the doorway entering Plato's Academy in Athens.  Euclid, an ancient Greek mathematician of Alexandria and a contemporary of Plato, considered the '\textit{Father of geometry}', is chiefly known for his treatise of 13 books called the \textit{Elements}.  A mathematical and logical masterpiece, the \textit{Elements} is a collection of definitions, postulates, propositions (consisting of theorems, problems and constructions), as well as the logical proofs of these propositions.  The \textit{Elements} has been referred to as the most successful and influential textbook ever written.

According to the ancient Greek historian Proclus, when the King of Egypt asked Euclid if there was an easier way to learn geometry, Euclid famously replied: \textit{"There is no royal road to geometry."} A well-known quote, often used to emphasize the importance of hard work, discipline, and persistence in the pursuit of knowledge and understanding, it suggests that there are no shortcuts to true understanding, and that the only way to master a subject is through diligent study and practice.

Enter Oliver Byrne.  An Irish-born civil engineer and surveyor, Byrne (1810-1880) is best known for his
illustrated edition of the first six books of Euclid’s \textit{Elements}, published in 1847 under the title
\textit{Byrne’s Euclid}. Byrne’s edition is noted for its distinctive use of colour-coded diagrams and symbols, intended to make the complex concepts of Euclidean geometry more accessible and understandable to a wider audience.  Each of the book’s geometric figures is rendered in bold primary colors, with different colors being used to distinguish between various parts of the figure and to indicate different types of lines and angles.  Byrne's edition of the \textit{Elements} was a commercial success and went through several editions in the years following its initial publication. While the book's approach to illustrating geometry was unconventional for its time, it has since become a popular and influential work in the field of graphic design, as well as a fascinating example of the intersection between mathematics, art, and visual communication.

Overall, \textit{Byrne's Euclid} represents a unique and innovative approach to the study of geometry -- one that combines technical precision with a bold and imaginative visual style. Inspiring generations of students and scholars to approach the study of mathematics with a sense of creativity and wonder, it
remains an important and enduring work in the history of mathematical literature.  The book's use of colorful diagrams and illustrations, along with concise and straightforward explanations, can make it easier for students to understand abstract concepts and develop a deeper appreciation for the beauty and elegance of mathematics. Additionally, studying \textit{Byrne's Euclid} can help students develop problem-solving skills and logical reasoning, which are valuable not just in mathematics but in many areas of life.

My first experience with \textit{Byrne’s Euclid} occurred while I was working through Euclid’s \textit{Elements} and searching on the Internet for more information about a particular proposition, when I came across Nicholas Rougeux's exquisite reproduction of Oliver Byrne's celebrated work --- \url{https://c82.net/euclid}.  Awed by the stunning beauty of the \textit{Elements} and its logical precision, as well as Byrne's masterful and imaginative approach, I was filled with inspiration to create this book.  Both G. H. Hardy's quote and \textit{Byrne's Euclid}, underscore the creative and aesthetic dimensions of mathematics. Hardy's quote highlights how mathematicians, like painters or poets, create enduring patterns with ideas, while \textit{Byrne's Euclid}, visually showcases the beauty of mathematical concepts through intricate illustrations. Together, they remind us that mathematics is not solely a logical pursuit, but also a richly imaginative and expressive one.  It is my hope that this rendition of \textit{Byrne's Euclid} continues in this spirit.

While Byrne's original work featured the elegant \textit{Caslon} typeface, I have chosen to use the open source \href{https://github.com/georgd/EB-Garamond}{\textit{EB Garmond}} typeface for my edition. While the two typefaces share many similarities in ligatures and glyphs, those who aren't typography experts may not even notice the difference.  For example, both \textit{Caslon} and \textit{EB Garamond} are serif typefaces, which means they have small decorative lines at the ends of each letter stroke. Classic and elegant, both have had a long history in printing and publishing.

In terms of their specific design features, these typefaces have similar letter shapes and proportions. For example, they both have a lowercase "a" with a curved tail, and a lowercase "g" with a descending loop. They also both have a tall and narrow uppercase "H", and a diagonal crossbar on the uppercase "A".  Finally, both typefaces feature ligatures (two or more letters that are joined together into a single glyph), such as "fi" and "fl", which have a similar design and placement in the two typefaces.

First time readers of \textit{Byrne's Euclid} should be made aware that the long s ({\color{cred}{ſ}}) is used throughout the book. The long s ({\color{cred}{ſ}}) is a letterform of the Latin alphabet that was commonly used in Europe from the Middle Ages until the 19th century. It looks like a lowercase "s", but with a longer, more elongated shape, resembling an "f" without its crossbar. In printed materials from the time, the long s was used in place of a normal "s" at the beginning or in the middle of a word, but not at the end of a word or after certain letters like "m", "n", or "u."

A stylistic convention, the use of the long s was thought to make text easier to read and more aesthetically pleasing, as it allowed letters to be more closely spaced and made words look more uniform in appearance. However, as printing technology evolved and more uniform letter spacing became possible, the long s fell out of use and was gradually replaced by the modern short "s" in the 19th century. Despite its decline in usage, the long s can still be found in some historic texts and remains a fascinating example of the evolution of written language over time.  Here is an illustration of the long s,
\begin{center}
  {\color{cred}{ſ}}orts $=$ sorts $\neq$ forts\\
  \textit{ca{\color{cred}{ſ}}e} $=$ \textit{case} $\neq$ \textit{cafe}
\end{center}

I would like to thank Nicholas Rougeux for giving me permission to use the scalar vector graphic images from his website in producing this book.  I would also like to thank all the logicians and mathematicians, past and present, whose shoulders we all stand on.  And to you, the reader.  It's my sincere hope and desire that you enjoy this book and the logical truths found herein.  I encourage you to study and practice the propositions so that you may walk your {\color{cred}{o}}{\color{cblue}{w}}{\color{cyellow}{n}} road to geometry.

\hfill

\begin{flushright}
  Newell Jensen, 2023
\end{flushright}

\newpage

\vspace*{\fill}

\begin{center}
  \textit{Reason is immortal, all else is mortal.}

  \hspace{8em}--- Pythagoras
\end{center}

\vspace*{\fill}

\newpage

% TOC
\tableofcontents

\newpage

% Introduction
\pagenumbering{roman}
\begin{minipage}{0.80\textwidth}
    \section[Introduction]{\centering {INTRODUCTION.}}
    \label{sec:intro}

    \hfill

    \lettrine[lines=3, loversize=1, nindent=0pt]{\euclidinitials{T}}{}HE arts and ſciences have become ſo extenſive, that to facilitate their acquirement is of as much importance as to extend their boundaries. Illuſtration, if it does not ſhorten the time of ſtudy, will at leaſt make it more agreeable. T\textsc{HIS WORK} has a greater aim than mere illuſtration; we do not introduce colours for the purpoſe of entertainment, or to amuſe \textit{by certain combinations of tint and form}, but to aſſiſt the mind in its reſearches after truth, to increaſe the facilities of inſlruction, and to diffuſe permanent knowledge. If we wanted authorities to prove the importance and uſefulneſs of geometry, we might quote every philoſopher ſince the days of Plato. Among the Greeks, in ancient, as in the ſchool of Peſtalozzi and others in recent times, geometry was adopted as the beſt gymnaſtic of the mind. In fact, Euclid’s Elements have become, by common conſent, the baſis of mathematical ſcience all over the civilized globe. But this will not appear extraordinary, if we conſider that this ſublime ſcience is not only better calculated than any other to call forth the ſpirit of inquiry, to elevate the mind, and to ſtrengthen the reaſoning faculties, but alſo it forms the beſt introduction to moſt of the uſeful and important vocations of human life.  Arithmetic, land-ſurveying, menſuration, engineering, navigation, mechanics, hydroſtatics, pneumatics, optics, phyſical aſtronomy, \&c. are all dependent on the propoſitions of geometry.

    \hfill

    Much however depends on the firſt communication of any ſcience to a learner, though the beſt and moſt eaſy methods are ſeldom adopted.  Propoſitions are placed before a ſtudent,
\end{minipage}

\newpage

\begin{minipage}{0.20\textwidth}
    \phantom{}
\end{minipage}%
\begin{minipage}{0.80\textwidth}
    who though having a ſufficient underſtanding, is told juſt as much about them on entering at the very threſhold of the ſcience, as gives him a prepoſſeſſion moſt unfavourable to his future ſtudy of this delightful ſubject; or “the formalities and paraphernalia of rigour are ſo oſtentatiouſly put forward, as almoſt to hide the reality. Endleſs and perplexing repetitions, which do not confer greater exactitude on the reaſoning, render the demonſtrations involved and obſcure, and conceal from the view of the ſtudent the conſecution of evidence.”\\

    Thus an averſion is created in the mind of the pupil, and a ſubject ſo calculated to improve the reaſoning powers, and give the habit of cloſe thinking, is degraded by a dry and rigid courſe of inſtruction into an unintereſting exerciſe of the memory. To raiſe the curioſity, and to awaken the liſtleſs and dormant powers of younger minds ſhould be the aim of every teacher; but where examples of excellence are wanting, the attempts to attain it are but few, while eminence excites attention and produces imitation. The object of this Work is to introduce a method of teaching geometry, which has been much approved of by many ſcientific men in this country, as well as in France and America. The plan here adopted forcibly appeals to the eye, the moſt ſenſitive and the moſt comprehenſive of our external organs, and its pre-eminence to imprint it ſubject on the mind is ſupported by the incontrovertible maxim expreſſed in the well known words of Horace:—\\

    \begin{quotation}
        \noindent \small{\textit{Segnius irritant animos demiſſa per aurem\\
                Quàm quæ ſunt oculis ſubjecta fidelibus}}.\\

        \noindent \small{A feebler impreſs through the ear is made,\\
            Than what is by the faithful eye conveyed}.
    \end{quotation}
\end{minipage}

\newpage

\begin{minipage}{0.80\textwidth}
    All language conſiſts of repreſentative ſigns, and thoſe ſigns are the beſt which effect their purpoſes with the greateſt preciſion and diſpatch. Such for all common purpoſes are the audible ſigns called words, which are ſtill conſidered as audible, whether addreſſed immediately to the ear, or through the medium of letters to the eye. Geometrical diagrams are not ſigns, but the materials of geometrical ſcience, the object of which is to ſhow the relative quantities of their parts by a proceſs of reaſoning called Demonſtration.

    This reaſoning has been generally carried on by words, letters, and black or uncoloured diagrams but as the uſe of coloured ſymbols, ſigns, and diagrams in the linear arts and ſciences, renders the proceſs of reaſoning more preciſe, and the attainment more expeditious, they have been in this inſtance accordingly adopted.\\

    Such is the expedition of this enticing mode of communicating knowledge, that the Elements of Euclid can be acquired in leſs than one third the time uſually employed, and the retention by the memory is much more permanent; theſe facts have been aſcertained by numerous experiments made by the inventor, and ſeveral others who have adopted his plans. The particulars of which are few and obvious; the letters annexed to points, lines, or other parts of a diagram are in fact but arbitrary names, and repreſent them in the demonſtration; inſtead of theſe, the parts being differently coloured, are made to name themſelves, for their forms in correſponding colours represent them in the demonſtration.\\

    In order to give a better idea of this ſyſtem, and of the advantages gained by its adoption, let us take a right angled triangle,
\end{minipage}

\newpage

\begin{minipage}{0.20\textwidth}
    \phantom{}
\end{minipage}%
\begin{minipage}{0.80\textwidth}
    \begin{center}
        $\img[0][3][175]{triangle_abc}$
    \end{center}
    \hfill\\
    and expreſs ſome of its properties both by colours and the
    method generally employed.\\
    \begin{center}
        \textit{Some of the properties of the right angled triangle ABC, expreſſed by the method generally employed}.\\
        \hfill\\
        \begin{enumerate}
            \item $\img{red_angle_60} \bplus \img{yellow_angle_65} \bplus \img{blue_angle_64} \bequals {\text{\large 2}} \img{yellow_angle_65}$\\ $\bequals \tworightangles \bperiod$\\ That is, the red angle added to the yellow angle added to the blue angle, equal twice the yellow angle, equal two right angles.
            \item $\img{red_angle_60} \bplus \img{blue_angle_64} \bequals \img{yellow_angle_65} \bperiod$\\ Or in words, the red angle added to the blue angle, equal the yellow angle.
            \item $\img{yellow_angle_65} \bgt \img{red_angle_60}$ or $\img{blue_angle_64} \bperiod$\\ The yellow angle is greater than either the red or blue angle.
            \item $\img{red_angle_60}$ or $\img{blue_angle_64} \blt \img{yellow_angle_65} \bperiod$\\ Either the red or blue angle is leſs than the yellow angle.
            \item $\img{yellow_angle_65}$ minus $\img{blue_angle_64} \bequals \img{red_angle_60} \bperiod$\\ In other terms, the yellow angle made leſs by the blue angle equal the red angle.
            \item $\yellowline^{\text{\large 2}} \bequals \blueline^{\text{\large 2}} \bplus \redline^{\text{\large 2}} \bperiod$\\ That is, the ſquare of the yellow line is equal to the ſum of the ſquares of the blue and red lines.
        \end{enumerate}
    \end{center}
\end{minipage}

\newpage

\begin{minipage}{0.80\textwidth}
    In oral demonſtrations we gain with colours this important advantage, the eye and the ear can be addreſſed at the ſame moment, ſo that for teaching geometry, and other linear arts and ſciences, in claſſes, the ſyſtem is the beſt ever propoſed, this is apparent from the examples juſt given.\\

    Whence it is evident that a reference from the text to the diagram is more rapid and ſure, by giving the forms and colours of the parts, or by naming the parts and their colours, than naming the parts and letters on the diagram. Beſides the ſuperior ſimplicity, this ſyſtem is likewiſe conſpicuous for concentration, and wholly excludes the injurious through prevalent practice of allowing the ſtudent to commit the demonſtration to memory; until reaſon, and fact, and proof only make impreſſions on the underſtanding.\\

    Again, when lecturing on the principles or properties of figures, if we mention the colour of the part or parts referred to, as in ſaying, the red angle, the blue line, or lines, \&c. the part or parts thus named will be immediately ſeen by all in the claſs at the ſame inſtant; not ſo if we ſay the angle ABC, the triangle PFQ, the figure EGKt, and ſo on; for the letters muſt be traced one by one before the ſtudents arrange in their minds the particular magnitude referred to, which often occaſions confuſion and error, as well as loſs of time.\\

    Alſo if the parts which are given as equal, have the ſame colours in any diagram, the mind will not wander from the object before it; that is, ſuch an arrangement preſents an ocular demonſtration of the parts to be proved equal, and the learner retains the data throughout the whole of the reaſoning.
\end{minipage}

\newpage

\begin{minipage}{0.20\textwidth}
    \phantom{}
\end{minipage}%
\begin{minipage}{0.80\textwidth}
    But whatever may be the advantages of the preſent plan, if it be not ſubſtituded for, it can always be made a powerful auxiliary to the other methods, for the purpoſe of introduction, or of a more ſpeedy reminiſcence, or of more permanent retention by the memory.\\

    The experience of all who have formed ſyſtems to impreſs facts on the underſtanding, agree in proving that coloured repreſentations, as pictures, cuts, diagrams, \&c. are more eaſily fixed in the mind than mere ſentences unmarked by any peculiarity. Curious as it may appear, poets ſeem to be aware of this fact more than mathematicians; many modern poets allude to this viſible ſyſtem of communicating knowledge, one of them has thus expreſſed himſelf:\\
    \begin{quotation}
        \small{\noindent Sounds which addreſs the ear are loſt and die\\
            In one ſhort hour, but theſe which ſtrike the eye,\\
            Live long upon the mind, the faithful ſight\\
            Engraves the knowledge with a beam of light.}
    \end{quotation}

    \hfill

    This perhaps may be reckoned the only improvement which plane geometry has received ſince the days of Euclid, and if there were any geometers of note before that time, Euclid’s ſucceſs has quite eclipſed their memory, and even occaſioned all good things of that kind to be aſſigned to him; like Æſop among the writers of Fables. It may alſo be worthy of remark, as tangible diagrams afford the only medium through which geometry and other linear arts and ſciences can be taught to the blind, this viſible ſyſtem is no leſs adapted to the exigencies of the deaf and dumb.\\

    Care muſt be taken to ſhow that colour has nothing to do with the lines, angles, or magnitudes, except merely to name them.
\end{minipage}

\newpage

\begin{minipage}{0.80\textwidth}
    A mathematical line, which is length without breadth, cannot poſſeſs colour, yet the junction of the two colours on the ſame plane gives a good idea of what is meant by a mathematical line; recollect we are ſpeaking familiarly, ſuch a junction is to be underſtood and not the colour, when we ſay the black line, the red line or lines, \&c.  Colours and coloured diagrams may at firſt appear a clumſy method to convey proper notations of the properties and parts of mathematical figures and magnitudes, however they will be found to afford a means more refined and extenſive than any that has been hitherto propoſed.\\

    We ſhall here define a point, a line, and a surface, and demonſtrate a propoſition in order to ſhow the truth of this aſſertion.  A point is that which has poſition, but not magnitude; or a point is poſition only, abſtracted from the conſideration of length, breadth, and thickneſs. Perhaps the following deſcription is better calculated to explain the nature of a mathematical point to thoſe who have not acquired the idea, than the above ſpecious definition.\\

    Let three colours meet and cover a portion of the paper, where they meet is not blue, nor is it yellow, nor is it red,
    \begin{wrapfigure}{l}{0.5\textwidth}
        \centering
        \includesvg[width=100pt]{circle_divided_into_three_equal_slices}
    \end{wrapfigure}
    as it occupies no portion of the plane, for if it did, it would belong to the blue, the red, or the yellow part; yet it exiſts, and has poſition without magnitude, ſo that with a little reflection, this junction of three colours on a plane gives a good idea of a mathematical point.
\end{minipage}

\newpage

\begin{minipage}{0.80\textwidth}
    A line is length without breadth. With the aſſiſtance of colours, nearly in the ſame manner as before, an idea of a line may be thus given:—\\

    Let two colours meet and cover a portion of the paper; where they meet is not red, nor is it blue;
    \begin{wrapfigure}{l}{0.5\textwidth}
        \centering
        \includesvg[width=100pt]{two_colors_stacked}
    \end{wrapfigure}
    therefore the junction occupies no portion of the plane, and therefore it cannot have breadth but only length: from which we can readily form an idea of what is meant by a mathematical line.  For the purpoſe of illuſtration, one colour differing from the colour of the paper, or plane upon which it is drawn, would have been ſufficient; hence in future, if we ſay the red line, the blue line, or lines, \&c. it is the junctions with the plane upon which they are drawn are to be underſtood.\\

    Surface is that which has length and breadth without thickneſs.
    \begin{wrapfigure}{l}{0.5\textwidth}
        \centering
        \includesvg[width=80pt]{figure_pq}
    \end{wrapfigure}
    When we conſider a ſolid body (PQ), we perceive at once that it has three dimensions, namely:—length, breadth, and thickneſs;  ſuppoſe one part of this ſolid (PS) to be red, and the other part (QR) yellow, and that the colours be diſtinct without commingling, the blue surface (RS) which ſeparates theſe parts, or which is the ſame thing, that which divides the ſolid without loſs of material, muſt be without thickneſs, and only poſſeſſes length and breadth;
\end{minipage}%

\newpage

\begin{minipage}{0.20\textwidth}
    \phantom{}
\end{minipage}%
\begin{minipage}{0.80\textwidth}
    this plainly appears from reaſoning, ſimilar to that juſt employed in defining, or rather deſcribing a point and a line.\\

    The propoſition which we have ſelected to elucidate the manner in which the principles are applied is the fifth of the firſt Book.\\

    \begin{wrapfigure}{r}{0.5\textwidth}
        \centering
        \includesvg[width=150pt]{proposition_5_figure}
    \end{wrapfigure}
    In an iſoſceles triangle ABC, the internal angles at the baſe ABC, ACB are equal, and when the ſides AB, AC are produced, the external angles at the baſe BCE, CBD are alſo equal.\\

    \raggedright Produce $\redline$ and $\redline \bperiod$  Make $\yellowline \bequals \yellowline \bperiod$

    \hfill

    \hfill

    \begin{center}
        Draw $\blueline \bequals \blueline$ [\hyperref[book1pr3]{\textsc{I.} 3}]\\
        in $\img{left_triangle}$ and $\img{right_triangle}$\\
        we have $\bimg{red_and_yellow_lines} \bequals \bimg{red_and_yellow_lines}$\\
        $\redline \bequals \redline$ and $\img[0][0][20]{black_angle}$ common:\\
        $\btherefore \blueline \bequals \blueline \bcomma \img[0][0][20]{left_blue_and_yellow_angles} \bequals \img[0][0][20]{right_blue_and_yellow_angles}$\\
        and $\img{left_red_angle} \bequals \img{right_red_angle}$ [\hyperref[book1pr4]{\textsc{I.} 4}].\\
        Again in $\img[0][0][40]{lower_left_triangle}$ and $\img[0][0][40]{lower_right_triangle} \bcomma$\\
    \end{center}
\end{minipage}

\newpage

\begin{minipage}{0.80\textwidth}
    \begin{center}
        $\yellowline \bequals \yellowline \bcomma$\\
        $\blueline \bequals \blueline \bcomma$\\
        and $\img{left_red_angle} \bequals \img{right_red_angle} \bsemicolon$\\
        $\btherefore \img[-0.5]{left_yellow_angle_plus_remaining_angle} \bequals \img[-0.5]{right_yellow_angle_plus_remaining_angle}$\\
        and $\img[-0.5]{left_yellow_angle} \bequals \img[-0.5]{right_yellow_angle}$ [\hyperref[book1pr4]{\textsc{I.} 4}].\\
        But $\img[0][0][20]{left_blue_and_yellow_angles} \bequals \img[0][0][20]{right_blue_and_yellow_angles} \bcomma$\\
        $\btherefore \img{left_blue_angle} \bequals \img{right_blue_angle} \bperiod$\\
        \hfill\\
        \hfill Q.E.D.\\
        \hfill\\
        \textit{By annexing Letters to the Diagram}.
    \end{center}

    \hfill

    L\textsc{ET} the equal ſides AB and AC be produced through the extremities BC, of the third ſide, and in the produced part BD of either, let any point D be aſſumed, and from the other let AE be cut off equal to AD [\hyperref[book1pr3]{\textsc{I.} 3}]. Let the points E and D, ſo taken in the produced ſides, be connected by ſtraight lines DC and BE with the alternate extremities of the third ſide of the triangle.\\

    In the triangles DAC and EAB the ſides DA and AC are reſpectively equal to EA and AB, and the included angle A is common to both triangles. Hence [\hyperref[book1pr4]{\textsc{I.} 4}] the line DC is equal to BE, the angle ADC to the angle AEB, and the angle ACD to the angle ABE; if from the equal lines AD and AE the equal ſides AB and AC be taken, the remainders BD and CE will be equal. Hence in the triangles BDC and CEB, the ſides BD and DC are reſpectively equal to CE and EB, and the angles D and E included by thoſe ſides are alſo equal. Hence [\hyperref[book1pr4]{\textsc{I.} 4}] the angles DBC and ECB, which are thoſe included by the third ſide BC and the
\end{minipage}

\newpage

\begin{minipage}{0.20\textwidth}
    \phantom{}
\end{minipage}%
\begin{minipage}{0.80\textwidth}
    productions of the equal ſides AB and AC are equal.  Alſo the angles DCB and EBC are equal if thoſe equals be taken from the angles DCA and EBA before proved equal, the remainders, which are the angles ABC and ACB oppoſite to the equal ſides, will be equal.\\

    \textit{Therefore in an iſoſceles triangle, \&c}.\\

    \hfill Q.E.D.\\

    Our object in this place being to introduce the ſyſtem rather than to teach any particular ſet of propoſitions, we have therefore ſelected the foregoing out of the regular courſe. For ſchools and other public places of inſtruction, dyed chalks will anſwer to deſcribe diagrams, \&c. for private uſe coloured pencils will be found very convenient.\\

    We are happy to find that the Elements of Mathematics now forms a conſiderable part of every ſound female education, therefore we call the attention of thoſe intereſted or engaged in the education of ladies to this very attractive mode of communicating knowledge, and to the ſucceeding work for its future development.\\

    We ſhall for the preſent conclude by obſerving, as the ſenſes of ſight and hearing can be ſo forcibly and inſtantaneously addreſſed alike with one thouſand as with one, \textit{the million} might be taught geometry and other branches of mathematics with great eaſe, this would advance the purpoſe of education more than any thing that \textit{might} be named, for it would teach the people how to think, and not what to think; it is in this particular the great error of education originates.
\end{minipage}

\pagenumbering{arabic}

% Book sections
\pagestyle{fancy}
\fancyhf{}
\renewcommand{\headrulewidth}{0pt}
\fancyfoot[LE,RO]{\textsc{I.} \thepage}%

\begin{minipage}{0.165\textwidth}
    \phantom{}
\end{minipage}
\begin{minipage}{0.67\textwidth}
    \null\removelastskip\nointerlineskip\vspace*{-\baselineskip}
    \section[Book I]{\centering THE ELEMENTS OF EUCLID\\ BOOK I.}
    \label{sec:book1}

    \hfill

    \subsection[Definitions]{\centering \scshape{\LARGE{DEFINITIONS.}}}
    \label{subsec:definitions}

    \hfill

    \subsubsection{def. 1}
    \begin{center}
        I.\phantomsection\label{book1def1}\\
        \hfill\\
        A \textit{point} is that which has no part.\\
    \end{center}
    \subsubsection{def. 2}
    \begin{center}
        II.\phantomsection\label{book1def2}\\
        \hfill\\
        A \textit{line} is length without breadth.\\
    \end{center}
    \subsubsection{def. 3}
    \begin{center}
        III.\phantomsection\label{book1def3}\\
        \hfill\\
        The extremities of a line are points.\\
    \end{center}
    \subsubsection{def. 4}
    \begin{center}
        IV.\phantomsection\label{book1def4}\\
        \hfill\\
        \raggedright A ſtraight or right line is that which lies evenly between its extremities.
    \end{center}
    \subsubsection{def. 5}
    \begin{center}
        V.\phantomsection\label{book1def5}\\
        \hfill\\\
        A ſurface is that which has length and breadth only.\\
    \end{center}
    \subsubsection{def. 6}
    \begin{center}
        VI.\phantomsection\label{book1def6}\\
        \hfill\\
        The extremities of a ſurface are lines.\\
    \end{center}
\end{minipage}%
\begin{minipage}{0.165\textwidth}
    \phantom{}
\end{minipage}

\hfill

\begin{minipage}{0.165\textwidth}
    \phantom{}
\end{minipage}
\begin{minipage}{0.67\textwidth}
    \subsubsection{def. 7}
    \begin{center}
        VII.\phantomsection\label{book1def7}\\
        \hfill\\
        \raggedright A plane ſurface is that which lies evenly between its extremities.\\
    \end{center}
    \subsubsection{def. 8}
    \begin{center}
        VIII.\phantomsection\label{book1def8}\\
        \hfill\\
        \raggedright A plane angle is the inclination of two lines to one another, in a plane, which meet together, but are not in the ſame direction.
    \end{center}
\end{minipage}%
\begin{minipage}{0.165\textwidth}
    \phantom{}
\end{minipage}

\hfill

\begin{center}
    IX.\phantomsection\label{book1def9}\\
\end{center}
\begin{minipage}{0.67\textwidth}
    \subsubsection{def. 9}
    \begin{center}
        \raggedright A plane rectilinear angle is the inclination of two ſtraight lines to one another, which meet together, but are not in the ſame ſtraight line.
    \end{center}
\end{minipage}%
\begin{minipage}{0.33\textwidth}
    \begin{center}
        $\img[0][0][60]{book1_definition_9_figure}$
    \end{center}
\end{minipage}

\hfill

\begin{center}
    X.\phantomsection\label{book1def10}\\
\end{center}
\begin{minipage}{0.67\textwidth}
    \subsubsection{def. 10}
    \begin{center}
        \raggedright When one ſtraight line ſtanding on another ſtraight line makes the adjacent angles equal, each of theſe angles is called a \textit{right angle}, and each of theſe lines is ſaid to be \textit{perpendicular} to the other.
    \end{center}
\end{minipage}%
\begin{minipage}{0.33\textwidth}
    \begin{center}
        $\img[0][0][60]{book1_definition_10_figure}$
    \end{center}
\end{minipage}

\hfill

\begin{center}
    XI.\phantomsection\label{book1def11}\\
\end{center}
\begin{minipage}{0.67\textwidth}
    \subsubsection{def. 11}
    \begin{center}
        An obtuſe angle is an angle greater than a right angle.
    \end{center}
\end{minipage}%
\begin{minipage}{0.33\textwidth}
    \begin{center}
        $\img[0][0][60]{book1_definition_11_figure}$
    \end{center}
\end{minipage}

\hfill

\begin{center}
    XII.\phantomsection\label{book1def12}\\
\end{center}
\begin{minipage}{0.33\textwidth}
    \begin{center}
        $\img[0][0][60]{book1_definition_12_figure}$
    \end{center}
\end{minipage}%
\begin{minipage}{0.67\textwidth}
    \subsubsection{def. 12}
    \begin{center}
        An acute angle is leſs than a right angle.
    \end{center}
\end{minipage}

\hfill

\begin{minipage}{0.165\textwidth}
    \phantom{}
\end{minipage}%
\begin{minipage}{0.67\textwidth}
    \subsubsection{def. 13}
    \begin{center}
        XIII.\phantomsection\label{book1def13}\\
        \hfill\\
        A term or boundary is the extremity of any thing.
    \end{center}
    \subsubsection{def. 14}
    \begin{center}
        XIV.\phantomsection\label{book1def14}\\
        \hfill\\
        A figure is a ſurface encloſed on all ſides by a line or lines.
    \end{center}
\end{minipage}%
\begin{minipage}{0.165\textwidth}
    \phantom{}
\end{minipage}%


\hfill

\begin{center}
    XV.\phantomsection\label{book1def15}\\
\end{center}
\begin{minipage}{0.33\textwidth}
    \begin{center}
        $\img[0][0][60]{book1_definition_15_figure}$
    \end{center}
\end{minipage}%
\begin{minipage}{0.67\textwidth}
    \subsubsection{def. 15}
    \begin{center}
        \raggedright A circle is a plane figure, bounded by one continued line, called its circumference or periphery; and having a certain point within it, from which all ſtraight lines drawn to its circumference are equal.
    \end{center}
\end{minipage}

\hfill

\begin{minipage}{0.165\textwidth}
    \phantom{}
\end{minipage}%
\begin{minipage}{0.67\textwidth}
    \subsubsection{def. 16}
    \begin{center}
        XVI.\phantomsection\label{book1def16}\\
        \hfill\\
        \raggedright This point (from which the equal lines are drawn) is called the \mbox{centre} of the circle.
    \end{center}
\end{minipage}%
\begin{minipage}{0.165\textwidth}
    \phantom{}
\end{minipage}%

\pagebreak

\begin{center}
    XVII.\phantomsection\label{book1def17}\\
\end{center}
\begin{minipage}{0.67\textwidth}
    \subsubsection{def. 17}
    \begin{center}
        \raggedright A diameter of a circle is a ſtraight line drawn through the centre, terminated both ways in the circumference.
    \end{center}
\end{minipage}%
\begin{minipage}{0.33\textwidth}
    \begin{center}
        $\img[0][0][60]{book1_definition_17_figure}$
    \end{center}
\end{minipage}

\hfill

\begin{center}
    XVIII.\phantomsection\label{book1def18}\\
\end{center}
\begin{minipage}{0.67\textwidth}
    \subsubsection{def. 18}
    \begin{center}
        \raggedright A ſemicircle is the figure contained by the diameter, and the part of the circle cut off by the diameter.
    \end{center}
\end{minipage}%
\begin{minipage}{0.33\textwidth}
    \begin{center}
        $\img[0][0][60]{book1_definition_18_figure}$
    \end{center}
\end{minipage}

\hfill

\begin{center}
    XIX.\phantomsection\label{book1def19}\\
\end{center}
\begin{minipage}{0.67\textwidth}
    \subsubsection{def. 19}
    \begin{center}
        \raggedright A ſegment of a circle is a figure contained by a ſtraight line, and the part of the circumference which it cuts off.
    \end{center}
\end{minipage}%
\begin{minipage}{0.33\textwidth}
    \begin{center}
        $\img[0][0][60]{book1_definition_19_figure}$
    \end{center}
\end{minipage}

\hfill

\begin{minipage}{0.165\textwidth}
    \phantom{}
\end{minipage}%
\begin{minipage}{0.67\textwidth}
    \subsubsection{def. 20}
    \begin{center}
        XX.\phantomsection\label{book1def20}\\
        \hfill\\
        \raggedright A figure contained by ſtraight lines only, is called a rectilinear figure.\\
    \end{center}
\end{minipage}%
\begin{minipage}{0.165\textwidth}
    \phantom{}
\end{minipage}%

\hfill

\begin{minipage}{0.165\textwidth}
    \phantom{}
\end{minipage}%
\begin{minipage}{0.67\textwidth}
    \subsubsection{def. 21}
    \begin{center}
        XXI.\phantomsection\label{book1def21}\\
        \hfill\\
        \raggedright A triangle is a rectilinear figure included by three ſides.\\
    \end{center}
\end{minipage}%
\begin{minipage}{0.165\textwidth}
    \phantom{}
\end{minipage}

\hfill

\begin{center}
    XXII.\phantomsection\label{book1def22}\\
\end{center}
\begin{minipage}{0.33\textwidth}
    \begin{center}
        $\img[0][0][50]{book1_definition_22_figure}$
    \end{center}
\end{minipage}%
\begin{minipage}{0.67\textwidth}
    \subsubsection{def. 22}
    \begin{center}
        \raggedright A quadrilateral figure is one which is bounded by four ſides. The ſtraight lines $\blueline$ and $\redline$ connecting the vertices of the oppoſite angles of a quadrilateral figure, are called its diagonal.
    \end{center}
\end{minipage}

\hfill

\begin{minipage}{0.165\textwidth}
    \phantom{}
\end{minipage}%
\begin{minipage}{0.67\textwidth}
    \subsubsection{def. 23}
    \begin{center}
        XXIII.\phantomsection\label{book1def23}\\
        \hfill\\
        \raggedright A polygon is a rectilinear figure bounded by more than four ſides.\\
    \end{center}
\end{minipage}%
\begin{minipage}{0.165\textwidth}
    \phantom{}
\end{minipage}%

\hfill

\begin{center}
    XXIV.\phantomsection\label{book1def24}\\
\end{center}
\begin{minipage}{0.33\textwidth}
    \begin{center}
        $\img[0][0][50]{book1_definition_24_figure}$
    \end{center}
\end{minipage}%
\begin{minipage}{0.67\textwidth}
    \subsubsection{def. 24}
    \begin{center}
        \raggedright A triangle whoſe three ſides are equal, is ſaid to be equilateral.
    \end{center}
\end{minipage}

\hfill

\begin{center}
    XXV.\phantomsection\label{book1def25}\\
\end{center}
\begin{minipage}{0.33\textwidth}
    \begin{center}
        $\img[0][0][50]{book1_definition_25_figure}$
    \end{center}
\end{minipage}%
\begin{minipage}{0.67\textwidth}
    \subsubsection{def. 25}
    \begin{center}
        \raggedright A triangle which has only two ſides equal is called an iſoſceles triangle.
    \end{center}
\end{minipage}

\hfill

\begin{minipage}{0.165\textwidth}
    \phantom{}
\end{minipage}%
\begin{minipage}{0.67\textwidth}
    \subsubsection{def. 26}
    \begin{center}
        XXVI.\phantomsection\label{book1def26}\\
        \hfill\\
        \raggedright A ſcalene triangle is one which has no two ſides equal.\\
    \end{center}
\end{minipage}%
\begin{minipage}{0.165\textwidth}
    \phantom{}
\end{minipage}%

\hfill

\begin{center}
    XXVII.\phantomsection\label{book1def27}\\
\end{center}
\begin{minipage}{0.67\textwidth}
    \subsubsection{def. 27}
    \begin{center}
        A right angled triangle is that which has a right angle.
    \end{center}
\end{minipage}%
\begin{minipage}{0.33\textwidth}
    \begin{center}
        $\img[0][0][60]{book1_definition_27_figure}$
    \end{center}
\end{minipage}

\hfill

\begin{center}
    XXVIII.\phantomsection\label{book1def28}\\
\end{center}
\begin{minipage}{0.67\textwidth}
    \subsubsection{def. 28}
    \begin{center}
        \raggedright An obtuſe angled triangle is that which has an obtuſe angle.
    \end{center}
\end{minipage}%
\begin{minipage}{0.33\textwidth}
    \begin{center}
        $\img[0][0][60]{book1_definition_28_figure}$
    \end{center}
\end{minipage}

\hfill

\begin{center}
    XXIX.\phantomsection\label{book1def29}\\
\end{center}
\begin{minipage}{0.67\textwidth}
    \subsubsection{def. 29}
    \begin{center}
        \raggedright An acute angled triangle is that which has three acute angles.
    \end{center}
\end{minipage}%
\begin{minipage}{0.33\textwidth}
    \begin{center}
        $\img[0][0][60]{book1_definition_29_figure}$
    \end{center}
\end{minipage}

\hfill

\begin{center}
    XXX.\phantomsection\label{book1def30}\\
\end{center}
\begin{minipage}{0.67\textwidth}
    \subsubsection{def. 30}
    \begin{center}
        \raggedright Of four-ſided figures, a ſquare is that which has all its ſides equal, and all its angles right angles.
    \end{center}
\end{minipage}%
\begin{minipage}{0.33\textwidth}
    \begin{center}
        $\img[0][0][60]{book1_definition_30_figure}$
    \end{center}
\end{minipage}

\hfill

\begin{center}
    XXXI.\phantomsection\label{book1def31}\\
\end{center}
\begin{minipage}{0.67\textwidth}
    \subsubsection{def. 31}
    \begin{center}
        \raggedright A rhombus is that which has all its ſides equal, but its angles are not right angles.
    \end{center}
\end{minipage}%
\begin{minipage}{0.33\textwidth}
    \begin{center}
        $\img[0][0][70]{book1_definition_31_figure}$
    \end{center}
\end{minipage}

\hfill

\begin{center}
    XXXII.\phantomsection\label{book1def32}\\
\end{center}
\begin{minipage}{0.33\textwidth}
    \begin{center}
        $\img[0][0][70]{book1_definition_32_figure}$
    \end{center}
\end{minipage}%
\begin{minipage}{0.67\textwidth}
    \subsubsection{def. 32}
    \begin{center}
        \raggedright An oblong is that which has all its angles right angles, but has not all its ſides equal.
    \end{center}
\end{minipage}

\hfill

\begin{center}
    XXXIII.\phantomsection\label{book1def33}\\
\end{center}
\begin{minipage}{0.33\textwidth}
    \begin{center}
        $\img[0][0][70]{book1_definition_33_figure}$
    \end{center}
\end{minipage}%
\begin{minipage}{0.67\textwidth}
    \subsubsection{def. 33}
    \begin{center}
        \raggedright A rhomboid is that which has its oppoſite ſides equal to one another, but all its ſides are not equal, nor its angles right angles.
    \end{center}
\end{minipage}

\hfill

\begin{minipage}{0.165\textwidth}
    \phantom{}
\end{minipage}%
\begin{minipage}{0.67\textwidth}
    \subsubsection{def. 34}
    \begin{center}
        XXXIV.\phantomsection\label{book1def34}\\
        \hfill\\
        All other quadrilateral figures are called trapeziums.\\
    \end{center}
\end{minipage}%
\begin{minipage}{0.165\textwidth}
    \phantom{}
\end{minipage}%

\hfill

\begin{center}
    XXXV.\phantomsection\label{book1def35}\\
\end{center}
\begin{minipage}{0.33\textwidth}
    \begin{center}
        $\img[0][0][70]{book1_definition_35_figure}$
    \end{center}
\end{minipage}%
\begin{minipage}{0.67\textwidth}
    \subsubsection{def. 35}
    \begin{center}
        \raggedright Parallel ſtraight lines are ſuch as are in the ſame plane, and which being produced continually in both directions, would never meet.
    \end{center}
\end{minipage}

\hfill

\hfill

\begin{minipage}{0.165\textwidth}
    \phantom{}
\end{minipage}%
\begin{minipage}{0.67\textwidth}
    \subsection[Postulates]{\centering \scshape{\LARGE{POSTULATES.}}}
    \label{subsec:postulates}

    \hfill

    \subsubsection{poſt. 1}
    \begin{center}
        I.\phantomsection\label{post1}\\
        \hfill\\
        \raggedright Let it be granted that a ſtraight line may be drawn from any one point to any other point.
    \end{center}
    \subsubsection{poſt. 2}
    \begin{center}
        II.\phantomsection\label{post2}\\
        \hfill\\
        \raggedright Let it be granted that a finite ſtraight line may be produced to any length in a ſtraight line.
    \end{center}
    \subsubsection{poſt. 3}
    \begin{center}
        III.\phantomsection\label{post3}\\
        \hfill\\
        \raggedright Let it be granted that a circle may be deſcribed with any centre at any diſtance from that centre.
    \end{center}
\end{minipage}
\begin{minipage}{0.165\textwidth}
    \phantom{}
\end{minipage}%

\hfill

\begin{minipage}{0.165\textwidth}
    \phantom{}
\end{minipage}%
\begin{minipage}{0.67\textwidth}
    \subsection[Axioms]{\centering \scshape{\LARGE{AXIOMS.}}}
    \label{subsec:axioms}

    \hfill

    \subsubsection{ax. 1}
    \begin{center}
        I.\phantomsection\label{ax1}\\
        \hfill\\
        \raggedright Magnitudes which are equal to the ſame are equal to each other.\\
    \end{center}
    \subsubsection{ax. 2}
    \begin{center}
        II.\phantomsection\label{ax2}\\
        \hfill\\
        If equals be added to equals the ſums will be equal.\\
    \end{center}
    \subsubsection{ax. 3}
    \begin{center}
        III.\phantomsection\label{ax3}\\
        \hfill\\
        \raggedright If equals be taken away from equals the remainders will be equal.\\
    \end{center}
    \subsubsection{ax. 4}
    \begin{center}
        IV.\phantomsection\label{ax4}\\
        \hfill\\
        \raggedright If equals be added to unequals the ſums will be unequal.\\
    \end{center}
    \subsubsection{ax. 5}
    \begin{center}
        V.\phantomsection\label{ax5}\\
        \hfill\\
        \raggedright If equals be taken away from unequals the remainders will be unequal.
    \end{center}
\end{minipage}
\begin{minipage}{0.165\textwidth}
    \phantom{}
\end{minipage}%

\hfill

\begin{minipage}{0.165\textwidth}
    \phantom{}
\end{minipage}%
\begin{minipage}{0.67\textwidth}
    \subsubsection{ax. 6}
    \begin{center}
        VI.\phantomsection\label{ax6}\\
        \hfill\\
        \raggedright The doubles of the ſame or equal magnitudes are equal.\\
    \end{center}
    \subsubsection{ax. 7}
    \begin{center}
        VII.\phantomsection\label{ax7}\\
        \hfill\\
        \raggedright The halves of the ſame or equal magnitudes are equal.\\
    \end{center}
    \subsubsection{ax. 8}
    \begin{center}
        VIII.\phantomsection\label{ax8}\\
        \hfill\\
        \raggedright Magnitudes which coincide with one another, or exactly fill the ſame ſpace, are equal.
    \end{center}
\end{minipage}%
\begin{minipage}{0.165\textwidth}
    \phantom{}
\end{minipage}%

\hfill

\begin{minipage}{0.165\textwidth}
    \phantom{}
\end{minipage}%
\begin{minipage}{0.67\textwidth}
    \subsubsection{ax. 9}
    \begin{center}
        IX.\phantomsection\label{ax9}\\
        \hfill\\
        The whole is greater than its part.\\
    \end{center}
    \subsubsection{ax. 10}
    \begin{center}
        X.\phantomsection\label{ax10}\\
        \hfill\\
        Two ſtraight lines cannot include a ſpace.\\
    \end{center}
    \subsubsection{ax. 11}
    \begin{center}
        XI.\phantomsection\label{ax11}\\
        \hfill\\
        All right angles are equal.\\
    \end{center}
\end{minipage}%
\begin{minipage}{0.165\textwidth}
    \phantom{}
\end{minipage}

\pagebreak

\begin{center}
    XII.\phantomsection\label{ax12}\\
    \hfill
\end{center}
\begin{minipage}{0.67\textwidth}
    \subsubsection{ax. 12}
    \begin{center}
        \raggedright If two ſtraight lines (\hspace{-1ex}$\img{red_and_blue_lines}$\hspace{-1ex}) meet a third ſtraight line (\hspace{-1ex}$\blackline$\hspace{-1ex}) ſo as to make the two interior angles\\ (\hspace{-1ex}$\img[0][0][20]{yellow_angle_14}$ and $\img[0][0][20]{red_angle_4}$\hspace{-1ex}) on the ſame ſide leſs than two right angles, theſe two ſtraight lines will meet if they be produced on that ſide on which the angles are leſs than two right angles.\\
        \hfill\\
        The fifth poſtulate may be expreſſed in any of the following ways:\\
        \begin{enumerate}
            \item Two diverging ſtraight lines cannot be both parallel to the ſame ſtraight line.
            \item If a ſtraight line interſect one of the two parallel ſtraight lines it muſt also interſect the other.
            \item Only one ſtraight line can be drawn through a given point, parallel to a given ſtraight line.
        \end{enumerate}
    \end{center}
\end{minipage}%
\begin{minipage}{0.33\textwidth}
    \begin{center}
        $\img[0][0][70]{axiom_12_figure}$
    \end{center}
\end{minipage}

\pagebreak

\begin{center}
    \subsection[Elucidations]{\centering \scshape{\LARGE{ELUCIDATIONS.}}}
    \label{subsec:elucidations}
\end{center}

Geometry has for its principal objects the expoſition and explanation of the properties of \textit{figure}, and figure is defined to be the relation which ſubſiſts between the boundaries of ſpace. Space or magnitude is of three kinds, \textit{linear, ſuperficial}, and \textit{ſolid}.

\begin{wrapfigure}{r}{0.3\textwidth}
    \centering
    \includesvg[width=70pt]{vertex}
\end{wrapfigure}
Angles might properly be conſidered as a fourth ſpecies of magnitude. Angular magnitude evidently conſiſts of parts, and muſt therefore be admitted to be a ſpecies of quantity. The ſtudent muſt not ſuppoſe that the magnitude of an angle is affected by the length of the ſtraight lines which include it, and of whoſe mutual divergence it is the meaſure. The \textit{vertex} of an angle is the point where the \textit{ſides} or the \textit{legs} of the angle meet, as A.

\begin{wrapfigure}{r}{0.5\textwidth}
    \centering
    \includesvg[width=150pt]{angles_diagram}
\end{wrapfigure}
An angle is often deſignated by a ſingle letter when its legs are the only lines which meet together at its vertex. Thus the red and blue lines form the yellow angle, which in other ſyſtems would be called the angle A. But when more than two lines meet in the ſame point, it was neceſſary by former methods, in order to avoid confuſion, to employ three letters to deſignate an angle about that point, the letter which marked the vertex of the angle being always placed in the middle.  Thus the black and red lines meeting together at C, form the blue angle, and has been uſually denominated the angle FCD or DCF. The lines FC and CD are the legs of the angle; the point C is its vertex. In like manner the black angle would be deſignated the angle DCB or BCD. The red and blue angles added together, or the angle HCF added to FCD, make the angle HCD; and ſo of the other angles.  When the legs of an angle are produced or prolonged beyond its vertex, the angles made by them on both ſides of the vertex are ſaid to be vertically oppoſite to each other: Thus the red and yellow angles are ſaid to be \textit{vertically oppoſite} angles.

\textit{Superpoſition} is the proceſs by which one magnitude may be conceived to be placed upon another, ſo as exactly to cover it, or ſo that every part of each ſhall exactly coincide.

A line is ſaid to be \textit{produced}, when it is extended, prolonged, or has its length increaſed, and the increaſe of length which it receives is called its \textit{produced part}, or its \textit{production}.

The entire length of the line or lines which encloſe a figure, is called its \textit{perimeter}. The firſt ſix books of Euclid treat of plane figures only. A line drawn from the centre of a circle to its circumference, is called a \textit{radius}. The lines which include a figure are called its \textit{ſides}. That ſide of a right angled triangle, which is oppoſite to the right angle, is called the \textit{hypotenuſe}. An \textit{oblong} is defined in the ſecond book, called a \textit{rectangle}. All the lines which are conſidered in the firſt ſix books of the Elements are ſuppoſed to be in the ſame plane.

The \textit{ſtraight-edge} and \textit{compaſſes} are the only inſtruments, the uſe of which is permitted in Euclid, or plane Geometry. To declare this reſtriction is the object of the \textit{poſtulates}.

The \textit{Axioms} of geometry are certain general propoſitions, the truth of which is taken to be ſelf-evident and incapable of being eſtablished by demonſtration.

\textit{Propoſitions} are thoſe reſults which are obtained in geometry by a proceſs of reaſoning. There are two ſpecies of propoſitions in geometry, \textit{problems} and \textit{theorems}.

A \textit{Problem} is a propoſition in which ſomething is propoſed to be done; as a line to be drawn under ſome given conditions, a circle to be deſcribed, ſome figure to be conſtructed, \&c.

The \textit{ſolution} of the problem conſiſts in ſhowing how the thing required may be done by the aid of the rule or ſtraight-edge and compaſſes.

The \textit{demonſtration} conſiſts in proving that the proceſs indicated in the ſolution really attains the required end.

A \textit{Theorem} is a propoſition in which the truth of ſome principle is aſſerted. This principle muſt be deduced from the axioms and definitions, or other truths previously and independently eſtablished. To ſhow this is the object of demonſtration.

A \textit{Problem} is analogous to a poſtulate.

A \textit{Theorem} reſembles an axiom.

A \textit{Poſtulate} is a problem, the ſolution of which is aſſumed.

An \textit{Axiom} is a theorem, the truth of which is granted without demonſtration.

A \textit{Corollary} is an inference deduced immediately from a propoſition.

A \textit{Scholium} is a note or obſervation on a propoſition not containing an inference of ſufficient importance to entitle it to the name of a \textit{corollary}.

A \textit{Lemma} is a propoſition merely introduced for the purpoſe of eſtablishing ſome more important propoſition.

\pagebreak

\subsection[Symbols and Abbreviations]{\centering \scshape{\LARGE{SYMBOLS AND ABBREVIATIONS.}}}
\label{subsec:symbolsandabbreviations}

\begin{minipage}[t]{0.20\textwidth}
    \begin{center}
        $\btherefore$
    \end{center}
\end{minipage}%
\begin{minipage}[t]{0.80\textwidth}
    expreſſes the word \textit{therefore}.
\end{minipage}

\begin{minipage}[t]{0.20\textwidth}
    \begin{center}
        $\bequals$
    \end{center}
\end{minipage}%
\begin{minipage}[t]{0.80\textwidth}
    $\ldots$ \textit{equal}. This ſign of equality may be read \textit{equal to}, or \textit{is equal to}, or \textit{are equal to}; but any diſcrepancy in regard to the introduction of the auxiliary verbs \textit{is, are}, \&c. cannot affect the geometrical rigour.
\end{minipage}

\begin{minipage}[t]{0.20\textwidth}
    \begin{center}
        $\bnequals$
    \end{center}
\end{minipage}%
\begin{minipage}[t]{0.80\textwidth}
    means the ſame as if the words ‘not equal’ were written.
\end{minipage}

\begin{minipage}[t]{0.20\textwidth}
    \begin{center}
        $\bgt$
    \end{center}
\end{minipage}%
\begin{minipage}[t]{0.80\textwidth}
    ſignifies \textit{greater than}.
\end{minipage}

\begin{minipage}[t]{0.20\textwidth}
    \begin{center}
        $\blt$
    \end{center}
\end{minipage}%
\begin{minipage}[t]{0.80\textwidth}
    $\ldots$ \textit{leſs than}.
\end{minipage}

\begin{minipage}[t]{0.20\textwidth}
    \begin{center}
        $\bplus$
    \end{center}
\end{minipage}%
\begin{minipage}[t]{0.80\textwidth}
    is read \textit{plus} (\textit{more}), the ſign of addition; when interpoſed between two or more magnitudes, ſignifies their ſum.
\end{minipage}

\begin{minipage}[t]{0.20\textwidth}
    \begin{center}
        $\bminus$
    \end{center}
\end{minipage}%
\begin{minipage}[t]{0.80\textwidth}
    is read \textit{minus} (\textit{leſs}), ſignifies ſubtraction; and when placed between two quantities denotes that the latter is to be taken from the former.
\end{minipage}

\begin{minipage}[t]{0.20\textwidth}
    \begin{center}
        $\bcross$
    \end{center}
\end{minipage}%
\begin{minipage}[t]{0.80\textwidth}
    this ſign expreſſes the product of two or more numbers when placed between them in arithmetic and algebra; but in geometry it is generally uſed to expreſs a \textit{rectangle}, when placed between “two ſtraight lines which contain one of its right angles.” A \textit{rectangle} may alſo be repreſented by placing a point between two of its conterminous ſides.
\end{minipage}

% Really weird why I need this to push up the minipages to fill up
% the Symbols and Abbreviations subsection
\hfill

\hfill

\hfill

\hfill

\begin{minipage}[t]{0.20\textwidth}
    \begin{center}
        $\bcolon\ \bbcolon\ \bcolon$
    \end{center}
\end{minipage}%
\begin{minipage}[t]{0.80\textwidth}
    expreſſes an analogy or proportion; thus, if A, B, C and D, repreſent four magnitudes, and A has to B the ſame ratio that C has to D, the propoſition is thus briefly written,\\
    \begin{center}
        A $\bcolon$ B $\bbcolon$ C $\bcolon$ D\\
        A $\bcolon$ B $\bequals$ C $\bcolon$ D\\
        or $\dfrac{\text{A}}{\text{B}} \bequals \dfrac{\text{C}}{\text{D}}$.
    \end{center}
\end{minipage}

\begin{minipage}[t]{0.20\textwidth}
    \phantom{}
\end{minipage}
\begin{minipage}[t]{0.80\textwidth}
    This equality or ſameneſs of ratio is read,\\
    \begin{center}
        as A is to B, ſo is C to D;\\
        or A is to B, as C is to D.
    \end{center}
\end{minipage}

\begin{minipage}[t]{0.20\textwidth}
    \begin{center}
        $\bparallel$
    \end{center}
\end{minipage}
\begin{minipage}[t]{0.80\textwidth}
    ſignifies \textit{parallel to}.
\end{minipage}

\begin{minipage}[t]{0.20\textwidth}
    \begin{center}
        $\bperp$
    \end{center}
\end{minipage}
\begin{minipage}[t]{0.80\textwidth}
    $\ldots$ \textit{perpendicular to}.
\end{minipage}

\begin{minipage}[t]{0.20\textwidth}
    \begin{center}
        $\img{small_pie_slice}$
    \end{center}
\end{minipage}
\begin{minipage}[t]{0.80\textwidth}
    $\ldots$ \textit{angle}.
\end{minipage}

\begin{minipage}[t]{0.20\textwidth}
    \begin{center}
        $\img[0][0][20]{right_angle_2}$
    \end{center}
\end{minipage}
\begin{minipage}[t]{0.80\textwidth}
    $\ldots$ \textit{right angle}.
\end{minipage}

\begin{minipage}[t]{0.20\textwidth}
    \begin{center}
        $\tworightangles$
    \end{center}
\end{minipage}
\begin{minipage}[t]{0.80\textwidth}
    $\ldots$ \textit{two right angles}.
\end{minipage}

\begin{minipage}[t]{0.20\textwidth}
    \begin{center}
        \hspace{-2ex}$\img{three_lines_joined_at_a_point}$ or $\img[0][0][15]{two_lines_joined_at_a_point}$
    \end{center}
\end{minipage}
\begin{minipage}[t]{0.80\textwidth}
    briefly deſignates a \textit{point}.
\end{minipage}

\begin{minipage}[t]{0.20\textwidth}
    \begin{center}
        $\blackline^{\text{\large 2}}$
    \end{center}
\end{minipage}
\begin{minipage}[t]{0.80\textwidth}
    The ſquare deſcribed on a line is conciſely written thus.
\end{minipage}

\begin{minipage}[t]{0.20\textwidth}
    \begin{center}
        $\text{\large 2} \blackline^{\text{\large 2}}$
    \end{center}
\end{minipage}
\begin{minipage}[t]{0.80\textwidth}
    In the ſame manner twice the ſquare of, is expreſſed.
\end{minipage}

\begin{minipage}[t]{0.20\textwidth}
    \begin{center}
        def.
    \end{center}
\end{minipage}
\begin{minipage}[t]{0.80\textwidth}
    ſignifies \textit{definition}.
\end{minipage}

\begin{minipage}[t]{0.20\textwidth}
    \begin{center}
        poſt.
    \end{center}
\end{minipage}
\begin{minipage}[t]{0.80\textwidth}
    $\ldots$ \textit{poſtulate}.
\end{minipage}

\begin{minipage}[t]{0.20\textwidth}
    \begin{center}
        ax.
    \end{center}
\end{minipage}
\begin{minipage}[t]{0.80\textwidth}
    $\ldots$ \textit{axiom}.
\end{minipage}

\begin{minipage}[t]{0.20\textwidth}
    \begin{center}
        hyp.
    \end{center}
\end{minipage}
\begin{minipage}[t]{0.80\textwidth}
    $\ldots$ \textit{hypotheſis}. It may be neceſſary here to remark that the \textit{hypotheſis} is the condition aſſumed or taken for granted. Thus, the hypotheſis of the propoſition given in the Introduction, is that the triangle is iſoſceles, or that its legs are equal.
\end{minipage}

\begin{minipage}[t]{0.20\textwidth}
    \begin{center}
        conſt.
    \end{center}
\end{minipage}
\begin{minipage}[t]{0.80\textwidth}
    $\ldots$ \textit{conſtruction}. The \textit{conſtruction} is the change made in the original figure, by drawing lines, making angles, deſcribing circles, \&c. in order to adapt it to the argument of the demonſtration or the ſolution of the problem. The conditions under which theſe changes are made, are indisputable as thoſe contained in the hypotheſis. For inſtance, if we make an angle equal to a given angle, theſe two angles are equal by conſtruction.
\end{minipage}

\begin{minipage}[t]{0.20\textwidth}
    \begin{center}
        Q.E.F.
    \end{center}
\end{minipage}
\begin{minipage}[t]{0.80\textwidth}
    $\ldots$ \textit{Quod erat faciendum}.\\
    $\ldots$ Which was to be done.
\end{minipage}

\begin{minipage}[t]{0.20\textwidth}
    \begin{center}
        Q.E.D.
    \end{center}
\end{minipage}
\begin{minipage}[t]{0.80\textwidth}
    $\ldots$ \textit{Quod erat demonſtrandum}.\\
    $\ldots$ Which was to be demonſtrated.
\end{minipage}

\pagebreak

\subsection[Propositions]{\centering \scshape{\LARGE{PROPOSITIONS.}}}
\label{subsec:propositions}

\iconsectioninToC
% Propositions
\foreach \c in {1,...,48}{
        \input{book1/prop\c.tex}
        \newpage
    }
\stdsectioninToC

\pagestyle{fancy}
\fancyhf{}
\renewcommand{\headrulewidth}{0pt}
\fancyfoot[LE,RO]{\textsc{II.} \thepage}%

\section[Book II]{\centering BOOK II.}
\label{sec:book2}

\hfill

\subsection[Definitions]{\centering \scshape{\LARGE{DEFINITIONS.}}}
\label{subsec:definitions}

% Definitions
\foreach \c in {1,2}{
    \vspace*{\fill}
    \input{book2/def\c.tex}
    \vspace*{\fill}
    \newpage
  }

\subsection[Propositions]{\centering \scshape{\LARGE{PROPOSITIONS.}}}
\label{subsec:propositions}

\hfill

\iconsectioninToC
% Propositions
\foreach \c in {1,...,14}{
    \input{book2/prop\c.tex}
    \newpage
  }
\stdsectioninToC



\begin{minipage}{0.165\textwidth}
    \phantom{}
\end{minipage}%
\begin{minipage}{0.67\textwidth}
    \section[Book III]{\centering BOOK III.}
    \label{sec:book3}

    \hfill

    \subsection[Definitions]{\centering \scshape{\LARGE{DEFINITIONS.}}}
    \label{subsec:definitions}

    \hfill

    \subsubsection{def. 1}
    \begin{center}
        I.\phantomsection\label{book3def1}\\
        \hfill\\
        \raggedright \lettrine[lines=3, loversize=1, nindent=0pt]{\euclidinitials{E}}{}QUAL circles are thoſe whoſe diameters\\ are equal.
    \end{center}
\end{minipage}%
\begin{minipage}{0.165\textwidth}
    \phantom{}
\end{minipage}

\hfill

\begin{center}
    II.\phantomsection\label{book3def2}\\
\end{center}
\begin{minipage}{0.67\textwidth}
    \subsubsection{def. 2}
    \begin{center}
        \raggedright A right line is said to touch a circle when it meets the circle, and being produced does not cut it.
    \end{center}
\end{minipage}%
\begin{minipage}{0.33\textwidth}
    \begin{center}
        $\img[0][0][60]{book3_definition_2_figure}$
    \end{center}
\end{minipage}%

\hfill

\begin{center}
    III.\phantomsection\label{book3def3}\\
\end{center}
\begin{minipage}{0.67\textwidth}
    \subsubsection{def. 3}
    \begin{center}
        \raggedright Circles are ſaid to touch one another which meet but do not cut one another.
    \end{center}
\end{minipage}
\begin{minipage}{0.33\textwidth}
    \begin{center}
        $\img[0][0][60]{book3_definition_3_figure}$
    \end{center}
\end{minipage}

\hfill

\begin{center}
    IV.\phantomsection\label{book3def4}\\
\end{center}
\begin{minipage}{0.67\textwidth}
    \subsubsection{def. 4}
    \begin{center}
        \raggedright Right lines are ſaid to be equally diſtant from the centre of a circle when the perpendiculars are drawn to them from the centre are equal.
    \end{center}
\end{minipage}%
\begin{minipage}{0.33\textwidth}
    \begin{center}
        $\img[0][0][70]{book3_definition_4_figure}$
    \end{center}
\end{minipage}

\hfill

\begin{minipage}{0.165\textwidth}
    \phantom{}
\end{minipage}%
\begin{minipage}{0.67\textwidth}
    \subsubsection{def. 5}
    \begin{center}
        V.\phantomsection\label{book3def5}\\
        \hfill\\
        \raggedright And the ſtraight line on which the greater perpendicular falls is ſaid to be farther from the centre.
    \end{center}
\end{minipage}
\begin{minipage}{0.165\textwidth}
    \phantom{}
\end{minipage}%

\hfill

\begin{center}
    VI.\phantomsection\label{book3def6}\\
\end{center}
\begin{minipage}{0.33\textwidth}
    \begin{center}
        $\img[0][0][60]{book3_definition_6_figure}$
    \end{center}
\end{minipage}%
\begin{minipage}{0.67\textwidth}
    \subsubsection{def. 6}
    \begin{center}
        \raggedright A ſegment of a circle is the figure contained by a ſtraight line and the part of the circumference it cuts off.
    \end{center}
\end{minipage}%

\hfill

\begin{center}
    VII.\phantomsection\label{book3def7}\\
\end{center}
\begin{minipage}{0.33\textwidth}
    \begin{center}
        $\img[0][0][70]{book3_definition_7_figure}$
    \end{center}
\end{minipage}%
\begin{minipage}{0.67\textwidth}
    \subsubsection{def. 7}
    \begin{center}
        \raggedright An angle in a ſegment is the angle contained by two ſtraight lines drawn from any in the circumference of the ſegment to the extremities of the ſtraight line which is the baſe of the ſegment.
    \end{center}
\end{minipage}%

\hfill

\begin{center}
    VIII.\phantomsection\label{book3def8}\\
\end{center}
\begin{minipage}{0.33\textwidth}
    \begin{center}
        $\img[0][0][70]{book3_definition_8_figure}$
    \end{center}
\end{minipage}%
\begin{minipage}{0.67\textwidth}
    \subsubsection{def. 8}
    \begin{center}
        \raggedright An angle is ſaid to ſtand on the part of the circumference, or the arch, intercepted between the right lines that contain the angle.
    \end{center}
\end{minipage}%

\pagebreak

\begin{center}
    IX.\phantomsection\label{book3def9}\\
\end{center}
\begin{minipage}{0.67\textwidth}
    \subsubsection{def. 9}
    \begin{center}
        \raggedright A ſector of a circle is the figure contained by two radii and the arch between them.
    \end{center}
\end{minipage}%
\begin{minipage}{0.33\textwidth}
    \begin{center}
        $\img[0][0][70]{book3_definition_9_figure}$
    \end{center}
\end{minipage}%

\hfill

\begin{center}
    X.\phantomsection\label{book3def10}\\
\end{center}
\begin{minipage}{0.67\textwidth}
    \subsubsection{def. 10}
    \begin{center}
        \raggedright Similar ſegments of circles are thoſe which contain equal angles.
    \end{center}
\end{minipage}%
\begin{minipage}{0.33\textwidth}
    \begin{center}
        $\img[0][0][70]{book3_definition_10_figure_a}$
    \end{center}
\end{minipage}

\hfill

\begin{minipage}{0.67\textwidth}
    \begin{center}
        \hfill\\
        Circles which have the ſame centre are called \textit{concentric circles}.
    \end{center}
\end{minipage}%
\begin{minipage}{0.33\textwidth}
    \begin{center}
        $\img[0][0][70]{book3_definition_10_figure_b}$
    \end{center}
\end{minipage}%

\newpage

\subsection[Propositions]{\centering \scshape{\LARGE{PROPOSITIONS.}}}
\label{subsec:propositions}

\iconsectioninToC
% Propositions
\foreach \c in {1,...,37}{
        \input{book3/prop\c.tex}
        \newpage
    }
\stdsectioninToC

\pagestyle{fancy}
\fancyhf{}
\renewcommand{\headrulewidth}{0pt}
\fancyfoot[LO,RE]{\textsc{IV.} \thepage}%

\begin{minipage}{0.165\textwidth}
    \phantom{}
\end{minipage}%
\begin{minipage}{0.67\textwidth}
    \section[Book IV]{\centering BOOK IV.}
    \label{sec:book4}

    \hfill

    \subsection[Definitions]{\centering \scshape{\LARGE{DEFINITIONS.}}}
    \label{subsec:definitions}
\end{minipage}%
\begin{minipage}{0.165\textwidth}
    \phantom{}
\end{minipage}%

\hfill

\begin{center}
    I.\phantomsection\label{book4def1}\\
\end{center}
\begin{minipage}{0.67\textwidth}
    \subsubsection{def. 1}
    \begin{center}
        \raggedright \lettrine[lines=3, loversize=1, nindent=0pt]{\euclidinitials{A}}{} rectilinear figure is ſaid to be \textit{inſcribed} in another, when all the angular points of the inſcribed figure are on the ſides of the figure in which it is ſaid to be inſcribed.
    \end{center}
\end{minipage}%
\begin{minipage}{0.33\textwidth}
    \begin{center}
        $\img[0][0][60]{book4_definition_1_figure}$
    \end{center}
\end{minipage}%

\hfill

\begin{minipage}{0.1\textwidth}
    \phantom{}
\end{minipage}%
\begin{minipage}{0.80\textwidth}
    \subsubsection{def. 2}
    \begin{center}
        II.\phantomsection\label{book4def2}\\
        \hfill\\
        \raggedright A \textsc{FIGURE} is ſaid to be \textit{deſcribed about} another figure, when all the ſides of the circumſcribed figure paſs through the angular points of the other figure.
    \end{center}
\end{minipage}
\begin{minipage}{0.1\textwidth}
    \phantom{}
\end{minipage}%

\hfill

\begin{center}
    III.\phantomsection\label{book4def3}\\
\end{center}
\begin{minipage}{0.67\textwidth}
    \subsubsection{def. 3}
    \begin{center}
        \raggedright A \textsc{RECTILINEAR} figure is ſaid to be \textit{inſcribed in} a circle, when the vertex of each angle of the figure is in the circumference of the circle.
    \end{center}
\end{minipage}%
\begin{minipage}{0.33\textwidth}
    \begin{center}
        $\img[0][0][60]{book4_definition_3_figure}$
    \end{center}
\end{minipage}%

\vspace{\baselineskip}

\begin{center}
    IV.\phantomsection\label{book4def4}\\
\end{center}
\begin{minipage}{0.67\textwidth}
    \subsubsection{def. 4}
    \begin{center}
        \raggedright A \textsc{RECTILINEAR} figure is ſaid to be \textit{circumſcribed about} a circle, when each of its ſides is a tangent to the circle.
    \end{center}
\end{minipage}%
\begin{minipage}{0.33\textwidth}
    \begin{center}
        $\img[0][0][60]{book4_definition_4_figure}$
    \end{center}
\end{minipage}%

\newpage

\begin{center}
    V.\phantomsection\label{book4def5}\\
\end{center}
\begin{minipage}{0.33\textwidth}
    \begin{center}
        $\img[0][0][60]{book4_definition_5_figure}$
    \end{center}
\end{minipage}%
\begin{minipage}{0.67\textwidth}
    \subsubsection{def. 5}
    \begin{center}
        \raggedright A \textsc{CIRCLE} is ſaid to be inſcribed in a rectilinear figure, when each ſide of the figure is a tangent to the circle.
    \end{center}
\end{minipage}%

\hfill

\begin{center}
    VI.\phantomsection\label{book4def6}\\
\end{center}
\begin{minipage}{0.33\textwidth}
    \begin{center}
        $\img[0][0][70]{book4_definition_6_figure}$
    \end{center}
\end{minipage}%
\begin{minipage}{0.67\textwidth}
    \subsubsection{def. 6}
    \begin{center}
        \raggedright A \textsc{CIRCLE} is ſaid to be \textit{circumſcribed about} a rectilinear figure, when the circumference paſſes through the vertex of each angle of the figure.
    \end{center}
\end{minipage}%
\begin{center}
    $\img{triangle_14}$ is circumſcribed.
\end{center}

\hfill

\begin{center}
    VII.\phantomsection\label{book4def7}\\
\end{center}
\begin{minipage}{0.33\textwidth}
    \begin{center}
        $\img[0][0][60]{book4_definition_7_figure}$
    \end{center}
\end{minipage}%
\begin{minipage}{0.67\textwidth}
    \subsubsection{def. 7}
    \begin{center}
        \raggedright A \textsc{STRAIGHT} line is ſaid to be \textit{inſcribe in} a circle, when its extremities are in the circumference.
    \end{center}
\end{minipage}%

\hfill

\begin{minipage}{0.165\textwidth}
    \phantom{}
\end{minipage}%
\begin{minipage}{0.67\textwidth}
    \raggedright \textit{The Fourth Book of the Elements is devoted to the ſolution of problems, chiefly relating to the inſcription and circumſcription of regular polygons and circles}.\\
    \hfill\\
    A regular polygon is one whoſe angles and ſides are equal.
\end{minipage}
\begin{minipage}{0.165\textwidth}
    \phantom{}
\end{minipage}%

\newpage

\subsection[Propositions]{\centering \scshape{\LARGE{PROPOSITIONS.}}}
\label{subsec:propositions}

\iconsectioninToC
% Propositions
\foreach \c in {1,...,16}{
        \input{book4/prop\c.tex}
        \newpage
    }
\stdsectioninToC

% \pagestyle{fancy}
% \fancyhf{}
% \renewcommand{\headrulewidth}{0pt}
% \fancyfoot[C]{\textsc{V.} \thepage}%

% \section[Book V]{\centering BOOK V.}
% \label{sec:book5}

% \hfill

% \subsection[Definitions]{\centering \scshape{\LARGE{DEFINITIONS.}}}
% \label{subsec:definitions}

% \hfill

% \begin{center}
%     \begin{minipage}{0.8\textwidth}
%         \subsubsection{def. 1}
%         \begin{center}
%             I.\phantomsection\label{book5def1}\\
%             \raggedright \lettrine[lines=3, loversize=1, nindent=0pt]{\euclidinitials{A}}{} leſs magnitude is ſaid to be an aliquot part or ſubmultiple of a greater magnitude, when the leſs meaſures the greater; that is, when the leſs is contained a certain number of times exactly in the greater.
%         \end{center}
%         \subsubsection{def. 2}
%         \begin{center}
%             II.\phantomsection\label{book5def2}\\
%             \hfill\\
%             \raggedright A \textsc{GREATER} magnitude is ſaid to be a multiple of a leſs, when the greater is meaſured by the leſs; that is, when the greater contains the leſs a certain number of times exactly.
%         \end{center}
%         \subsubsection{def. 3}
%         \begin{center}
%             III.\phantomsection\label{book5def3}\\
%             \hfill\\
%             \raggedright R\textsc{ATIO} is the relation which one quantity bears to another of the ſame kind, with reſpect to magnitude.
%         \end{center}
%         \subsubsection{def. 4}
%         \begin{center}
%             IV.\phantomsection\label{book5def4}\\
%             \hfill\\
%             \raggedright M\textsc{AGNITUDES} are ſaid to have a ratio to one another, when they are of the ſame kind and the one which is not the greater can be multiplied ſo as to exceed the other.
%         \end{center}
%         \hfill\\
%         \textit{The other definitions will be given throughout the book where their aid is firſt required}.
%     \end{minipage}
% \end{center}

% \newpage

% \subsection[Axioms]{\centering \scshape{\LARGE{AXIOMS.}}}
% \label{subsec:axioms}

% \hfill

% \begin{center}
%     \begin{minipage}{0.80\textwidth}
%         \subsubsection{ax. 1}
%         \begin{center}
%             I.\phantomsection\label{book5ax1}\\
%             \raggedright \lettrine[lines=3, loversize=1, nindent=0pt]{\euclidinitials{E}}{}QUIMULTIPLES or equiſubmultiples of the ſame, or of equal magnitudes, are equal.
%         \end{center}

%         \hfill

%         \centering

%         $\begin{aligned}
%                 \text{if A}                                                       & \bequals \text{B, then}                                         \\
%                 \text{twice A}                                                    & \bequals \text{twice B, that is,}                               \\
%                 \text{2 A}                                                        & \bequals \text{2 B;}                                            \\
%                 \text{3 A}                                                        & \bequals \text{3 B;}                                            \\
%                 \text{4 A}                                                        & \bequals \text{4 B;}                                            \\
%                                                                                   & \text{\&c. \&c.}                                                \\
%                 \text{and } \dfrac{\text{\large 1}}{\text{\large 2}} \text{ of A} & \bequals \dfrac{\text{\large 1}}{\text{\large 2}} \text{ of B;} \\
%                 \dfrac{\text{\large 1}}{\text{\large 3}} \text{ of A}             & \bequals \dfrac{\text{\large 1}}{\text{\large 3}} \text{ of B;} \\
%                                                                                   & \text{\&c. \&c.}
%             \end{aligned}$\\

%         \hfill

%         \hfill

%         \subsubsection{ax. 2}
%         II.\phantomsection\label{book5ax2}\\
%         \hfill\\
%         {\raggedright A \textsc{MULTIPLE} of a greater magnitude is greater than the ſame multiple of a leſs.}\\
%         \hfill\\
%         $\begin{aligned}
%                 \text{Let A} & \bgt \text{B, then} \\
%                 \text{2 A}   & \bgt \text{2 B;}    \\
%                 \text{3 A}   & \bgt \text{3 B;}    \\
%                 \text{4 A}   & \bgt \text{4 B;}    \\
%                              & \text{\&c. \&c.}
%             \end{aligned}$
%     \end{minipage}%
% \end{center}

% \hfill

% \begin{center}
%     \begin{minipage}{0.80\textwidth}
%         \subsubsection{ax. 3}
%         \begin{center}

%             III.\phantomsection\label{book5ax3}\\
%             \hfill\\
%             {\begingroup
%             \raggedright T\textsc{HAT} magnitude, of which a multiple is greater than the ſame multiple of another, is greater than the other.\\
%             \endgroup}

%             \hfill\\
%             $\begin{aligned}
%                     \text{Let 2 A}              & \bgt \text{2 B, then}          \\
%                     \text{A}                    & \bgt \text{B;}                 \\
%                     \text{or, let 3 A}          & \bgt \text{3 B, then}          \\
%                     \text{A}                    & \bgt \text{B}                  \\
%                     \text{or, let \textit{m} A} & \bgt \text{\textit{m} B, then} \\
%                     \text{A}                    & \bgt \text{B.}                 \\
%                                                 & \text{\&c. \&c.}
%                 \end{aligned}$
%         \end{center}
%     \end{minipage}
% \end{center}

% \newpage

% \subsection[Propositions]{\centering \scshape{\LARGE{PROPOSITIONS.}}}
% \label{subsec:propositions}

% % \hfill
% \newpage

% Propositions
\foreach \c in {prop1,prop2,prop3,def5,prop4,prop5,prop6,propA,def14,propB,propC,propD,prop7,def7,prop8,prop9,prop10,prop11,prop12,prop13,prop14,prop15,def13,prop16,def16,prop17,def15,prop18,prop19,def17,propE,def18,def19,def20,prop20,prop21,prop22,prop23,prop24,prop25,def10,def11,defA,propF,propG,propH,propK}{
        \input{book5/\c.tex}
        \newpage
    }

\pagestyle{fancy}
\fancyhf{}
\renewcommand{\headrulewidth}{0pt}
\fancyfoot[LE,RO]{\textsc{VI.} \thepage}%

\begin{minipage}{0.67\textwidth}
    \section[Book VI]{\centering \textcolor{black}{BOOK VI.}}
    \label{sec:book6}

    \hfill

    \subsection[Definitions]{\centering \scshape{\LARGE{DEFINITIONS.}}}
    \label{subsec:definitions}
\end{minipage}

\hfill

\begin{minipage}{0.67\textwidth}
    \subsubsection{def. 1}
    \begin{center}
        I.\phantomsection\label{book6def1}\\
        \hfill\\
        \raggedright \lettrine[lines=3, loversize=1, nindent=0pt]{\euclidinitials{R}}{}ECTILINEAR figures are ſaid to be ſimilar, when they have their ſeveral angles equal, each to each, and the ſides about the equal angles proportional.
    \end{center}
\end{minipage}%
\begin{minipage}{0.33\textwidth}
    \begin{center}
        $\img[-13][0][75]{book6_definition_1_figure}$
    \end{center}
\end{minipage}

\hfill

\begin{minipage}{0.67\textwidth}
    \subsubsection{def. 2}
    \begin{center}
        II.\phantomsection\label{book6def2}\\
        \hfill\\
        \raggedright{T\textsc{WO} ſides of one figure are ſaid to be reciprocally proportional to two ſides of another figure when one of the ſides of the firſt is to the ſecond, as the remaining ſide of the ſecond is to the remaining ſide of the firſt.}
        \hfill\\
        \hfill\\
        \centering
        \subsubsection{def. 3}
        III.\phantomsection\label{book6def3}\\
        \hfill\\
        \raggedright{A \textsc{STRAIGHT} line is ſaid to be cut in extreme and mean ratio, when the whole is to the greater ſegment, as the greater ſegment is to the leſs.}
    \end{center}
\end{minipage}

\hfill

\begin{minipage}{0.33\textwidth}
    \phantom{}
\end{minipage}%
\begin{minipage}{0.67\textwidth}
    \subsubsection{def. 4}
    \begin{center}
        IV.\phantomsection\label{book6def4}\\
        \hfill\\
        \raggedright T\textsc{HE} altitude of any figure is the ſtraight line drawn from its vertex perpendicular to its baſe, or the baſe produced.
    \end{center}
\end{minipage}
\begin{center}
    $\img[0][0][300]{book6_definition_4_figure}$
\end{center}

\newpage

\begin{minipage}{0.46\textwidth}
    \phantom{}
\end{minipage}%
\begin{minipage}{0.54\textwidth}
    \subsection[Propositions]{\centering \scshape{\LARGE{PROPOSITIONS.}}}
    \label{subsec:propositions}
\end{minipage}

\hfill

\iconsectioninToC
% Propositions
\foreach \c in {1,...,33,A,B,C,D}{
        \input{book6/prop\c.tex}
        \newpage
    }
\stdsectioninToC




\newpage

% The End
\thispagestyle{empty}
\vspace*{\fill}
\begin{center}
  \scshape{\LARGE{THE END.}}
\end{center}
\vspace*{\fill}

\end{document}