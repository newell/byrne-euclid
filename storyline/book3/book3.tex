\pagestyle{fancy}
\fancyhf{}
\renewcommand{\headrulewidth}{0pt}
\fancyfoot[LE,RO]{\textsc{III.} \thepage}%

\begin{minipage}{0.67\textwidth}
    \section[Book III]{\centering BOOK III.}
    \label{sec:book3}

    \hfill

    \subsection[Definitions]{\centering \scshape{\LARGE{DEFINITIONS.}}}
    \label{subsec:definitions}

    \hfill

    \subsubsection{def. 1}
    \begin{center}
        I.\phantomsection\label{book3def1}\\
        \hfill\\
        \raggedright \lettrine[lines=3, loversize=1, nindent=0pt]{\euclidinitials{E}}{}QUAL circles are thoſe whoſe diameters\\ are equal.
    \end{center}
\end{minipage}

\hfill

\begin{minipage}{0.67\textwidth}
    \subsubsection{def. 2}
    \begin{center}
        II.\phantomsection\label{book3def2}\\
        \hfill\\
        A right line is said to touch a circle when it meets the circle, and being produced does not cut it.
    \end{center}
\end{minipage}%
\begin{minipage}{0.33\textwidth}
    \begin{center}
        $\img[0][0][70]{book3_definition_2_figure}$
    \end{center}
\end{minipage}%

\hfill

\begin{minipage}{0.67\textwidth}
    \subsubsection{def. 3}
    \begin{center}
        III.\phantomsection\label{book3def3}\\
        \hfill\\
        Circles are ſaid to touch one another which meet but do not cut one another.
    \end{center}
\end{minipage}
\begin{minipage}{0.33\textwidth}
    \begin{center}
        $\img[0][0][70]{book3_definition_3_figure}$
    \end{center}
\end{minipage}

\hfill

\begin{minipage}{0.67\textwidth}
    \subsubsection{def. 4}
    \begin{center}
        IV.\phantomsection\label{book3def4}\\
        \hfill\\
        Right lines are ſaid to be equally diſtant from the centre of a circle when the perpendiculars are drawn to them from the centre are equal.
    \end{center}
\end{minipage}%
\begin{minipage}{0.33\textwidth}
    \begin{center}
        $\img[0][0][70]{book3_definition_4_figure}$
    \end{center}
\end{minipage}

\hfill

\begin{minipage}{0.33\textwidth}
    \phantom{}
\end{minipage}%
\begin{minipage}{0.67\textwidth}
    \subsubsection{def. 5}
    \begin{center}
        V.\phantomsection\label{book3def5}\\
        \hfill\\
        And the ſtraight line on which the greater perpendicular falls is ſaid to be farther from the centre.
    \end{center}
\end{minipage}

\hfill

\begin{minipage}{0.33\textwidth}
    \begin{center}
        $\img[0][0][70]{book3_definition_6_figure}$
    \end{center}
\end{minipage}%
\begin{minipage}{0.67\textwidth}
    \subsubsection{def. 6}
    \begin{center}
        VI.\phantomsection\label{book3def6}\\
        \hfill\\
        A ſegment of a circle is the figure contained by a ſtraight line and the part of the circumference it cuts off.
    \end{center}
\end{minipage}%

\hfill

\begin{minipage}{0.33\textwidth}
    \begin{center}
        $\img[0][0][70]{book3_definition_7_figure}$
    \end{center}
\end{minipage}%
\begin{minipage}{0.67\textwidth}
    \subsubsection{def. 7}
    \begin{center}
        VII.\phantomsection\label{book3def7}\\
        \hfill\\
        An angle in a ſegment is the angle contained by two ſtraight lines drawn from any in the circumference of the ſegment to the extremities of the ſtraight line which is the baſe of the ſegment.
    \end{center}
\end{minipage}%

\hfill

\begin{minipage}{0.33\textwidth}
    \begin{center}
        $\img[0][0][70]{book3_definition_8_figure}$
    \end{center}
\end{minipage}%
\begin{minipage}{0.67\textwidth}
    \subsubsection{def. 8}
    \begin{center}
        VIII.\phantomsection\label{book3def8}\\
        \hfill\\
        An angle is ſaid to ſtand on the part of the circumference, or the arch, intercepted between the right lines that contain the angle.
    \end{center}
\end{minipage}%

\hfill

\begin{minipage}{0.33\textwidth}
    \begin{center}
        $\img[0][0][70]{book3_definition_9_figure}$
    \end{center}
\end{minipage}%
\begin{minipage}{0.67\textwidth}
    \subsubsection{def. 9}
    \begin{center}
        IX.\phantomsection\label{book3def9}\\
        \hfill\\
        A ſector of a circle is the figure contained by two radii and the arch between them.
    \end{center}
\end{minipage}%

\hfill

\begin{minipage}{0.67\textwidth}
    \subsubsection{def. 10}
    \begin{center}
        X.\phantomsection\label{book3def10}\\
        \hfill\\
        Similar ſegments of circles are thoſe which contain equal angles.
    \end{center}
\end{minipage}%
\begin{minipage}{0.33\textwidth}
    \begin{center}
        $\img[0][0][70]{book3_definition_10_figure_a}$
    \end{center}
\end{minipage}

\hfill

\begin{minipage}{0.67\textwidth}
    \begin{center}
        \hfill\\
        Circles which have the ſame centre are called \textit{concentric circles}.
    \end{center}
\end{minipage}%
\begin{minipage}{0.33\textwidth}
    \begin{center}
        $\img[0][0][70]{book3_definition_10_figure_b}$
    \end{center}
\end{minipage}%

\newpage

\subsection[Propositions]{\centering \scshape{\LARGE{PROPOSITIONS.}}}
\label{subsec:propositions}

\pagebreak

\iconsectioninToC
% Propositions
\foreach \c in {21}{%{1,...,37}{
        \input{book3/prop\c.tex}
        \newpage
    }
\stdsectioninToC
