% !TEX TS-program = xelatex
% !TEX options = -shell-escape -synctex=1 -interaction=nonstopmode -file-line-error "%DOC%"
\documentclass[11pt,preview]{standalone}

\usepackage{standalone}
\usepackage{lipsum}
\usepackage{fancyhdr}
\usepackage{xpatch}
% \usepackage{tocloft}
\usepackage{titletoc}
\usepackage{setspace}
\usepackage{minibox}
\usepackage{enumitem}
\usepackage{fullpage}
\usepackage{mathtools,mathrsfs}
\usepackage{amssymb,amsthm}
\usepackage{graphicx,xcolor}
\usepackage[scale=2]{ccicons}
\usepackage[breakable]{tcolorbox}
\tcbuselibrary{breakable}
\usepackage{subfig}
\usepackage{float}
\usepackage{parskip}
\usepackage{lettrine}
\usepackage{fontspec}
\usepackage[compact]{titlesec}
\usepackage{calc}
\usepackage{xparse}
\usepackage{tikz}
\usepackage{svg}
\usepackage{xr-hyper}
\usepackage[colorlinks=true, citecolor=violet, linkcolor=black, urlcolor=black]{hyperref}

% Set this PATH to the root directory of your repository
\newcommand*{\MyPath}{/home/newell/code/byrne-euclid}%
% Don't show subsubsection titles -- these still show up in TOC (as desired)
\makeatletter
\titleformat{\subsubsection}[runin]{}{}{0pt}{\@gobble}
\makeatother

% \pagestyle{fancy}
% \fancyhf{}
% \renewcommand{\headrulewidth}{0pt}
% \fancyfoot[LO,RE]{\thepage}%

% TOC - image setup
\newcounter{propimage}
\makeatletter
\newcommand\stdsectioninToC{
    \titlecontents{subsubsection}[3.8em]
    {}%
    {\contentslabel{2.3em}}%
    {\hspace*{2.3em}}%
    {\titlerule*[1em]{.}\contentspage}
}
\newcommand\iconsectioninToC{
    \titlecontents{subsubsection}[3.8em]
    {\vskip 2ex}%
    {\hspace*{-2.3em}}%
    {
        \contentslabel{2.3em}%
        \stepcounter{propimage}%
        \smash{\includegraphics[width=50pt,height=25pt,keepaspectratio]{\MyPath/toc-images/image-\the\value{propimage}}}\hspace{0.5em}%
    }%
    {\titlerule*[1em]{.}\contentspage}%
}
\AtBeginDocument{\stdsectioninToC}
\makeatother

\setcounter{secnumdepth}{-3}
\setcounter{tocdepth}{3}

% Set the Font
\setmainfont{EBGaramond-Regular.otf}[
    BoldFont = EBGaramond-Bold.otf,
    ItalicFont = EBGaramond-Italic.otf,
    BoldItalicFont = EBGaramond-BoldItalic.otf,
    SmallCapsFont = EBGaramond12-AllSC.otf,
    Numbers = {OldStyle,Proportional},
    Ligatures = {Discretionary},
    RawFeature = {+ss06}
]
\newfontfamily\euclidinitials{EuclidInitialsnormal.ttf}

% Set the SVG paths (when I tried multiple lines to make this more readable it errored on me)
\svgpath{{\MyPath/symbols/}{\MyPath/byrne-euclid-svg/lines/}{\MyPath/byrne-euclid-svg/lines/points/}{\MyPath/byrne-euclid-svg/lines/arcs/}{\MyPath/byrne-euclid-svg/circles/}{\MyPath/byrne-euclid-svg/triangles/}{\MyPath/byrne-euclid-svg/figures/}{\MyPath/byrne-euclid-svg/figures/definitions/}{\MyPath/byrne-euclid-svg/figures/propositions/}{\MyPath/byrne-euclid-svg/angles/}{\MyPath/byrne-euclid-svg/symbols/}{\MyPath/byrne-euclid-svg/lines/}{\MyPath/byrne-euclid-svg/lines/points/}{\MyPath/byrne-euclid-svg/lines/arcs/}{\MyPath/byrne-euclid-svg/circles/}{\MyPath/byrne-euclid-svg/triangles/}{\MyPath/byrne-euclid-svg/figures/}{\MyPath/byrne-euclid-svg/figures/definitions/}{\MyPath/byrne-euclid-svg/figures/propositions/}{\MyPath/byrne-euclid-svg/angles/}{\MyPath/byrne-euclid-svg/icons/}}

% Create counter for lists.
\newcounter{listcounter}

% Colors
\definecolor{cred}{RGB}{212,42,32} % red
\definecolor{cyellow}{RGB}{250,194,43} % yellow
\definecolor{cblue}{RGB}{14,99,142} % blue
\definecolor{ctrans}{RGB}{252,243,217} % transparent
\definecolor{background}{HTML}{fcf3d9}

\newlength\mytemplena
\newlength\mytemplenb
\DeclareDocumentCommand\myalignalign{sm}
{
    \settowidth{\mytemplena}{$\displaystyle #2$}%
    \setlength\mytemplenb{\widthof{$\displaystyle=$}/2}%
    \hskip-\mytemplena%
    \hskip\IfBooleanTF#1{-\mytemplenb}{+\mytemplenb}%
}

% Images
\NewDocumentCommand{\img}{%
    O{0}
    O{0}
    O{25}
    O{0}
    m
}{%
    \begin{array}{c}
        \vspace{#1ex}\hspace{#2ex}\includesvg[width=#3pt]{#5}\hspace{#4ex}
    \end{array}
}

% Common Symbols
\newcommand*{\bplus}{\img[-0.8][0][10]{plus}}
\newcommand*{\bminus}{\img[0.8][0][10]{minus}}
\newcommand*{\bcross}{\img[-0.85][0][10]{cross}}
\newcommand*{\bequals}{\img[-0.1][0][10]{equals}}
\newcommand*{\bnequals}{\img[-0.85][0][10]{nequals}}
\newcommand*{\bparallel}{\img[-0.85][0][7]{parallel}}
\newcommand*{\bnparallel}{\img[-0.85][0][11]{not_parallel}}
\newcommand*{\bperiod}{\img[-0.5][-1][3]{period}}
\newcommand*{\bmark}{\img[-0.5][-1][3]{mark}}
\newcommand*{\bmultiply}{\img[1][0][3]{period}}
\newcommand*{\bsemicolon}{\img[-1.4][-1][3]{semicolon}}
\newcommand*{\bcolon}{\img[-0.5][0][3]{colon}}
\newcommand*{\bbcolon}{\bcolon\hspace{-1.5ex}\bcolon}
\newcommand*{\bcomma}{\img[-1.5][-1][3]{comma}}
\newcommand*{\btherefore}{\img[-0.5][0][10]{therefore}}
\newcommand*{\bgt}{\img[-0.85][0][10]{greater_than}}
\newcommand*{\blt}{\img[-0.85][0][10]{less_than}}
\newcommand*{\bperp}{\img[0][0][15]{perpendicular}}

% Common Lines
\newcommand*{\bimg}[1]{\img[0.8][0][45]{#1}}
\newcommand*{\redlines}{\bimg{red_lines}}
\newcommand*{\bluelines}{\bimg{blue_lines}}
\newcommand*{\blacklines}{\bimg{black_lines}}

\newcommand*{\redline}{\img[0.8][0][30]{red_line}}
\newcommand*{\redthinline}{\img[0.8][0][30]{red_thin_line}}
\newcommand*{\dottedredline}{\img[0.8][0][30]{red_dotted_line}}
\newcommand*{\blueline}{\img[0.8][0][30]{blue_line}}
\newcommand*{\bluethinline}{\img[0.8][0][30]{blue_thin_line}}
\newcommand*{\dottedblueline}{\img[0.8][0][30]{blue_dotted_line}}
\newcommand*{\blackline}{\img[0.8][0][30]{black_line}}
\newcommand*{\blackthinline}{\img[0.8][0][30]{black_thin_line}}
\newcommand*{\blackthindottedline}{\img[0.8][0][30]{black_thin_dotted_line}}
\newcommand*{\dottedblackline}{\img[0.8][0][30]{black_dotted_line}}
\newcommand*{\yellowline}{\img[0.8][0][30]{yellow_line}}
\newcommand*{\yellowthinline}{\img[0.8][0][30]{yellow_thin_line}}
\newcommand*{\dottedyellowline}{\img[0.8][0][30]{yellow_dotted_line}}

% Common Angles
\newcommand*{\tworightangles}{\img[0][0][30]{two_right_angles}}

% Common Shapes -- Book 5
\newcommand*{\reddome}{\img[-0.8][-0.5][10][-0.5]{red_dome_icon}}
\newcommand*{\bluedome}{\img[-0.8][-0.5][10][-0.5]{blue_dome_icon}}
\newcommand*{\blackdome}{\img[-0.8][-0.5][10][-0.5]{black_dome_icon}}
\newcommand*{\yellowdome}{\img[-0.8][-0.5][10][-0.5]{yellow_dome_icon}}

\newcommand*{\redcircle}{\img[-0.8][-0.5][10][-0.5]{red_circle_icon}}
\newcommand*{\bluecircle}{\img[-0.8][-0.5][10][-0.5]{blue_circle_icon}}
\newcommand*{\blackcircle}{\img[-0.8][-0.5][10][-0.5]{black_circle_icon}}
\newcommand*{\yellowcircle}{\img[-0.8][-0.5][10][-0.5]{yellow_circle_icon}}

\newcommand*{\reddrop}{\img[-0.8][-0.5][10][-0.5]{red_drop_icon}}
\newcommand*{\bluedrop}{\img[-0.8][-0.5][10][-0.5]{blue_drop_icon}}
\newcommand*{\blackdrop}{\img[-0.8][-0.5][10][-0.5]{black_drop_icon}}
\newcommand*{\yellowdrop}{\img[-0.8][-0.5][10][-0.5]{yellow_drop_icon}}

\newcommand*{\redhome}{\img[-0.8][-0.5][10][-0.5]{red_home_icon}}
\newcommand*{\bluehome}{\img[-0.8][-0.5][10][-0.5]{blue_home_icon}}
\newcommand*{\blackhome}{\img[-0.8][-0.5][10][-0.5]{black_home_icon}}
\newcommand*{\yellowhome}{\img[-0.8][-0.5][10][-0.5]{yellow_home_icon}}

\newcommand*{\redsquare}{\img[-0.8][-0.5][10][-0.5]{red_square_icon}}
\newcommand*{\smallredsquare}{\img[-0.8][-0.5][5][-0.5]{red_square_icon}}
\newcommand*{\bluesquare}{\img[-0.8][-0.5][10][-0.5]{blue_square_icon}}
\newcommand*{\bblacksquare}{\img[-0.8][-0.5][10][-0.5]{black_square_icon}}
\newcommand*{\yellowsquare}{\img[-0.8][-0.5][10][-0.5]{yellow_square_icon}}
\newcommand*{\transsquare}{\img[-0.8][-0.5][10][-0.5]{trans_square_icon}}

\newcommand*{\reddiamond}{\img[-0.8][-0.5][8][-0.5]{red_diamond_icon}}
\newcommand*{\bluediamond}{\img[-0.8][-0.5][8][-0.5]{blue_diamond_icon}}
\newcommand*{\blackdiamond}{\img[-0.8][-0.5][8][-0.5]{black_diamond_icon}}
\newcommand*{\yellowdiamond}{\img[-0.8][-0.5][8][-0.5]{yellow_diamond_icon}}
\newcommand*{\transdiamond}{\img[-0.8][-0.5][8][-0.5]{trans_diamond_icon}}

\newcommand*{\redtriangle}{\img[-0.8][-0.5][10][-0.5]{red_triangle_icon}}
\newcommand*{\bluetriangle}{\img[-0.8][-0.5][10][-0.5]{blue_triangle_icon}}
\newcommand*{\bblacktriangle}{\img[-0.8][-0.5][10][-0.5]{black_triangle_icon}}

\newcommand*{\blackrectangle}{\img[-0.8][-0.5][10][-0.5]{black_rectangle_icon}}
\newcommand*{\yellowrectangle}{\img[-0.8][-0.5][10][-0.5]{yellow_rectangle_icon}}






\begin{document}
\null\removelastskip\nointerlineskip\vspace*{-\baselineskip}

\subsubsection{VIII. Theorem}

\begin{minipage}[t]{0.54\textwidth}
    \begin{center}
        The original text of this propoſition is here divided into three parts.\\
        \hfill\\
        \textit{PROPOSITION VIII. THEOREM.}\phantomsection\label{book3pr8}
    \end{center}

    \hfill

    \begin{center}
        I.\\
        \raggedright \lettrine[lines=3, loversize=1, nindent=0pt]{\euclidinitials{I}}{}F \textit{from a point without a circle,\\ ſtraight lines} $\left\{\begin{aligned}                                                            & \bimg{black_and_dotted_black_line_7} \\[-2ex] & \redline \\[-2ex] & \blueline \textit{\&c.}
            \end{aligned}\right\}$
    \end{center}
    \raggedright \textit{are drawn to the circumference; of thoſe falling upon the concave circumference the greateſt is that} (\hspace{-1ex}$\bimg{black_and_dotted_black_line_7}$\hspace{-1ex}) \textit{which paſſes through the centre, and the line} (\hspace{-1ex}$\redline$\hspace{-1ex}) \textit{which is nearer the greateſt is greater than  that} (\hspace{-1ex}$\blueline$\hspace{-1ex}) \textit{which is more remote}.
\end{minipage}%
\hfill
\begin{minipage}[t]{0.43\textwidth}
    \vspace{55pt}
    \includesvg[width=\textwidth]{book3_proposition_8_figure_1}
\end{minipage}%

\hfill

\begin{center}
    Draw $\dottedblueline$ and $\dottedredline$ to the centre.
    \hfill\\
    \hfill\\
    \raggedright Then, $\bimg{black_and_dotted_black_line_7}$ which paſſes through the centre, is greateſt; for ſince\\
    $\dottedblackline \bequals \dottedredline \bcomma$ if $\blackline$ be added to both, $\bimg{black_and_dotted_black_line_7} \bequals \blackline \bplus \dottedredline \bsemicolon$ but $\bgt \redline$ [\hyperref[book1pr20]{\textsc{I.} 20}] $\btherefore \bimg{black_and_dotted_black_line_7}$ is greater than any other line drawn from the ſame point to the concave circumference.
\end{center}

\begin{center}
    Again in $\img[0][0][17]{blue_triangle_5}$ and $\img[0][0][10]{red_triangle_6} \bcomma \dottedblueline \bequals \dottedredline \bcomma$\\
    and $\blackline$ common, but $\img[0][0][25]{yellow_and_black_angles} \bgt \img{yellow_angle_31} \bcomma$\\
    $\btherefore \redline \bgt \blueline$ [\hyperref[book1pr24]{\textsc{I.} 24}];\\
    and in like manner $\redline$ may be ſhewn $\bgt$ than any other line more remote from $\bimg{black_and_dotted_black_line_7} \bperiod$
\end{center}

\begin{minipage}[t]{0.43\textwidth}
    \vspace{0pt}
    \includesvg[width=\textwidth]{book3_proposition_8_figure_2}
\end{minipage}%
\hfill
\begin{minipage}[t]{0.54\textwidth}
    \vspace{0pt}

    \begin{center}
        II.
    \end{center}
    \hfill\\
    \raggedright \textit{Of thoſe lines falling on the convex circumference the leaſt is that} (\hspace{-1ex}$\dottedblackline$\hspace{-1ex}) \textit{which being produced would paſs through the centre, and the line which is nearer to the leaſt is leſs than that which is more remote}.
\end{minipage}%

\hfill

\begin{center}
    For, ſince $\redline \bplus \dottedredline \bgt \bimg{black_and_dotted_black_line_8}$ [\hyperref[book1pr20]{\textsc{I.} 20}]\\
    and $\redline \bequals \blackline \bcomma$\\
    $\btherefore \dottedredline \bgt \dottedblackline$ [\hyperref[ax5]{ax. 5}].\\
    And again, ſince $\blueline \bplus \dottedblueline \bgt \redline \bplus \dottedredline$\\
    \hspace{0ex} [\hyperref[book1pr21]{\textsc{I.} 21}], and $\blueline \bequals \redline \bcomma$\\
    $\btherefore \dottedredline \blt \dottedblueline \bperiod$ And ſo of others.
\end{center}

\hfill

\begin{minipage}[t]{0.54\textwidth}
    \vspace{0pt}

    \begin{center}
        III.
    \end{center}
    \hfill\\
    \raggedright \textit{Alſo the lines making equal angles with that which paſſes through the centre are equal, whether falling on the concave or convex circumference; and no third line can be drawn equal to them from the ſame point to the circumference}.
\end{minipage}%
\hfill
\begin{minipage}[t]{0.43\textwidth}
    \vspace{0pt}
    \includesvg[width=\textwidth]{book3_proposition_8_figure_3}
\end{minipage}%

\hfill

\begin{center}
    For if $\bimg{dotted_red_and_yellow_line} \bgt \dottedblueline \bcomma$ but making $\img[0][0][15]{yellow_angle_32} \bequals \img[0][0][15]{blue_angle_29} \bsemicolon$\\
    make $\dottedredline \bequals \dottedblueline \bcomma$ and draw $\bimg{dotted_black_and_red_line_2} \bperiod$\\
    Then in $\img[0][0][10]{red_triangle_7}$ and $\img[0][0][13]{blue_triangle_6}$ we have $\dottedredline \bequals \dottedblueline \bcomma$\\
    and $\blackline$ common, and alſo $\img[0][0][15]{blue_angle_29} \bequals \img[0][0][15]{yellow_angle_32} \bcomma$\\
    $\btherefore \bimg{dotted_black_and_red_line_2} \bequals \blueline$ [\hyperref[book1pr4]{\textsc{I.} 4}];\\
    but $\blueline \bequals \dottedblackline \bsemicolon$\\
    $\btherefore \dottedblackline \bequals \bimg{dotted_black_and_red_line_2} \bcomma$ which is abſurd.\\
    $\btherefore \dottedblueline$ is not $\bequals \dottedredline \bcomma$ not to any part\\
    of $\bimg{dotted_red_and_yellow_line} \bcomma \btherefore \bimg{dotted_red_and_yellow_line}$ is not $\bgt \dottedblueline \bperiod$\\
    Neither is $\dottedblueline \bgt \bimg{dotted_red_and_yellow_line} \bcomma$ they are\\
    $\btherefore \bequals$ to each other.
\end{center}
\hfill\\
And any other line drawn from the ſame point to the circumference muſt lie at the ſame ſide with one of theſe lines, and be more or leſs remote than it from the line paſſing through the centre, and cannot therefore be equal to it.

\hfill

\hfill Q.E.D.

\end{document}