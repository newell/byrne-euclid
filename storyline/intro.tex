% % \begin{minipage}{0.20\textwidth}
% %     \phantom{}
% % \end{minipage}%
% % \begin{minipage}{0.80\textwidth}
%     \section[Introduction]{\centering {INTRODUCTION.}}
%     \label{sec:intro}

%     \hfill

%     \lettrine[lines=3, loversize=1, nindent=0pt]{\euclidinitials{T}}{}HE arts and ſciences have become ſo extenſive, that to facilitate their acquirement is of as much importance as to extend their boundaries. Illuſtration, if it does not ſhorten the time of ſtudy, will at leaſt make it more agreeable. T\textsc{HIS WORK} has a greater aim than mere illuſtration; we do not introduce colours for the purpoſe of entertainment, or to amuſe \textit{by certain combinations of tint and form}, but to aſſiſt the mind in its reſearches after truth, to increaſe the facilities of inſlruction, and to diffuſe permanent knowledge. If we wanted authorities to prove the importance and uſefulneſs of geometry, we might quote every philoſopher ſince the days of Plato. Among the Greeks, in ancient, as in the ſchool of Peſtalozzi and others in recent times, geometry was adopted as the beſt gymnaſtic of the mind. In fact, Euclid’s Elements have become, by common conſent, the baſis of mathematical ſcience all over the civilized globe. But this will not appear extraordinary, if we conſider that this ſublime ſcience is not only better calculated than any other to call forth the ſpirit of inquiry, to elevate the mind, and to ſtrengthen the reaſoning faculties, but alſo it forms the beſt introduction to moſt of the uſeful and important vocations of human life.  Arithmetic, land-ſurveying, menſuration, engineering, navigation, mechanics, hydroſtatics, pneumatics, optics, phyſical aſtronomy, \&c. are all dependent on the propoſitions of geometry.

%     \hfill

%     Much however depends on the firſt communication of any ſcience to a learner, though the beſt and moſt eaſy methods are ſeldom adopted.  Propoſitions are placed before a ſtudent,
% % \end{minipage}

% % \newpage

% % \begin{minipage}{0.80\textwidth}
%     who though having a ſufficient underſtanding, is told juſt as much about them on entering at the very threſhold of the ſcience, as gives him a prepoſſeſſion moſt unfavourable to his future ſtudy of this delightful ſubject; or “the formalities and paraphernalia of rigour are ſo oſtentatiouſly put forward, as almoſt to hide the reality. Endleſs and perplexing repetitions, which do not confer greater exactitude on the reaſoning, render the demonſtrations involved and obſcure, and conceal from the view of the ſtudent the conſecution of evidence.”\\

%     Thus an averſion is created in the mind of the pupil, and a ſubject ſo calculated to improve the reaſoning powers, and give the habit of cloſe thinking, is degraded by a dry and rigid courſe of inſtruction into an unintereſting exerciſe of the memory. To raiſe the curioſity, and to awaken the liſtleſs and dormant powers of younger minds ſhould be the aim of every teacher; but where examples of excellence are wanting, the attempts to attain it are but few, while eminence excites attention and produces imitation. The object of this Work is to introduce a method of teaching geometry, which has been much approved of by many ſcientific men in this country, as well as in France and America. The plan here adopted forcibly appeals to the eye, the moſt ſenſitive and the moſt comprehenſive of our external organs, and its pre-eminence to imprint it ſubject on the mind is ſupported by the incontrovertible maxim expreſſed in the well known words of Horace:—\\

%     \begin{quotation}
%         \noindent \small{\textit{Segnius irritant animos demiſſa per aurem\\
%                 Quàm quæ ſunt oculis ſubjecta fidelibus}}.\\

%         \noindent \small{A feebler impreſs through the ear is made,\\
%             Than what is by the faithful eye conveyed}.
%     \end{quotation}
% % \end{minipage}

% % \newpage

% % \begin{minipage}{0.20\textwidth}
%     % \phantom{}
% % \end{minipage}%
% % \begin{minipage}{0.80\textwidth}
%     All language conſiſts of repreſentative ſigns, and thoſe ſigns are the beſt which effect their purpoſes with the greateſt preciſion and diſpatch. Such for all common purpoſes are the audible ſigns called words, which are ſtill conſidered as audible, whether addreſſed immediately to the ear, or through the medium of letters to the eye. Geometrical diagrams are not ſigns, but the materials of geometrical ſcience, the object of which is to ſhow the relative quantities of their parts by a proceſs of reaſoning called Demonſtration.

%     This reaſoning has been generally carried on by words, letters, and black or uncoloured diagrams but as the uſe of coloured ſymbols, ſigns, and diagrams in the linear arts and ſciences, renders the proceſs of reaſoning more preciſe, and the attainment more expeditious, they have been in this inſtance accordingly adopted.\\

%     Such is the expedition of this enticing mode of communicating knowledge, that the Elements of Euclid can be acquired in leſs than one third the time uſually employed, and the retention by the memory is much more permanent; theſe facts have been aſcertained by numerous experiments made by the inventor, and ſeveral others who have adopted his plans. The particulars of which are few and obvious; the letters annexed to points, lines, or other parts of a diagram are in fact but arbitrary names, and repreſent them in the demonſtration; inſtead of theſe, the parts being differently coloured, are made to name themſelves, for their forms in correſponding colours represent them in the demonſtration.\\

%     In order to give a better idea of this ſyſtem, and of the advantages gained by its adoption, let us take a right angled triangle,
% % \end{minipage}

% % \newpage

% % \begin{minipage}{0.80\textwidth}
%     \begin{center}
%         $\img[0][3][175]{triangle_abc}$
%     \end{center}
%     \hfill\\
%     and expreſs ſome of its properties both by colours and the
%     method generally employed.\\
%     \begin{center}
%         \textit{Some of the properties of the right angled triangle ABC, expreſſed by the method generally employed}.\\
%         \hfill\\
%         \begin{enumerate}
%             \item $\img{red_angle_60} \bplus \img{yellow_angle_65} \bplus \img{blue_angle_64} \bequals {\text{\large 2}} \img{yellow_angle_65}$\\ $\bequals \tworightangles \bperiod$\\ That is, the red angle added to the yellow angle added to the blue angle, equal twice the yellow angle, equal two right angles.
%             \item $\img{red_angle_60} \bplus \img{blue_angle_64} \bequals \img{yellow_angle_65} \bperiod$\\ Or in words, the red angle added to the blue angle, equal the yellow angle.
%             \item $\img{yellow_angle_65} \bgt \img{red_angle_60}$ or $\img{blue_angle_64} \bperiod$\\ The yellow angle is greater than either the red or blue angle.
%             \item $\img{red_angle_60}$ or $\img{blue_angle_64} \blt \img{yellow_angle_65} \bperiod$\\ Either the red or blue angle is leſs than the yellow angle.
%             \item $\img{yellow_angle_65}$ minus $\img{blue_angle_64} \bequals \img{red_angle_60} \bperiod$\\ In other terms, the yellow angle made leſs by the blue angle equal the red angle.
%             \item $\yellowline^{\text{\large 2}} \bequals \blueline^{\text{\large 2}} \bplus \redline^{\text{\large 2}} \bperiod$\\ That is, the ſquare of the yellow line is equal to the ſum of the ſquares of the blue and red lines.
%         \end{enumerate}
%     \end{center}
% % \end{minipage}

% % \newpage

% % \begin{minipage}{0.20\textwidth}
% %     \phantom{}
% % \end{minipage}%
% % \begin{minipage}{0.80\textwidth}
%     In oral demonſtrations we gain with colours this important advantage, the eye and the ear can be addreſſed at the ſame moment, ſo that for teaching geometry, and other linear arts and ſciences, in claſſes, the ſyſtem is the beſt ever propoſed, this is apparent from the examples juſt given.\\

%     Whence it is evident that a reference from the text to the diagram is more rapid and ſure, by giving the forms and colours of the parts, or by naming the parts and their colours, than naming the parts and letters on the diagram. Beſides the ſuperior ſimplicity, this ſyſtem is likewiſe conſpicuous for concentration, and wholly excludes the injurious through prevalent practice of allowing the ſtudent to commit the demonſtration to memory; until reaſon, and fact, and proof only make impreſſions on the underſtanding.\\

%     Again, when lecturing on the principles or properties of figures, if we mention the colour of the part or parts referred to, as in ſaying, the red angle, the blue line, or lines, \&c. the part or parts thus named will be immediately ſeen by all in the claſs at the ſame inſtant; not ſo if we ſay the angle ABC, the triangle PFQ, the figure EGKt, and ſo on; for the letters muſt be traced one by one before the ſtudents arrange in their minds the particular magnitude referred to, which often occaſions confuſion and error, as well as loſs of time.\\

%     Alſo if the parts which are given as equal, have the ſame colours in any diagram, the mind will not wander from the object before it; that is, ſuch an arrangement preſents an ocular demonſtration of the parts to be proved equal, and the learner retains the data throughout the whole of the reaſoning.
% % \end{minipage}

% % \newpage

% % \begin{minipage}{0.80\textwidth}
%     But whatever may be the advantages of the preſent plan, if it be not ſubſtituded for, it can always be made a powerful auxiliary to the other methods, for the purpoſe of introduction, or of a more ſpeedy reminiſcence, or of more permanent retention by the memory.\\

%     The experience of all who have formed ſyſtems to impreſs facts on the underſtanding, agree in proving that coloured repreſentations, as pictures, cuts, diagrams, \&c. are more eaſily fixed in the mind than mere ſentences unmarked by any peculiarity. Curious as it may appear, poets ſeem to be aware of this fact more than mathematicians; many modern poets allude to this viſible ſyſtem of communicating knowledge, one of them has thus expreſſed himſelf:\\
%     \begin{quotation}
%         \small{\noindent Sounds which addreſs the ear are loſt and die\\
%             In one ſhort hour, but theſe which ſtrike the eye,\\
%             Live long upon the mind, the faithful ſight\\
%             Engraves the knowledge with a beam of light.}
%     \end{quotation}

%     \hfill

%     This perhaps may be reckoned the only improvement which plane geometry has received ſince the days of Euclid, and if there were any geometers of note before that time, Euclid’s ſucceſs has quite eclipſed their memory, and even occaſioned all good things of that kind to be aſſigned to him; like Æſop among the writers of Fables. It may alſo be worthy of remark, as tangible diagrams afford the only medium through which geometry and other linear arts and ſciences can be taught to the blind, this viſible ſyſtem is no leſs adapted to the exigencies of the deaf and dumb.\\

%     Care muſt be taken to ſhow that colour has nothing to do with the lines, angles, or magnitudes, except merely to name them.
% % \end{minipage}

% % \newpage

% % \begin{minipage}{0.20\textwidth}
% %     \phantom{}
% % \end{minipage}%
% % \begin{minipage}{0.80\textwidth}
%     A mathematical line, which is length without breadth, cannot poſſeſs colour, yet the junction of the two colours on the ſame plane gives a good idea of what is meant by a mathematical line; recollect we are ſpeaking familiarly, ſuch a junction is to be underſtood and not the colour, when we ſay the black line, the red line or lines, \&c.  Colours and coloured diagrams may at firſt appear a clumſy method to convey proper notations of the properties and parts of mathematical figures and magnitudes, however they will be found to afford a means more refined and extenſive than any that has been hitherto propoſed.\\

%     We ſhall here define a point, a line, and a surface, and demonſtrate a propoſition in order to ſhow the truth of this aſſertion.  A point is that which has poſition, but not magnitude; or a point is poſition only, abſtracted from the conſideration of length, breadth, and thickneſs. Perhaps the following deſcription is better calculated to explain the nature of a mathematical point to thoſe who have not acquired the idea, than the above ſpecious definition.\\

%     Let three colours meet and cover a portion of the paper, where they meet is not blue, nor is it yellow, nor is it red,
%     \begin{wrapfigure}{l}{0.5\textwidth}
%         \centering
%         \includesvg[width=100pt]{circle_divided_into_three_equal_slices}
%     \end{wrapfigure}
%     as it occupies no portion of the plane, for if it did, it would belong to the blue, the red, or the yellow part; yet it exiſts, and has poſition without magnitude, ſo that with a little reflection, this junction of three colours on a plane gives a good idea of a mathematical point.
% % \end{minipage}

% % \newpage

% % \begin{minipage}{0.80\textwidth}
%     A line is length without breadth. With the aſſiſtance of colours, nearly in the ſame manner as before, an idea of a line may be thus given:—\\

%     Let two colours meet and cover a portion of the paper; where they meet is not red, nor is it blue;
%     \begin{wrapfigure}{l}{0.5\textwidth}
%         \centering
%         \includesvg[width=100pt]{two_colors_stacked}
%     \end{wrapfigure}
%     therefore the junction occupies no portion of the plane, and therefore it cannot have breadth but only length: from which we can readily form an idea of what is meant by a mathematical line.  For the purpoſe of illuſtration, one colour differing from the colour of the paper, or plane upon which it is drawn, would have been ſufficient; hence in future, if we ſay the red line, the blue line, or lines, \&c. it is the junctions with the plane upon which they are drawn are to be underſtood.\\

%     Surface is that which has length and breadth without thickneſs.
%     \begin{wrapfigure}{l}{0.5\textwidth}
%         \centering
%         \includesvg[width=80pt]{figure_pq}
%     \end{wrapfigure}
%     When we conſider a ſolid body (PQ), we perceive at once that it has three dimensions, namely:—length, breadth, and thickneſs;  ſuppoſe one part of this ſolid (PS) to be red, and the other part (QR) yellow, and that the colours be diſtinct without commingling, the blue surface (RS) which ſeparates theſe parts, or which is the ſame thing, that which divides the ſolid without loſs of material, muſt be without thickneſs, and only poſſeſſes length and breadth;
% % \end{minipage}%

% % \newpage

% % \begin{minipage}{0.20\textwidth}
% %     \phantom{}
% % \end{minipage}%
% % \begin{minipage}{0.80\textwidth}
%     this plainly appears from reaſoning, ſimilar to that juſt employed in defining, or rather deſcribing a point and a line.\\

%     The propoſition which we have ſelected to elucidate the manner in which the principles are applied is the fifth of the firſt Book.\\

%     \begin{wrapfigure}{r}{0.5\textwidth}
%         \centering
%         \includesvg[width=150pt]{proposition_5_figure}
%     \end{wrapfigure}
%     In an iſoſceles triangle ABC, the internal angles at the baſe ABC, ACB are equal, and when the ſides AB, AC are produced, the external angles at the baſe BCE, CBD are alſo equal.\\

%     \raggedright Produce $\redline$ and $\redline \bperiod$  Make $\yellowline \bequals \yellowline \bperiod$

%     \hfill

%     \hfill

%     \begin{center}
%         Draw $\blueline \bequals \blueline$ [\hyperref[book1pr3]{\textsc{I.} 3}]\\
%         in $\img{left_triangle}$ and $\img{right_triangle}$\\
%         we have $\bimg{red_and_yellow_lines} \bequals \bimg{red_and_yellow_lines}$\\
%         $\redline \bequals \redline$ and $\img[0][0][20]{black_angle}$ common:\\
%         $\btherefore \blueline \bequals \blueline \bcomma \img[0][0][20]{left_blue_and_yellow_angles} \bequals \img[0][0][20]{right_blue_and_yellow_angles}$\\
%         and $\img{left_red_angle} \bequals \img{right_red_angle}$ [\hyperref[book1pr4]{\textsc{I.} 4}].\\
%         Again in $\img[0][0][40]{lower_left_triangle}$ and $\img[0][0][40]{lower_right_triangle} \bcomma$\\
%     \end{center}
% % \end{minipage}

% % % \newpage

% % \begin{minipage}{0.80\textwidth}
%     \begin{center}
%         $\yellowline \bequals \yellowline \bcomma$\\
%         $\blueline \bequals \blueline \bcomma$\\
%         and $\img{left_red_angle} \bequals \img{right_red_angle} \bsemicolon$\\
%         $\btherefore \img[-0.5]{left_yellow_angle_plus_remaining_angle} \bequals \img[-0.5]{right_yellow_angle_plus_remaining_angle}$\\
%         and $\img[-0.5]{left_yellow_angle} \bequals \img[-0.5]{right_yellow_angle}$ [\hyperref[book1pr4]{\textsc{I.} 4}].\\
%         But $\img[0][0][20]{left_blue_and_yellow_angles} \bequals \img[0][0][20]{right_blue_and_yellow_angles} \bcomma$\\
%         $\btherefore \img{left_blue_angle} \bequals \img{right_blue_angle} \bperiod$\\
%         \hfill\\
%         \hfill Q.E.D.\\
%         \hfill\\
%         \textit{By annexing Letters to the Diagram}.
%     \end{center}

%     \hfill

%     L\textsc{ET} the equal ſides AB and AC be produced through the extremities BC, of the third ſide, and in the produced part BD of either, let any point D be aſſumed, and from the other let AE be cut off equal to AD [\hyperref[book1pr3]{\textsc{I.} 3}]. Let the points E and D, ſo taken in the produced ſides, be connected by ſtraight lines DC and BE with the alternate extremities of the third ſide of the triangle.\\

%     In the triangles DAC and EAB the ſides DA and AC are reſpectively equal to EA and AB, and the included angle A is common to both triangles. Hence [\hyperref[book1pr4]{\textsc{I.} 4}] the line DC is equal to BE, the angle ADC to the angle AEB, and the angle ACD to the angle ABE; if from the equal lines AD and AE the equal ſides AB and AC be taken, the remainders BD and CE will be equal. Hence in the triangles BDC and CEB, the ſides BD and DC are reſpectively equal to CE and EB, and the angles D and E included by thoſe ſides are alſo equal. Hence [\hyperref[book1pr4]{\textsc{I.} 4}] the angles DBC and ECB, which are thoſe included by the third ſide BC and the
% % \end{minipage}

% % \newpage

% % \begin{minipage}{0.20\textwidth}
% %     \phantom{}
% % \end{minipage}%
% % \begin{minipage}{0.80\textwidth}
%     productions of the equal ſides AB and AC are equal.  Alſo the angles DCB and EBC are equal if thoſe equals be taken from the angles DCA and EBA before proved equal, the remainders, which are the angles ABC and ACB oppoſite to the equal ſides, will be equal.\\

%     \textit{Therefore in an iſoſceles triangle, \&c}.\\

%     \hfill Q.E.D.\\

%     Our object in this place being to introduce the ſyſtem rather than to teach any particular ſet of propoſitions, we have therefore ſelected the foregoing out of the regular courſe. For ſchools and other public places of inſtruction, dyed chalks will anſwer to deſcribe diagrams, \&c. for private uſe coloured pencils will be found very convenient.\\

%     We are happy to find that the Elements of Mathematics now forms a conſiderable part of every ſound female education, therefore we call the attention of thoſe intereſted or engaged in the education of ladies to this very attractive mode of communicating knowledge, and to the ſucceeding work for its future development.\\

%     We ſhall for the preſent conclude by obſerving, as the ſenſes of ſight and hearing can be ſo forcibly and inſtantaneously addreſſed alike with one thouſand as with one, \textit{the million} might be taught geometry and other branches of mathematics with great eaſe, this would advance the purpoſe of education more than any thing that \textit{might} be named, for it would teach the people how to think, and not what to think; it is in this particular the great error of education originates.
% % \end{minipage}


%%%%%%%%%%%%%%%%%%%%%%%%%%%%%%%%%%%%%%%%%%%%%%%%
\pagestyle{fancy}
\fancyhf{}
\renewcommand{\headrulewidth}{0pt}
\fancyfoot[C]{\thepage}%

\section[Introduction]{\centering {INTRODUCTION.}}
\label{sec:intro}

\hfill

\lettrine[lines=3, loversize=1, nindent=0pt]{\euclidinitials{T}}{}HE arts and ſciences have become ſo extenſive, that to facilitate their acquirement is of as much importance as to extend their boundaries. Illuſtration, if it does not ſhorten the time of ſtudy, will at leaſt make it more agreeable. T\textsc{HIS WORK} has a greater aim than mere illuſtration; we do not introduce colours for the purpoſe of entertainment, or to amuſe \textit{by certain combinations of tint and form}, but to aſſiſt the mind in its reſearches after truth, to increaſe the facilities of inſlruction, and to diffuſe permanent knowledge. If we wanted authorities to prove the importance and uſefulneſs of geometry, we might quote every philoſopher ſince the days of Plato. Among the Greeks, in ancient, as in the ſchool of Peſtalozzi and others in recent times, geometry was adopted as the beſt gymnaſtic of the mind. In fact, Euclid’s Elements have become, by common conſent, the baſis of mathematical ſcience all over the civilized globe. But this will not appear extraordinary, if we conſider that this ſublime ſcience is not only better calculated than any other to call forth the ſpirit of inquiry, to elevate the mind, and to ſtrengthen the reaſoning faculties, but alſo it forms the beſt introduction to moſt of the uſeful and important vocations of human life.  Arithmetic, land-ſurveying, menſuration, engineering, navigation, mechanics, hydroſtatics, pneumatics, optics, phyſical aſtronomy, \&c. are all dependent on the propoſitions of geometry.\\
\hfill\\
Much however depends on the firſt communication of any ſcience to a learner, though the beſt and moſt eaſy methods are ſeldom adopted.  Propoſitions are placed before a ſtudent, who though having a ſufficient underſtanding, is told juſt as much about them on entering at the very threſhold of the ſcience, as gives him a prepoſſeſſion moſt unfavourable to his future ſtudy of this delightful ſubject; or “the formalities and paraphernalia of rigour are ſo oſtentatiouſly put forward, as almoſt to hide the reality. Endleſs and perplexing repetitions, which do not confer greater exactitude on the reaſoning, render the demonſtrations involved and obſcure, and conceal from the view of the ſtudent the conſecution of evidence.”\\
\hfill\\
Thus an averſion is created in the mind of the pupil, and a ſubject ſo calculated to improve the reaſoning powers, and give the habit of cloſe thinking, is degraded by a dry and rigid courſe of inſtruction into an unintereſting exerciſe of the memory. To raiſe the curioſity, and to awaken the liſtleſs and dormant powers of younger minds ſhould be the aim of every teacher; but where examples of excellence are wanting, the attempts to attain it are but few, while eminence excites attention and produces imitation. The object of this Work is to introduce a method of teaching geometry, which has been much approved of by many ſcientific men in this country, as well as in France and America. The plan here adopted forcibly appeals to the eye, the moſt ſenſitive and the moſt comprehenſive of our external organs, and its pre-eminence to imprint it ſubject on the mind is ſupported by the incontrovertible maxim expreſſed in the well known words of Horace:—
\begin{quotation}
    \noindent \small{\textit{Segnius irritant animos demiſſa per aurem\\
            Quàm quæ ſunt oculis ſubjecta fidelibus}}.

    \noindent \small{A feebler impreſs through the ear is made,\\
        Than what is by the faithful eye conveyed}.
\end{quotation}
\vspace{1ex}

All language conſiſts of repreſentative ſigns, and thoſe ſigns are the beſt which effect their purpoſes with the greateſt preciſion and diſpatch. Such for all common purpoſes are the audible ſigns called words, which are ſtill conſidered as audible, whether addreſſed immediately to the ear, or through the medium of letters to the eye. Geometrical diagrams are not ſigns, but the materials of geometrical ſcience, the object of which is to ſhow the relative quantities of their parts by a proceſs of reaſoning called Demonſtration.  This reaſoning has been generally carried on by words, letters, and black or uncoloured diagrams but as the uſe of coloured ſymbols, ſigns, and diagrams in the linear arts and ſciences, renders the proceſs of reaſoning more preciſe, and the attainment more expeditious, they have been in this inſtance accordingly adopted.\\
\hfill\\
Such is the expedition of this enticing mode of communicating knowledge, that the Elements of Euclid can be acquired in leſs than one third the time uſually employed, and the retention by the memory is much more permanent; theſe facts have been aſcertained by numerous experiments made by the inventor, and ſeveral others who have adopted his plans. The particulars of which are few and obvious; the letters annexed to points, lines, or other parts of a diagram are in fact but arbitrary names, and repreſent them in the demonſtration;
\begin{wrapfigure}{r}{0.5\textwidth}
    \centering
    \includesvg[width=175pt]{triangle_abc}
\end{wrapfigure}
inſtead of theſe, the parts being differently coloured, are made to name themſelves, for their forms in correſponding colours represent them in the demonſtration. In order to give a better idea of this ſyſtem, and of the advantages gained by its adoption, let us take a right angled triangle, and expreſs ſome of its properties both by colours and the
method generally employed.
% \begin{center}
%     $\img[0][3][175]{triangle_abc}$
% \end{center}
% \hfill\\
\begin{center}
    \textit{Some of the properties of the right angled triangle ABC, expreſſed by the method generally employed}.\\
    \hfill\\
    \begin{enumerate}
        \item $\img{red_angle_60} \bplus \img{yellow_angle_65} \bplus \img{blue_angle_64} \bequals {\text{\large 2}} \img{yellow_angle_65} \bequals \tworightangles \bperiod$\\ That is, the red angle added to the yellow angle added to the blue angle, equal twice the yellow angle, equal two right angles.
        \item $\img{red_angle_60} \bplus \img{blue_angle_64} \bequals \img{yellow_angle_65} \bperiod$\\ Or in words, the red angle added to the blue angle, equal the yellow angle.
        \item $\img{yellow_angle_65} \bgt \img{red_angle_60}$ or $\img{blue_angle_64} \bperiod$\\ The yellow angle is greater than either the red or blue angle.
        \item $\img{red_angle_60}$ or $\img{blue_angle_64} \blt \img{yellow_angle_65} \bperiod$\\ Either the red or blue angle is leſs than the yellow angle.
        \item $\img{yellow_angle_65}$ minus $\img{blue_angle_64} \bequals \img{red_angle_60} \bperiod$\\ In other terms, the yellow angle made leſs by the blue angle equal the red angle.
        \item $\yellowline^{\text{\large 2}} \bequals \blueline^{\text{\large 2}} \bplus \redline^{\text{\large 2}} \bperiod$\\ That is, the ſquare of the yellow line is equal to the ſum of the ſquares of the blue and red lines.
    \end{enumerate}
\end{center}

In oral demonſtrations we gain with colours this important advantage, the eye and the ear can be addreſſed at the ſame moment, ſo that for teaching geometry, and other linear arts and ſciences, in claſſes, the ſyſtem is the beſt ever propoſed, this is apparent from the examples juſt given.\\
\hfill\\
Whence it is evident that a reference from the text to the diagram is more rapid and ſure, by giving the forms and colours of the parts, or by naming the parts and their colours, than naming the parts and letters on the diagram. Beſides the ſuperior ſimplicity, this ſyſtem is likewiſe conſpicuous for concentration, and wholly excludes the injurious through prevalent practice of allowing the ſtudent to commit the demonſtration to memory; until reaſon, and fact, and proof only make impreſſions on the underſtanding.\\
\hfill\\
Again, when lecturing on the principles or properties of figures, if we mention the colour of the part or parts referred to, as in ſaying, the red angle, the blue line, or lines, \&c. the part or parts thus named will be immediately ſeen by all in the claſs at the ſame inſtant; not ſo if we ſay the angle ABC, the triangle PFQ, the figure EGKt, and ſo on; for the letters muſt be traced one by one before the ſtudents arrange in their minds the particular magnitude referred to, which often occaſions confuſion and error, as well as loſs of time.  Alſo if the parts which are given as equal, have the ſame colours in any diagram, the mind will not wander from the object before it; that is, ſuch an arrangement preſents an ocular demonſtration of the parts to be proved equal, and the learner retains the data throughout the whole of the reaſoning.  But whatever may be the advantages of the preſent plan, if it be not ſubſtituded for, it can always be made a powerful auxiliary to the other methods, for the purpoſe of introduction, or of a more ſpeedy reminiſcence, or of more permanent retention by the memory.\\
\hfill\\
The experience of all who have formed ſyſtems to impreſs facts on the underſtanding, agree in proving that coloured repreſentations, as pictures, cuts, diagrams, \&c. are more eaſily fixed in the mind than mere ſentences unmarked by any peculiarity. Curious as it may appear, poets ſeem to be aware of this fact more than mathematicians; many modern poets allude to this viſible ſyſtem of communicating knowledge, one of them has thus expreſſed himſelf:
\begin{quotation}
    \small{\noindent Sounds which addreſs the ear are loſt and die\\
        In one ſhort hour, but theſe which ſtrike the eye,\\
        Live long upon the mind, the faithful ſight\\
        Engraves the knowledge with a beam of light.}
\end{quotation}

This perhaps may be reckoned the only improvement which plane geometry has received ſince the days of Euclid, and if there were any geometers of note before that time, Euclid’s ſucceſs has quite eclipſed their memory, and even occaſioned all good things of that kind to be aſſigned to him; like Æſop among the writers of Fables. It may alſo be worthy of remark, as tangible diagrams afford the only medium through which geometry and other linear arts and ſciences can be taught to the blind, this viſible ſyſtem is no leſs adapted to the exigencies of the deaf and dumb.\\
\hfill\\
Care muſt be taken to ſhow that colour has nothing to do with the lines, angles, or magnitudes, except merely to name them.  A mathematical line, which is length without breadth, cannot poſſeſs colour, yet the junction of the two colours on the ſame plane gives a good idea of what is meant by a mathematical line; recollect we are ſpeaking familiarly, ſuch a junction is to be underſtood and not the colour, when we ſay the black line, the red line or lines, \&c.\\
\hfill\\
Colours and coloured diagrams may at firſt appear a clumſy method to convey proper notations of the properties and parts of mathematical figures and magnitudes, however they will be found to afford a means more refined and extenſive than any that has been hitherto propoſed.\\
\hfill\\
We ſhall here define a point, a line, and a surface, and demonſtrate a propoſition in order to ſhow the truth of this aſſertion.\\
\hfill\\
A point is that which has poſition, but not magnitude; or a point is poſition only, abſtracted from the conſideration of length, breadth, and thickneſs. Perhaps the following deſcription is better calculated to explain the nature of a mathematical point to thoſe who have not acquired the idea, than the above ſpecious definition.\\
\begin{wrapfigure}{r}{0.4\textwidth}
    \centering
    \includesvg[width=100pt]{circle_divided_into_three_equal_slices}
\end{wrapfigure}
Let three colours meet and cover a portion of the paper, where they meet is not blue, nor is it yellow, nor is it red, as it occupies no portion of the plane, for if it did, it would belong to the blue, the red, or the yellow part;
yet it exiſts, and has poſition without magnitude, ſo that with a little reflection, this junction of three colours on a plane gives a good idea of a mathematical point.\\

A line is length without breadth. With the aſſiſtance of colours, nearly in the ſame manner as before, an idea of a line may be thus given:—\\
\begin{wrapfigure}{l}{0.4\textwidth}
    \centering
    \includesvg[width=100pt]{two_colors_stacked}
\end{wrapfigure}
Let two colours meet and cover a portion of the paper; where they meet is not red, nor is it blue; therefore the junction occupies no portion of the plane, and therefore it cannot have breadth but only length: from which we can readily form an idea of what is meant by a mathematical line. For the purpoſe of illuſtration, one colour differing from the colour of the paper, or plane upon which it is drawn, would have been ſufficient;
\begin{wrapfigure}{l}{0.4\textwidth}
    \centering
    \includesvg[width=80pt]{figure_pq}
\end{wrapfigure}
hence in future, if we ſay the red line, the blue line, or lines, \&c. it is the junctions with the plane upon which they are drawn are to be underſtood.\\

Surface is that which has length and breadth without thickneſs.
When we conſider a ſolid body (PQ), we perceive at once that it has three dimensions, namely:—length, breadth, and thickneſs;  ſuppoſe one part of this ſolid (PS) to be red, and the other part (QR) yellow, and that the colours be diſtinct without commingling, the blue surface (RS) which ſeparates theſe parts, or which is the ſame thing, that which divides the ſolid without loſs of material, muſt be without thickneſs, and only poſſeſſes length and breadth; this plainly appears from reaſoning, ſimilar to that juſt employed in defining, or rather deſcribing a point and a line.\\

\hfill

\begin{wrapfigure}{r}{0.5\textwidth}
    \centering
    \includesvg[width=150pt]{proposition_5_figure}
\end{wrapfigure}
The propoſition which we have ſelected to elucidate the manner in which the principles are applied is the fifth of the firſt Book.  In an iſoſceles triangle ABC, the internal angles at the baſe ABC, ACB are equal, and when the ſides AB, AC are produced, the external angles at the baſe BCE, CBD are alſo equal.\\

% {\raggedright Produce $\redline$ and $\redline \bperiod$\\  Make $\yellowline \bequals \yellowline \bperiod$\\ Draw $\blueline \bequals \blueline$ [\hyperref[book1pr3]{\textsc{I.} 3}]\\ in $\img{left_triangle}$ and $\img{right_triangle}$ we have\\ $\bimg{red_and_yellow_lines} \bequals \bimg{red_and_yellow_lines}$\\}

\begin{center}
    Produce $\redline$ and $\redline \bperiod$\\  Make $\yellowline \bequals \yellowline \bperiod$\\ Draw $\blueline \bequals \blueline$ [\hyperref[book1pr3]{\textsc{I.} 3}]\\ in $\img{left_triangle}$ and $\img{right_triangle}$\\ we have $\bimg{red_and_yellow_lines} \bequals \bimg{red_and_yellow_lines}$\\
    $\redline \bequals \redline$ and $\img[0][0][20]{black_angle}$ common:\\
    $\btherefore \blueline \bequals \blueline \bcomma \img[0][0][20]{left_blue_and_yellow_angles} \bequals \img[0][0][20]{right_blue_and_yellow_angles}$\\
    and $\img{left_red_angle} \bequals \img{right_red_angle}$ [\hyperref[book1pr4]{\textsc{I.} 4}].\\
    Again in $\img[0][0][40]{lower_left_triangle}$ and $\img[0][0][40]{lower_right_triangle} \bcomma$\\
\end{center}

\begin{center}
    $\yellowline \bequals \yellowline \bcomma$\\
    $\blueline \bequals \blueline \bcomma$\\
    and $\img{left_red_angle} \bequals \img{right_red_angle} \bsemicolon$\\
    $\btherefore \img[-0.5]{left_yellow_angle_plus_remaining_angle} \bequals \img[-0.5]{right_yellow_angle_plus_remaining_angle}$\\
    and $\img[-0.5]{left_yellow_angle} \bequals \img[-0.5]{right_yellow_angle}$ [\hyperref[book1pr4]{\textsc{I.} 4}].\\
    But $\img[0][0][20]{left_blue_and_yellow_angles} \bequals \img[0][0][20]{right_blue_and_yellow_angles} \bcomma$\\
    $\btherefore \img{left_blue_angle} \bequals \img{right_blue_angle} \bperiod$\\
    \hfill\\
    \hfill Q.E.D.\\
    \hfill\\
    \textit{By annexing Letters to the Diagram}.
\end{center}

L\textsc{ET} the equal ſides AB and AC be produced through the extremities BC, of the third ſide, and in the produced part BD of either, let any point D be aſſumed, and from the other let AE be cut off equal to AD [\hyperref[book1pr3]{\textsc{I.} 3}]. Let the points E and D, ſo taken in the produced ſides, be connected by ſtraight lines DC and BE with the alternate extremities of the third ſide of the triangle.  In the triangles DAC and EAB the ſides DA and AC are reſpectively equal to EA and AB, and the included angle A is common to both triangles. Hence [\hyperref[book1pr4]{\textsc{I.} 4}] the line DC is equal to BE, the angle ADC to the angle AEB, and the angle ACD to the angle ABE; if from the equal lines AD and AE the equal ſides AB and AC be taken, the remainders BD and CE will be equal. Hence in the triangles BDC and CEB, the ſides BD and DC are reſpectively equal to CE and EB, and the angles D and E included by thoſe ſides are alſo equal. Hence [\hyperref[book1pr4]{\textsc{I.} 4}] the angles DBC and ECB, which are thoſe included by the third ſide BC and the productions of the equal ſides AB and AC are equal.  Alſo the angles DCB and EBC are equal if thoſe equals be taken from the angles DCA and EBA before proved equal, the remainders, which are the angles ABC and ACB oppoſite to the equal ſides, will be equal.

\begin{center}
    \textit{Therefore in an iſoſceles triangle, \&c}. \hfill Q.E.D.
\end{center}

Our object in this place being to introduce the ſyſtem rather than to teach any particular ſet of propoſitions, we have therefore ſelected the foregoing out of the regular courſe. For ſchools and other public places of inſtruction, dyed chalks will anſwer to deſcribe diagrams, \&c. for private uſe coloured pencils will be found very convenient.\\
\hfill\\
We are happy to find that the Elements of Mathematics now forms a conſiderable part of every ſound female education, therefore we call the attention of thoſe intereſted or engaged in the education of ladies to this very attractive mode of communicating knowledge, and to the ſucceeding work for its future development.\\
\hfill\\
We ſhall for the preſent conclude by obſerving, as the ſenſes of ſight and hearing can be ſo forcibly and inſtantaneously addreſſed alike with one thouſand as with one, \textit{the million} might be taught geometry and other branches of mathematics with great eaſe, this would advance the purpoſe of education more than any thing that \textit{might} be named, for it would teach the people how to think, and not what to think; it is in this particular the great error of education originates.